The Sortino ratio measures the risk-adjusted return of an investment asset, portfolio, or strategy.[1] It is a modification of the Sharpe ratio but penalizes only those returns falling below a user-specified target or required rate of return, while the Sharpe ratio penalizes both upside and downside volatility equally. Though both ratios measure an investment's risk-adjusted return, they do so in significantly different ways that will frequently lead to differing conclusions as to the true nature of the investment's return-generating efficiency.

The Sortino ratio is used as a way to compare the risk-adjusted performance of programs with differing risk and return profiles. In general, risk-adjusted returns seek to normalize the risk across programs and then see which has the higher return unit per risk.[2]

Contents  [hide] 
1	Definition
2	Usage
3	See also
4	References
Definition[edit]
The ratio {\displaystyle S} S is calculated as

{\displaystyle S={\frac {R-T}{DR}}} S={\frac  {R-T}{DR}} ,
where {\displaystyle R} R is the asset or portfolio average realized return, {\displaystyle T} T is the target or required rate of return for the investment strategy under consideration (originally called the minimum acceptable return MAR), and {\displaystyle DR} DR is the target semi-deviation (the square root of target semi-variance), termed downside deviation. {\displaystyle S} S is expressed in percentages and therefore allows for rankings in the same way as standard deviation.

An intuitive way to view downside risk is the annualized standard deviation of returns below the target. Another is the square root of the probability-weighted squared below-target returns. The squaring of the below-target returns has the effect of penalizing failures at a quadratic rate. This is consistent with observations made on the behavior of individual decision making under uncertainty.

{\displaystyle DR={\sqrt {\int _{-\infty }^{T}(T-r)^{2}f(r)\,dr}}} DR={\sqrt  {\int _{{-\infty }}^{T}(T-r)^{2}f(r)\,dr}}
Here

{\displaystyle DR} DR = downside deviation or (commonly known in the financial community) "downside risk" (by extension, {\displaystyle DR^{2}} DR^{2} = downside variance),

{\displaystyle T} T = the annual target return, originally termed the minimum acceptable return MAR,

{\displaystyle r} r = the random variable representing the return for the distribution of annual returns {\displaystyle f(r)} f(r), and

{\displaystyle f(r)} f(r) = the distribution for the annual returns, e.g., the three-parameter lognormal distribution.
