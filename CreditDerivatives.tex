\documentclass[]{article}

%opening
\usepackage{amsmath}
\usepackage{amssymb}

\begin{document}


\section{Credit Derivatives }

Credit derivatives are derivative instruments that seek to trade in credit risks. All derivatives have some common features: they are related to some risk or volatility, typically do not require initial investment, and may be net settled. 

For example, the risk or volatility in an inter-rate swap is movements in interest rates. In a commodity derivative, it is commodity prices. Likewise, the subject matter of a credit derivative is the general credit risk of a reference entity. The general credit risk is indicated by the happening of certain events, called credit events, which include bankruptcy, failure to pay, restructuring etc.

\subsection{Contracting Parties}There is a party trying to transfer credit risk, called \textbf{protection buyer}, and the counterparty is trying to acquire credit risk, called Credit derivatives .

\subsection{Hedging}
The primary purpose of credit derivatives must have been to hedge - a bank having exposure in a reference entity seeks to protect itself by buying protection from another. But over time, credit derivatives market has become a trading market. Trades in credit derivatives are taken to be proxies for trades in actual loans or bonds of the reference entity. 

\begin{itemize}
\item For example, a bank willing to acquire exposure in a reference entity X would sell protection referenced to X; while a bank holding a bearish view on X will buy protection. Therefore, credit derivatives trades have become easy tools to replicate a funded cash bond or cash loan of a reference entity, minus all the inflexibilities, lack of availability or regulatory and geographical barriers.

\item Credit derivatives are typically unfunded - the protection seller is not required to put in any money upfront. The protection buyer typically pays a periodic premium. However, the credit derivative may be funded as well - for example, the protection buyer may require the protection seller to pre-pay the entire notional value of the contract upfront. In return, the protection buyer may issue a note, called credit linked note. The credit linked note is similar to any other bond or note, with the difference that from the amount due for repayment, the protection buyer (issuer) may deduct the amount of payments, if any, required on account of credit events.

\item A credit derivative being a derivative, does not require either of the parties - the protection seller or protection buyer - to actually hold the reference asset. Thus, a bank may buy protection for an exposure it has, or does not have, or irrespective of the amount or term for which it has actual exposure. Obviously, therefore, the amount of compensation that can be claimed under a credit derivative is not related to the actual losses suffered by the protection buyer.
\end{itemize}
%=================================================================%
\subsection{Occurence of a Credit Event}
When a credit event takes place, there are two ways of settlement - cash and physical. Cash settlement means the reference asset will be valued, and the difference between its par and fair value will be paid by the protection seller. Physical settlement means the protection seller will acquire the defaulted asset, for its full par.

\end{document}
 
