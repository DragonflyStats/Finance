
Islamic Finance
Islamic Finance is based on interpretations from the Qur'an. Its two central tenets are no interest can be earned on loans and socially responsible investing. The key difference from a financial perspective is the no-interest rule since the Islamic socially responsible investing paradigm is not much different from what other religions do.

The Quran is the holy book of Islam, in it, believers find the verses that forbid Riba (or interest). The book has the 4 key verses upon which the no-interest principle is based. Many of the differences in Islamic finance (especially Islamic banking) revolve around this no interest principle.

For example, Islamic banks must take equity positions in homes rather than taking a traditional mortgage. Others examples include essentially profit sharing plans, leasing, and repurchase plans. These allow the financial institution to make money while satisfying the no-interest principle.

The second difference between Islamic finance and traditional finance is the emphasis on socially responsible investing. While in the Western financial tradition there are many investors who invest in "socially responsible" means, socially responsible investing is not as wide spread as it is within the Islamic tradition.

Islam takes a holistic view of the person. Thus someone who is good does good things. This includes investing responsibly to assure that the money does not go for "bad" purposes. These "bad" purposes include the usual subjects such as drugs, weapons, alcohol, pornography, and of course terrorism. Again this is really no different from traditional socially responsible investing.

Today there are many financial institutions, even in the Western world, offering financial services and products in accordance with the rules of the Islamic finance. For example, legal changes introduced by Chancellor Gordon Brown in 2003, have enabled British banks and building societies to offer so-called Muslim mortgages for house purchase.

Islamic banking refers to a system of banking or banking activity which is consistent with Islamic law (Sharia) principles and guided by Islamic economics. In particular, Islamic law prohibits usury, the collection and payment of interest, also commonly called riba in Islamic discourse. Generally, Islamic law also prohibits trading in financial risk (which is seen as a form of gambling). In addition, Islamic law prohibits investing in businesses that are considered haram (such as businesses that sell alcohol or pork, or businesses that produce un-Islamic media). In the late 20th century, a number of Islamic banks were created, to cater to this particular banking market.

Contents [hide]
1 History of modern Islamic banking 
2 Principles in Islamic Banking 
3 Shariah Advisory Council/Consultant 
4 Concepts In Islamic debt banking 
4.1 Wadiah (Safekeeping) 
4.2 Mudharabah (Profit Loss Sharing) 
4.3 Musharakah (Joint Venture) 
4.4 Murabahah (Cost Plus) 
4.5 Bai' Bithaman Ajil (Deferred Payment Sale) 
4.6 Wakalah (Agency) 
4.7 Qardhul Hassan (Benevolent Loan) 
4.8 Ijarah Thumma Al Bai' (Hire Purchase) 
4.9 Bai' al-Inah (Sell and Buy Back Agreement) 
4.10 Hibah (Gift) 
4.11 Takaful (Islamic Insurance) 
4.12 Sukuk ( Islamic Bonds ) 
4.13 Islamic Equity Funds 
5 Islamic laws on trading 
6 Opposition 
7 See also 

History of modern Islamic banking
The first modern experiment with Islamic banking was undertaken in Egypt under cover, without projecting an Islamic image, for fear of being seen as a manifestation of Islamic fundamentalism which was anathema to the political regime. The pioneering effort, led by Ahmad El Najjar, took the form of a savings bank based on profit-sharing in the Egyptian town of Mit Ghamr in 1963. This experiment lasted until 1967 (Ready 1981), by which time there were nine such banks in the country.[1]


Principles in Islamic Banking
Islamic banking has the same purpose as conventional banking except that it claims to operate in accordance with the rules of Shariah, known as Fiqh al-Muamalat (Islamic rules on transactions). The basic principle of Islamic banking is the sharing of profit and loss and the prohibition of riba´ (interest). Amongst the common Islamic concepts used in Islamic banking are profit sharing (Mudharabah), safekeeping (Wadiah), joint venture (Musharakah), cost plus (Murabahah) and leasing (Ijarah).

In an Islamic mortgage transaction, instead of loaning the buyer money to purchase the item, a bank might buy the item itself from the seller, and re-sell it to the buyer at a profit, while allowing the buyer to pay the bank in installments. However, the fact that it is profit cannot be made explicit and therefore there are no additional penalties for late payment. In order to protect itself against default, the bank asks for strict collateral. The goods or Land is registered to the name of the buyer from the start of the transaction. This arrangement is called Murabaha. Another approach is Ijara wa Iqtina, which is similar to real estate leasing. Islamic banks handle loans for vehicles in a similar way (selling the vehicle at a higher-than-market price to the debtor and then retaining ownership of the vehicle until the loan is paid).

There are several other approaches used in business deals. Islamic banks lend their money to companies by issuing floating rate interest loans. The floating rate of interest is pegged to the company's individual rate of return. Thus the bank's profit on the loan is equal to a certain percentage of the company's profits. Once the principal amount of the loan is repaid, the profit-sharing arrangement is concluded. This practice is called Musharaka. Further, Mudaraba is venture capital funding of an entrepreneur who provides labor while financing is provided by the bank, so that both profit and risk are shared. Such participatory arrangements between capital and labor reflect the Islamic view that the borrower must not bear all the risk/cost of a failure, as it is Allah who determines that failure, and intends that it fall on all those involved.

Last, Islamic banking is restricted to Islamically acceptable deals, which exclude those involving alcohol, pork, gambling, etc. Thus ethical investing is the only acceptable form of investment, and moral purchasing is encouraged.

Islamic banks have grown recently in the Muslim world but are a very small share of the global banking system. Micro-lending institutions such as Grameen Bank use conventional lending practices, and are popular in some Muslim nations, but are clearly not Islamic banking.

In theory, Islamic banking should be synonymous with full-reserve banking, with banks achieving a 100% reserve ratio [2]. However in practice this is rarely the case [3].


Shariah Advisory Council/Consultant
Islamic banks and banking institutions that offer Islamic banking products and services (IBS banks) are required to establish Shariah advisory committees/ consultants to advise them and to ensure that the operations and activities of the bank comply with Shariah principles.

In Malaysia, the National Syariah Advisory Council additionally set up at Bank Negara Malaysia (BNM) advises BNM on the Shariah aspects of the operations of these institutions, as well as on their products and services. (See: Islamic banking in Malaysia)


Concepts In Islamic debt banking

Wadiah (Safekeeping)
In Wadiah, a bank is deemed as a keeper and trustee of funds. A person deposits funds in the bank and the bank guarantees refund of the entire amount of the deposit, or any part of the outstanding amount, when the depositor demands it. The depositor, at the bank's discretion, may be rewarded with a 'hibah' (gift) as a form of appreciation for the use of funds by the bank. In this case, the bank compensates depositors for the time-value of their money (i.e. pays interest) but refers to it as a "gift" because it does not officially guarantee payment of the gift.


Mudharabah (Profit Loss Sharing)
Mudharabah is an arrangement or agreement between a capital provider and an entrepreneur, whereby the entrepreneur can mobilise funds for its business activity. Any profits made will be shared between the capital provider and the entrepreneur according to an agreed ratio, where both parties share in profits and only capital provider bears all the losses if occurred. The profit-sharing continues until the loan is repaid. The bank is compensated for the time value of its money in the form of a floating interest rate that is pegged to the debtor's profits.


Musharakah (Joint Venture)
This concept is normally applied for business partnerships or joint ventures. The profits made are shared on an agreed ratio, while losses incurred will be divided based on the equity participation ratio. This concept is distinct from fixed-income investing (i.e. issuance of loans).


Murabahah (Cost Plus)
This concept refers to the sale of goods at a price, which includes a profit margin agreed to by both parties. The purchase and selling price, other costs and the profit margin must be clearly stated at the time of the sale agreement. The bank is compensated for the time value of its money in the form of the profit margin. This is a fixed-income loan for the purchase of a real asset (such as real estate or a vehicle), with a fixed rate of interest determined by the profit margin. The bank is not compensated for the time value of money outside of the contracted term (i.e. the bank cannot charge additional interest on late payments), however the asset remains in the ownership of the bank until the loan is paid in full.

This type of transaction is similar to "rent-to-own" arrangements for furniture or appliances that are very common in North American stores.


Bai' Bithaman Ajil (Deferred Payment Sale)
This concept refers to the sale of goods on a deferred payment basis at a price, which includes a profit margin agreed to by both parties. This is similar to Murabahah, except that the debtor makes only a single installment, on the maturity date of the loan. By the application of a discount rate, an islamic bank can collect the market rate of interest.


Wakalah (Agency)
This occurs when a person appoints a representative to undertake transactions on his/their behalf, similar to a power of attorney.


Qardhul Hassan (Benevolent Loan)
This is a loan extended on a goodwill basis, and the debtor is only required to repay the amount borrowed. However, the debtor may, at his or her discretion, pay an extra amount beyond the principal amount of the loan (without promising it) as a token of appreciation to the creditor. In the case that the debtor does not pay an extra amount to the creditor, this transaction is a true interest-free loan. Some Muslims consider this to be the only type of loan that does not violate the prohibition on riba, since it is the one type of loan that truly does not compensate the creditor for the time value of money [4].


Ijarah Thumma Al Bai' (Hire Purchase)
These are variations on a theme of purchase and lease back transactions. There are two contracts involved in this concept. The first contract, Ijarah contract (leasing/renting) and the second contract, Bai' contract (purchase) are undertaken one after the other. For example, in a car financing facility, a customer enters into the first contract and leases the car from the owner (bank) at an agreed rental over a specific period. When the lease period expires, the second contract comes into effect, which enables the customer to purchase the car at an agreed price.

In effect, the bank sells the product to the debtor, at an above market-price profit margin, in return for agreeing to receive the payment over a period of time; the profit margin on the lease is equivalent to interest earned at a fixed rate of return.

This type of transaction is particularly reminiscent of contractum trinius, a complicated legal trick used by European bankers and merchants during the Middle Ages, which involved combining three individually legal contracts in order to produce a transaction of an interest bearing loan (something that the Church made illegal).


Bai' al-Inah (Sell and Buy Back Agreement)
The financier sells an asset to the customer on a deferred payment basis and then the asset is immediately repurchased by the financier for cash at a discount. The buying back agreement allows the bank to assume ownership over the asset in order to protect against default without explicitly charging interest in the event of late payments or insolvency.


Hibah (Gift)
This is a token given voluntarily by a debtor to a creditor in return for a loan. Hibah usually arises in practice when Islamic banks voluntarily pay their customers interest on savings account balances.

Takaful (Islamic Insurance)
In modern business, one of the ways to reduce the risk of loss due to misfortunes is through insurance. The basic idea behind insurance is the sharing of risk. The concept of insurance where resources are pooled to help the needy does not contradict Shariah.

Conventional insurance involves the elements of uncertainty (Al-gharar) in the contract of insurance, gambling (Al-maisir) as the consequences of the presence of uncertainty and interest (Al-riba) in the investment activities of the conventional insurance companies which contravene the rules of Shariah. It is generally accepted by Muslim Jurists that the operation of conventional insurance does not conform to the rules and requirements of Shariah.

Takaful is an alternative form of cover which a Muslim can avail himself against the risk of loss due to misfortunes. The concept of takaful is not a new concept, in fact it had been practised by the Muhajrin of Mecca and the Ansar of Medina following the hijra of the Prophet over 1400 years ago.

Takaful is based on the idea that what is uncertain with respect to an individual may cease to be uncertain with respect to a very large number of similar individuals. Insurance by combining the risks of many people enables each individual to enjoy the advantage provided by the law of large numbers.


Sukuk ( Islamic Bonds )
Islamic Equity Funds
Islamic investment equity funds market is one of the fastest growing sectors within the Islamic financial system. Currently there are approximately 100 Islamic equity funds worldwide. The total assets managed through these funds currently exceed US$5 billion and is growing by 12-15% per annum. With the continuous interest in the Islamic financial system, there are positive signs that more funds will be launched. Some western majors have just joined the fray or are thinking of launching similar Islamic equity products.

Despite these successes, this market has seen a record of poor marketing as emphasis is on products and not on addressing the needs of investors. Over the last few years, quite a number of funds have closed down.

Most of the funds tend to target high net worth individuals and corporate institutions, minimum investments ranging from US$50,000 to as high as US$1,000,000.

Target markets for Islamic funds vary, some cater for their local markets e.g Malaysia and Gulf based investment funds. Others clearly target the Middle East and Gulf regions, neglecting local markets and have been accused of failing to serve Muslim communities.

Since the launch of Islamic equity funds in the early 90's, we have seen the establishment of credible equity benchmarks by Dow Jones Islamic market index and the FTSE Global Islamic Index Series. The website failaka.com monitors the performance of Islamic equity funds and provide a comprehensive list of the Islamic funds worldwide.

Islamic laws on trading
The Qur'an prohibits gambling (games of chance involving money). The hadith literature, in addition to prohibiting gambling (games of chance), also prohibits bayu al-gharar (trading in risk, where the Arabic word gharar is taken to mean "risk").

The Hanafi madhab (legal school) in Islam defines gharar as "that whose consequences are hidden." The Shafi legal school defined gharar as "that whose nature and consequences are hidden" or "that which admits two possibilities, with the less desirable one being more likely." The Hanbali school defined it as "that whose consequences are unknown" or "that which is undeliverable, whether it exists or not." Ibn Hazm of the Zahiri school wrote, "gharar is where the buyer does not know what he bought, or the seller does not know what he sold.” The modern scholar of Islam, Professor Mustafa Al-Zarqa wrote that, "gharar is the sale of probable items whose existence or characteristics are not certain, due to the risky nature which makes the trade similar to gambling". There are a number of hadith forbid trading in gharar (risk), often giving specific examples of gharhar transactions (e.g. selling "the birds in the sky or the fish in the water", "the catch of the diver", "an unborn calf in its mother’s womb", "the sperm and/or unfertilized eggs of camels", etc.). Jurists have sought many complete definitions of the term. Thye also came up with the concept of yasir (minor risk); a financial transaction with a minor risk is deemed to be halal (permissible) while trading in non-minor risk (bayu al-ghasar) is deemed to be haram. 

What gharar is, exactly, was never fully decided upon by the Muslim jurists. This was mainly due to the complication of having to decide what is and is not a "minor risk." Derivatives instruments (such as stock options) have only become common relatively recently . Some Islamic banks do provide brokerage services for stock trading and perhaps even for derivatives trading.


Opposition
Islamabad, Pakistan, June 16, 2004: Members of leading Islamist political party in Pakistan, the Muttahida Majlis-e-Amal (MMA) party, staged a protest walkout from the National Assembly of Pakistan against what they termed derogatory remarks by a minority member on interest banking: "Taking part in the budget debate, M.P. Bhindara, a minority MNA [Member of the National Assembly]... referred to a decree by an Al-Azhar University's scholar that bank interest was not un-Islamic. He said without interest the country could not get foreign loans and could not achieve the desired progress. A pandemonium broke out in the house over his remarks as a number of MMA members ... rose from their seats in protest and tried to respond to Mr Bhandara's observations. However, they were not allowed to speak on a point of order which led to their walkout ... Later, the opposition members were persuaded by a team of ministers ... to return to the house ... the government team accepted the right of the MMA to respond to the minority member's remarks ... Sahibzada Fazal Karim said the Council of Islamic ideology had decreed that interest in all its forms was 'haram' in an Islamic society. Hence, he said, no member had the right to negate this settled issue." [6] 
Some Muslims have opposed these Islamic banks, claiming that they do deal in interest but merely conceal it through legal tricks. Indeed, from an economic perspective, Islamic banks do compensate and charge for the time value of money, thus paying and receiving what is known in economics as "interest." Such Muslims compare Islamic banking to contractum trinius - a legal trick devised by European bankers and merchants during the Middle Ages, designed to facilitate the borrowing of money at a fixed rate of interest (something that the Church fiercely opposed) through combining three different contractual agreements which in and of themselves were not prohibited by the Church. While Islamic law prohibits the collection of interest, it does allow a seller to resell an item at a higher price than it was bought for, as long as there are clearly two transactions. 

Contractum trinius

A contractum trinius was a set of contracts devised by European bankers and merchants in the Middle Ages as a method of circumventing Canon law edicts prohibiting usury. At the time, most Christian nations heavily incorporated scripture into their laws, and as such it was illegal for any person to charge interest on a loan of money.

To get around this, a set of three separate contracts were presented to someone seeking a loan: an investment, a sale of profit and an insurance contract. Each of these contracts were permissible under Church law, but together replicated the effect of an interest-paying loan.

The way this procedure worked was as follows: The loaner would invest a sum equal to the amount to be borrowed to the loanee for one year. The loaner would then purchase insurance for the investment from the loanee, and finally sell back the right to any profit made over a pre-arranged percentage of the investment to them. This system replicated the effects of a loan with any interest rate agreed between the two, and provided protection to the loaner against defaulting, while the loanee remained under the protection of the law when it came to collection of the money by threats or force (loan sharking).

The Church proved utterly unable to legislate against the contractum trinius, and the idea quickly spread to merchants and bankers across Christendom. It helped in part to improve public perception of the practice of usury by moneylenders, and ultimately the doctrine was rewritten by the School of Salamanca, and the ban overturned in many Protestant countries, starting with England by Henry VIII.

There is a minority view that the present practice of Islamic banking relies on devices similar to the contractum trinius as a means of working around a ban of riba (usury) in religious scripture.

