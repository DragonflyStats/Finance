1.1 Nominal rates of interest
Consider transactions for a term of length h time units, where h > 0 and h need not be an integer. We define ih(t), the nominal rate of interest per unit time on transactions of term h
beginning at time t, to be such that the effective rate of interest for the period of length h beginning at time t is hih(t).
 
Thus, if the sum of C is invested at time t for a term h, the sum to be received at time t + h is, by definition,
C[1 + hih(t)](1.1.1)


If h = 1, the nominal rate of interest coincides with the effective rate of interest for the period t to t + 1, so
i1(t) = i(t) (1.1.2)
 
In many practical applications ih(t) does not depend on t, in which case we may write
ih(t) = ih for all t (1.1.3)
