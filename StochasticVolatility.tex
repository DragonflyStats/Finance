Stochastic volatility models are those in which the variance of a stochastic process is itself randomly distributed.[1] They are used in the field of mathematical finance to evaluate derivative securities, such as options. The name derives from the models' treatment of the underlying security's volatility as a random process, governed by state variables such as the price level of the underlying security, the tendency of volatility to revert to some long-run mean value, and the variance of the volatility process itself, among others.

Stochastic volatility models are one approach to resolve a shortcoming of the Black–Scholes model. In particular, models based on Black-Scholes assume that the underlying volatility is constant over the life of the derivative, and unaffected by the changes in the price level of the underlying security. However, these models cannot explain long-observed features of the implied volatility surface such as volatility smile and skew, which indicate that implied volatility does tend to vary with respect to strike price and expiry. By assuming that the volatility of the underlying price is a stochastic process rather than a constant, it becomes possible to model derivatives more accurately.
