CQF 2 : Risk and Return 
This unit deals with the classical portfolio theory of Markowitz, the Capital Asset Pricing Model, more recent developments of these theories, also option types and strategies.
 
•Simulations: The lognormal random walk, probability density functions.
•Risk and reward: Measuring return, expectation and standard deviation.
•Modern Portfolio Theory (Markowitz): Expected returns, variances and covariances, benefits of diversification, the opportunity set and the efficient frontier, the Sharpe ratio, and utility functions.
•Capital Asset Pricing Model: Single-index model, beta, diversification, optimal portfolios, the multiindexmodel.
•Value at risk: Profit and loss for simple portfolios, tails of distributions, Monte Carlo simulations and historical simulations, stress testing and worst-case scenarios. Portfolios of derivatives.
•Introducing futures, forwards and options: Simple contingent claims, definitions and uses.
•Review of option strategies: Building up special payoff structures using vanilla calls and puts, horizontal, vertical and diagonal spreads.
•Review of options as speculative investments: Taking a view, gearing, strategies that benefit from moves in the asset or in volatility.
•The binomial model: Up and down moves, delta hedging and self-financing replication, no arbitrage, a pricing model, risk-neutral probabilities.
•Martingale theory: Fundamental definitions, concepts, results and tools.
 

Lecture 2.1
Lecture 2.2
Lecture 2.3
Lecture 2.4
Lecture 2.5
Lecture 2.6: Methods for Quantitative Finance: II
Lecture 2.1
Measuring risk and return
Benefits of diversification
Modern Portfolio Theory and the Capital Asset Pricing Model
The efficient frontier
Optimizing your portfolio
How to analyze portfolio performance
Alphas and betas
Lecture 2.2
The time value of money
Equities, commodities, currencies and indices
Fixed and floating interest rates
Futures and forwards
No-arbitrage
The definitions of basic derivative instruments
Option jargon
No arbitrage again and put-call parity
How to draw payoff diagrams
Simple option strategies
Lecture 2.3
The meaning of Value at Risk (VaR)
How VaR is calculated in practice
Simulations and bootstrapping
Simple volatility estimates
The exponentially weighted moving average
Lecture 2.4
A simple model for an asset price random walk
Delta hedging
No arbitrage
The basics of the binomial method for valuing options
Risk neutrality
Lecture 2.5
Martingale definitions and concepts
Important results and tools
Lecture 2.6: Methods for Quantitative Finance: II
Numerical Methods
Stochastic Calculus
