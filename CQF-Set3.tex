 
The Black-Scholes theory, built on the principles of delta-hedging and no arbitrage, has been very successful and fruitful as a theoretical model and in practice. The theory and results are explained using different kinds of mathematics to make the student familiar with techniques in current use.
 
•The Black-Scholes model: A stochastic differential equation for an asset price, the delta-hedged portfolio and self-financing repliction, no arbitrage, the pricing partial differential equation and simple solutions.
•The greeks: delta, gamma, theta, vega and rho and their uses in hedging.
•Risk-neutrality: Fair value of an option as an expectation with respect to a risk-neutral density function.
•Early exercise: American options, elimination of arbitrage, modifying the binomial method, gradient conditions, formulation as a free boundary problem.
•Elementary numerical analysis: Monte Carlo simulation and the explicit finite-difference method.
•Value at Risk: Portfolios of derivatives.
•Martingales: The probabilistic mathematics used in derivatives theory
 

Lecture 3.1
Lecture 3.2
Lecture 3.3
Lecture 3.4
Lecture 3.5
 
Lecture 3.1
The assumptions that go into the Black-Scholes equation
Foundations of options theory: delta hedging and no arbitrage
The Black-Scholes partial differential equation
Modifying the equation for commodity and currency options
The Black-Scholes formulae for calls, puts and simple digitals
The meaning and importance of the Greeks, delta, gamma, theta, vega and rho
American options and early exercise
Relationship between option values and expectations
Lecture 3.2
The Greeks in detail
Delta, gamma, theta, vega and rho
Higher-order Greeks
How traders use the Greeks
Lecture 3.3
The justification for pricing by Monte Carlo simulation
Grids and discretization of derivatives
The explicit finite-difference method
Lecture 3.4
The many types of volatility
What the market prices of options tells us about volatility
The term structure of volatility
Volatility skews and smiles
Exploiting your volatility models
Should you hedge using implied or actual volatility?
Lecture 3.5
Martingale theory and its relevance to pricing
Its role in practice
Examples
