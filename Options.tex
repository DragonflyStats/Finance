\documentclass[11pt]{article} % use larger type; default would be 10pt
\usepackage[utf8]{inputenc} % set input encoding (not needed with XeLaTeX)
\usepackage{framed}
\usepackage{geometry} % to change the page dimensions
\geometry{a4paper} % or letterpaper (US) or a5paper or....



\usepackage{booktabs} % for much better looking tables
\usepackage{array} % for better arrays (eg matrices) in maths
\usepackage{paralist} % very flexible & customisable lists (eg. enumerate/itemize, etc.)
\usepackage{verbatim} % adds environment for commenting out blocks of text & for better verbatim
\usepackage{subfig} % make it possible to include more than one captioned figure/table in a 
\usepackage{fancyhdr} % This should be set AFTER setting up the page geometry
\pagestyle{fancy} % options: empty , plain , fancy
\renewcommand{\headrulewidth}{0pt} % customise the layout...
\lhead{FERM}\chead{}\rhead{Definitions}
\lfoot{}\cfoot{\thepage}\rfoot{}
\usepackage{sectsty}
\allsectionsfont{\sffamily\mdseries\upshape} % (See the fntguide.pdf for font help)
% (This matches ConTeXt defaults)

%%% ToC (table of contents) APPEARANCE
\usepackage[nottoc,notlof,notlot]{tocbibind} % Put the bibliography in the ToC
\usepackage[titles,subfigure]{tocloft} % Alter the style of the Table of Contents
\renewcommand{\cftsecfont}{\rmfamily\mdseries\upshape}
\renewcommand{\cftsecpagefont}{\rmfamily\mdseries\upshape} % No bold!


\title{Financial Engineering and Risk Management}
\author{Options - Revision Notes}
\begin{document}
\maketitle
\tableofcontents


\section{Binomial Option Pricing Model}

This is an options valuation method developed by Cox, et al, in 1979. The binomial option pricing model uses an iterative procedure, allowing for the specification of nodes, or points in time, during the time span between the valuation date and the option's expiration date. 
\newline

The model reduces possibilities of price changes, removes the possibility for arbitrage, assumes a perfectly efficient market, and shortens the duration of the option. Under these simplifications, it is able to provide a mathematical valuation of the option at each point in time specified.

\end{document}
