In finance, Jensen's alpha[1] (or Jensen's Performance Index, ex-post alpha) is used to determine the abnormal return of a security or portfolio of securities over the theoretical expected return. It is a version of the standard alpha based on a theoretical performance index instead of a market index.

The security could be any asset, such as stocks, bonds, or derivatives. The theoretical return is predicted by a market model, most commonly the capital asset pricing model (CAPM). The market model uses statistical methods to predict the appropriate risk-adjusted return of an asset. The CAPM for instance uses beta as a multiplier.


History[edit]
Jensen's alpha was first used as a measure in the evaluation of mutual fund managers by Michael Jensen in 1968.[2] The CAPM return is supposed to be 'risk adjusted', which means it takes account of the relative riskiness of the asset.

This is based on the concept that riskier assets should have higher expected returns than less risky assets. If an asset's return is even higher than the risk adjusted return, that asset is said to have "positive alpha" or "abnormal returns". Investors are constantly seeking investments that have higher alpha.

Since Eugene Fama, many academics believe financial markets are too efficient to allow for repeatedly earning positive Alpha, unless by chance. Nevertheless, Alpha is still widely used to evaluate mutual fund and portfolio manager performance, often in conjunction with the Sharpe ratio and the Treynor ratio.

Calculation[edit]
In the context of CAPM, calculating alpha requires the following inputs:

the realized return (on the portfolio),
the market return,
the risk-free rate of return, and
the beta of the portfolio.
Jensen's alpha = Portfolio Return − [Risk Free Rate + Portfolio Beta * (Market Return − Risk Free Rate)]

{\displaystyle \alpha _{J}=R_{i}-[R_{f}+\beta _{iM}\cdot (R_{M}-R_{f})]} \alpha _{J}=R_{i}-[R_{f}+\beta _{{iM}}\cdot (R_{M}-R_{f})]
An additional way of understanding the definition can be obtained by rewriting it as:

{\displaystyle \alpha _{J}=(R_{i}-R_{f})-\beta _{iM}\cdot (R_{M}-R_{f})} \alpha _{J}=(R_{i}-R_{f})-\beta _{{iM}}\cdot (R_{M}-R_{f})
If we define the excess return of the fund (market) over the risk free return as {\displaystyle \Delta _{R}\equiv (R_{i}-R_{f})} \Delta _{R}\equiv (R_{i}-R_{f}) and {\displaystyle \Delta _{M}\equiv (R_{M}-R_{f})} \Delta _{M}\equiv (R_{M}-R_{f}) then Jensen's alpha can be expressed as:

{\displaystyle \alpha _{J}=\Delta _{R}-\beta _{iM}\Delta _{M}} \alpha _{J}=\Delta _{R}-\beta _{{iM}}\Delta _{M}
