Vasicek model
From Wikipedia, the free encyclopedia

A trajectory of the short rate and the corresponding yield curves at T=0 (purple) and two later points in time
In finance, the Vasicek model is a mathematical model describing the evolution of interest rates. It is a type of one-factor short rate model as it describes interest rate movements as driven by only one source of market risk. The model can be used in the valuation of interest rate derivatives, and has also been adapted for credit markets. It was introduced in 1977 by Oldřich Vašíček[1] and can be also seen as a stochastic investment model.

Contents  [hide] 
1	Details
2	Discussion
3	Asymptotic mean and variance
4	See also
5	References
6	External links
Details[edit]
The model specifies that the instantaneous interest rate follows the stochastic differential equation:

{\displaystyle dr_{t}=a(b-r_{t})\,dt+\sigma \,dW_{t}} dr_{t}=a(b-r_{t})\,dt+\sigma \,dW_{t}
where Wt is a Wiener process under the risk neutral framework modelling the random market risk factor, in that it models the continuous inflow of randomness into the system. The standard deviation parameter, {\displaystyle \sigma } \sigma , determines the volatility of the interest rate and in a way characterizes the amplitude of the instantaneous randomness inflow. The typical parameters {\displaystyle b,a} b,a and {\displaystyle \sigma } \sigma , together with the initial condition {\displaystyle r_{0}} r_{0}, completely characterize the dynamics, and can be quickly characterized as follows, assuming {\displaystyle a} a to be non-negative:

{\displaystyle b} b: "long term mean level". All future trajectories of {\displaystyle r} r will evolve around a mean level b in the long run;
{\displaystyle a} a: "speed of reversion". {\displaystyle a} a characterizes the velocity at which such trajectories will regroup around {\displaystyle b} b in time;
{\displaystyle \sigma } \sigma : "instantaneous volatility", measures instant by instant the amplitude of randomness entering the system. Higher {\displaystyle \sigma } \sigma  implies more randomness
The following derived quantity is also of interest,

{\displaystyle {\sigma ^{2}}/(2a)} {\sigma ^{2}}/(2a): "long term variance". All future trajectories of {\displaystyle r} r will regroup around the long term mean with such variance after a long time.
{\displaystyle a} a and {\displaystyle \sigma } \sigma  tend to oppose each other: increasing {\displaystyle \sigma } \sigma  increases the amount of randomness entering the system, but at the same time increasing {\displaystyle a} a amounts to increasing the speed at which the system will stabilize statistically around the long term mean {\displaystyle b} b with a corridor of variance determined also by {\displaystyle a} a. This is clear when looking at the long term variance,

{\displaystyle {\frac {\sigma ^{2}}{2a}}} {\frac  {\sigma ^{2}}{2a}}
which increases with {\displaystyle \sigma } \sigma  but decreases with {\displaystyle a} a.

This model is an Ornstein–Uhlenbeck stochastic process. Making the long term mean stochastic to another SDE is a simplified version of the cointelation SDE.[2]

Discussion[edit]
