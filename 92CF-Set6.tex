

Section 6: : The choice of corporate capital structure
The Modigliani-Miller theorem


Section 6:  The choice of corporate capital structure

Basic features of debt and equity 
The Modigliani–Miller theorem
Modigliani–Miller and Black–Scholes
Modigliani–Miller and corporate taxation
Modigliani–Miller with corporate and personal taxation 

The Modigliani-Miller theorem
The Modigliani-Miller theorem, proposed by Franco Modigliani and Merton Miller, forms the basis for modern thinking on capital structure, though it is generally viewed as a purely theoretical result since it disregards many important factors in the capital structure decision. The theorem states that, in a perfect market, how a firm is financed is irrelevant to its value. This result provides the base with which to examine real world reasons why capital structure is relevant, that is, a company's value is affected by the capital structure it employs. 

The basic theorem states that, under a certain market price process (the classical random walk), in the absence of taxes, bankruptcy costs, agency costs, and asymmetric information, and in an efficient market, the value of a firm is unaffected by how that firm is financed. It does not matter if the firm's capital is raised by issuing stock or selling debt. It does not matter what the firm's dividend policy is. Therefore, the Modigliani–Miller theorem is also often called the capital structure irrelevance principle.

Some other reasons include bankruptcy costs, agency costs, taxes, and information asymmetry. This analysis can then be extended to look at whether there is in fact an optimal capital structure: the one which maximizes the value of the firm.


Modigliani and Miller’s argument on capital structure with corporate tax. 
The key issues are that:
interest on debt is tax deductible and hence a levered firm will increase its after tax cash flows to investors 
the higher the level of debt, the higher the interest on debt which attracts a higher tax saving 
the value of a levered firm is higher than the unlevered firm by the present value of the tax shield.
Firms should therefore gear up as much as possible to take advantage of the tax shield.



