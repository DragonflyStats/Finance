
\section{PRMIA 3.0 Capital Allocation and RAPM}
%--------------------------------------------------------------------------------%
Learning Outcome Statement

The candidate should be able to:

\begin{itemize}
\item  Describe the Role of Capital in a Financial Institution
\item  Define and Describe the different types of capital
\item  Demonstrate Economic Capital
\item  Describe the different approaches to calculating Economic Capital
\item  Describe Regulatory Capital
\item  Explain the Basel Norms
\item  Explain the Derivation of Regulatory Capital
\item  Explain Capital Allocation
\item  Demonstrate the Risk Contribution Methodologies for Economic Capital Allocation
\item  Explain Risk Adjusted Performance Measurement (RAPM)
\item  Demonstrate Risk Adjusted Return On Capital (RAROC)
\end{itemize}
%%--------------------------------------------------------------------------------


1. Introduction

2. Economic Capital

3. Regulatory Capital

4. Capital Allocation and Risk Contribution

5. RAROC and Risk adjusted performance

6. Summary and conclusion

Risk adjusted return on capital (RAROC) is a risk-based profitability measurement framework for analysing risk-adjusted financial performance and providing a consistent view of profitability across businesses. 


Risk-adjusted return on capital (RAROC) gives decision makers the ability to compare the returns on several different projects with varying risk levels.


RAROC was popularized by Bankers Trust in the 1980s as an adjustment to simple return on capital (ROC).



--------------------------------------------------------------------------------
\subsection{3.0.1 Introduction}




--------------------------------------------------------------------------------
\subsection{3.0.2 Economic Capital}

Economic capital is the amount of risk capital, assessed on a realistic basis, which a firm requires to cover the risks that it is running or collecting as a going concern, such as market risk, credit risk, and operational risk. It is the amount of money which is needed to secure survival in a worst case scenario.


Section 3 : The bottom-up approach to calculating EC

Section 4 : Stress testing of portfolio losses and Economic Capital

Section 5 : Entreprise Capital practices - Aggregation



--------------------------------------------------------------------------------
\subsection{3.0.3 Regulatory Capital}

Regulatory capital is the mandatory capital the regulators require to be maintained by financial institutions.


Section 3. Basel I regulation.


Market Risk Capital


Cook ratio


Section 4.  Basel II accord


The Basel II accord consists of three pillars.
1.
Minimum capital requirements

2.
supervisory review

3.
market discipline



--------------------------------------------------------------------------------
\subsection{3.0.4 Capital Allocation and Risk Contribution}


1) Stand Alone EC contributions

2) Marginal EC contributions

3) Incremental EC contribution


Additive decomposition of EC is of the form

    EC =iECi



%--------------------------------------------------------------------------------
\subsection{3.0.5 RAROC and Risk adjusted performance}


PRMIA 3.0.5.2.
Mechanics of RAROC

All Models follow the simplifed general formula
\[
RAROC = \frac{\mbox{Revenues} - \mbox{Costs} - \mbox{Expected Losses} }{\mbox{capital}}
\]

\begin{itemize}
\item Revenues include all the nominal returns on assets
\item Costs included all returns to the liability holders of the bank
\item Expected Losses should be determined by Risk Asssessment
\end{itemize}


Section 1: Objectives of RAPM


RAPM: Risk adjuested performance measurement.


Section 2:  Mechanic of RAROC


Simplified General Formula


RAROC=revenues - costs - expected lossescapital

%--------------------------------------------------------------------------------
\subsection{3.0.6 Summary and Conclusions}


Aside from ownership issues, the primary goal of cpaital in a firm is to act as a buffer against unexpected losses.


Three types of capital

1) actual physcial capital

2) Economic Capital

3) regulatory capital
