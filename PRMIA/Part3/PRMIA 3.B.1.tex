
 3.B.1

Introduction to Credit Risk management
Learning Outcome Statement
The candidate should be able to:
 Describe the responsibilities of a credit risk manager
 Describe the Review of Strategic Credit Positions
 Describe Credit Limits and Provisions
 Explain Credit Exposure Measurement Issues
 Demonstrate Credit Risk Reporting
 Describe Stress and Scenario Analysis
 Describe Provisioning
 Describe Documentation
 Describe Credit Protection
 Describe Annual tasks of the credit officer

3.B.2

Define Default Risk
 Define Exposure, Default and Recovery Processes
 Explain the Credit Loss Distribution
 Explain Expected and Unexpected Loss
 Describe Recovery Rates
 Discuss use of beta distribution in credit risk modeling

Introduction
what is default risk
Exposure DEfault and recovery processes
The credit Loss Distribution
Expected and Unexpected Losses
Recovery Rates
Conclusion


Investors typically utilize two approaches to measure default Risk

1) judge the financial characteristics of the securities and their issuers.

This is generally done by analysis the issuers financial statements, particularly financial ratios.

2) use credit ratings

	obtained from agencies such as Moody's and Standard and Poor.
The bond issuers pay the rating agency to evaluate the quality of their bond issue in order to increase the information flow, and hence increase demand for their bond.
The bond agencies determine the appropriate bond rating by assessing various factors.

The bond rating assigned to a bond issue directly affects its yield. The higher the rating, the lower the debt yield.

3.B.3 Credit Exposure
 Define Pre-settlement Risk
 Define Settlement Risk
 Demonstrate Exposure Profiles of Standard Debt Obligations
 Demonstrate Exposure Profiles of Derivatives
 Explain Mitigation of Exposures

1. Introduction
2. Pre-settlement Risk V Settlement Risk
	Pre Settlement Risk
	Settlement Risk	
3. Exposure Profiles
	Exposure Profiles of Standard Debt obligations
	Exposure Profiles of Derivatives
4. Migrations of Exposures
 	Netting Agreements
	Collateral
	Other Counterparty risk mitigation insruments


3.B.4 Default and Credit Migration
Learning Outcome Statement
The candidate should be able to:
 Define and Discuss Default Probabilities and Term Structures of
Default Rates
 Define Credit Ratings
 Demonstrate Measurement of Rating Accuracy
 Describe the Methodology of Credit Rating followed by Rating Agencies
 Demonstrate Transition Matrices, Default Probabilities and Credit Migration
as done by Rating Agencies
 Explain Credit Scoring
 Discuss the Estimation of the Probability of Default
 Demonstrate Market-Implied Default Probabilities
 Explain Credit Rating and Credit Spreads

1. Default Probabilities and Term structure of DEfault Rates
2. Credit Ratings
3. Agency Rating
4. Credit Scoring and Internal Rating Models
5. Market Im plied Default Probabilities
6. Credit Ratings and Credit Spreads
7. Conclusion

3.B.5 Portfolio Models of Credit Loss

Learning Outcome Statement
The candidate should be able to:
 Define Default
 Describe new approaches to Credit Risk Modelling
 Explain Credit VaR
 Define Credit Migration
 Describe the Credit Metrics Framework
 Demonstrate Credit VaR for a single Bond/Loan
 Demonstrate the Estimation of Default and Rating Changes Correlations
 Describe the Credit VaR approach of a Bond/Loan Portfolio
 Explain the Conditional Transition Probabilities – CreditPortfolioView Model
 Explain the idea of contingent claim approach in credit risk measurement
 Demonstrate Structural Model of Default Risk: Merton’s (1974) Model
 Demonstrate Estimation of Credit Risk as a function of Equity Value
 Demonstrate the KMV approach
 Demonstrate the Actuarial Approach

1. Introduction

2. What actually drives Credit Risk at the Portfolio Level

3. Credit Migration Framework
 1) Credit VaR for a single bond/loan
 2) Estimation of default and rating changes correlation
 3) Credit VaR of a bond/loan portfolio

4. Conditional Transition Probabilities CreditPortfolioView

5. Contigent Claim approach to measuring credit risk
 1) Structural Model of Default Risk
 2) Estimating Credit Risk as a function of equity value

6. The KMV approach
	1. Estimation of the Asset value VA and the volatility of Asset return
	2. Calculation of the Distance to Default
	3. Derivation of the probability of default from the Distance to Default.
	4. EDF as a predictor of default
7. The Actuarial approach

8. Summary and Conclusion





3.B.6 Credit Risk Capital Calculation

Learning Outcome Statement
The candidate should be able to:
 Explain the calculation of Economic Credit Capital using Credit
Portfolio Models
 Demonstrate Minimum Credit Capital Requirements under Basel I
 List the Weaknesses of the Basel I Accord for Credit Risk
 Explain the Latest proposal for Minimum Credit Capital requirements
 Describe the Standardised Approach in Basel II
 Describe the Internal Ratings Based Approach (IRB) for Corporate, Bank and
Sovereign Exposures
 Describe the Internal Ratings Based Approach (IRB) for Retail Exposures
 Describe the Internal Ratings Based Approach (IRB) for SME Exposures
 Describe the Internal Ratings Based Approach (IRB) for Specialised Lending
and Equity Exposures
 List the new components of Pillar II for credit risk
 Explain Credit Model Estimation and Validation in Basel II
 Describe Securitisation in Basel II
 Describe the application of credit risk contribution methodologies for
Economic Credit Capital Allocation
 Demonstrate the Shortcomings of VaR for Economic Credit Capital and
Coherent Risk Measures


1. Introduction
2. Economic Credit Capital Calculation
3. Regulatory Credit Capital Basel I
4. Regulatory Credit Capital Basel II
5. Basel II: Credit Model Estimation and Validation
6. Basel II: Securitisation
7. Advanced Topics on Economic Capital Credit
8. Summary and Conclusions


PRMIA III.B.1 Credit Risk Management

 

 



III.B.1.2.4 Credit Risk Reporting

III.B.1.2.5 Stress and Scenario analysis

III.B.1.2.6 Provisioning

III.B.1.2.7 Documentation

III.B.1.2.8 Credit Protection


III.B.1.2.4 Credit Risk Reporting

variables: sector/industry, risk factor, country

credit risk concentrations.

 
•
Time evolution of credit exposures

•
Arrow plots

•
Watch lists.


III.B.1.2.5 Stress and Scenario analysis

Noloss's Stress testing schedule
•
   Historical stress scnarios

•
   Potential stress scenarios

•
   Testing the models.


 

III.B.1.2.6 Provisioning 

 

III.B.1.2.7 Documentation

III.B.1.2.8 Credit Protection



--------------------------------------------------------------------------------
 



