\section*{Stress and Scenario Testing}
\begin{itemize}
\item Board of Governors of the Federal Reserve System – The Supervisory Capital Assessment Program;
\item Financial Services Authority (FSA) – Stress and scenario testing; 
\item Basel Committee on Banking Supervision (BCBS) – Principles for sound stress testing practices and supervision. 
\end{itemize}
They contain guidelines and recommendations for an effective and firm-wide stress, and scenario, testing regime.

%============================================================================%
\subsection*{Learning Outcome Statement}
The candidate should be able to:
\subsubsection*{(Federal Reserve)}
\begin{itemize}
\item  State clearly the need for, conditions governing, and outcomes of the Supervisory Capital Assessment Program (SCAP)
\item   Demonstrate an understanding of the processes involved in SCAP loss and resource projections
\item   Demonstrate an understanding of the SCAP capital buffer
\item   Describe the calculation of additional capital to build a SCAP buffer
\item   Identify the indicated additional capital buffer under SCAP
\end{itemize}
%============================================================================%
\subsubsection*{(FSA Paper)}
\begin{itemize}
\item  Describe the components of the FSA proposed changes to reverse stress testing
\item  Describe the clarifications to Pillar I and II proposed by the FSA
\end{itemize}
%============================================================================%
\subsubsection*{(BCBS paper)}
\begin{itemize}
\item Explain the findings of the BCBS relative to the performance of stress testing during the crisis
\item Describe the 15 recommendations made to banks made by the BCBS
\item Describe the 6 recommendations made to supervisors by the BCBS
\end{itemize}
