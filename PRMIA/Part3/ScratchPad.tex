


ARCH processes

Computation of VaR

VaR Example

Distance to Default

Delta equivalence positions

Altman's Z Score

Loss Databases

Marginal VaR

Incremental VaR

Component VaR

Actuarial Methods

Marginal and Cumulative Default Risk

 



ARCH processes

Autoregressive conditional heteroskedastic (ARCH) processes are a form of stochastic process that are widely used in finance and economics for modeling conditional heteroskedasticity and volatility clustering.


First proposed by Engle (1982), ARCH processes are univariate conditionally heteroskedastic white noises. 



Computation of VaR

Distribution of portfolio return:

 

 If a portfolio return follows a normal distribution with mean  and standard deviation σ , then the VaR number at the 99 percent confidence level can be obtained as follows:

 

1. Find the α value corresponding to the 99% confidence level in the normal table;

2. Compute the VaR number VaR=- .


--------------------------------------------------------------------------------
VaR Example 

Stock ABC has a market position of $200,000 and an annualized volatility of 30%. Calculate the linear VaR with 99% confidence level for a ten day business period. (trading days: 252 days).


VaR annual:    $200,00030 %            =$60,000


VaR 1 day:    VaR annual252                    = $3779.83




VaR 10 day:                  = $3779.83102.33        =$27,800





--------------------------------------------------------------------------------
Distance to Default

Decrease during recession, increase during prosperity




CA: Current Assets

CL: Current Liabilities

AV: Asset Values



--------------------------------------------------------------------------------
Delta equivalence positions


Daily VaR = [Daily volatility]  [Delta equivalent position] [exchange position]



VaR10days99%Conf. Int.


Daily VaR = 0.7%561.5 = 0.588


VaR(99%,10) =VaR10days99%= 0.5882.3310=4.33


Pooled VaR = VaR12+VaR22+ [2VaR1VaR212]]2



--------------------------------------------------------------------------------
Altman's Z Score


Suppose the financial rations of a potential borrowing firm look at the following values



x1= 0.2    x2=0    x3= -0.2    x4=0.1    x5= 2 



Weights 

Working capital /Total assets 1.2

Retained earnings / Total assets 1.4

EBIT    3.3

Total Liabilities 0.6

Sales     0.999



Z = (1.2)(0.2) + (1.4)(0) + (3.3)(-2) + (0.6)(0.1) + (1.0)(2.0) = 1.64






--------------------------------------------------------------------------------


Loss Databases

 

The data in external databases are edited so that names and other means of identification of the origin are deleted. Moreover, these databases are not designed for mutual spying, but for common progress. Every institution should design and structure these databases at the outset so that they can stand the test of time as well as external databases. A badly structured internal database is likely to be a costly and useless exercise. Internal databases should be, at the outset, regulation-compliant, as regulation is flexible enough to allow tools that are compliant as well as internally useful. For high-impact low-frequency data, internal data are likely to be too succinct. The collated data of several institutions are likely to be a much better guide to the future.


  

 

 

III.B.6.2 Economic Credit Capital Calculation


Capital is designed to absorb unexpected losses up to a certain confidence level, while credit reserves are expected to absorb expected losses.

 

Economic Capital and Credit portfolio model

 

Credit Loss definition

 

Quantile of the Loss distribution

III.B.2.6.2 Expected and unpexpected losses


Expected losses (EL)

 

III.B.2.6.3 Entreprise Credit Capital and Risk Aggregation



--------------------------------------------------------------------------------
Marginal VaR

Marginal VaR is the partial derivative with respect to the component weight, and it measures the change in portfolio VaR resulting from adding additional dollar to a component.

Incremental VaR

Incremental VaR measures the change in the VaR due to a new position on the portfolio.

Component VaR

Component VaR is a partition of the portfolio VaR that indicates the changeof VaR if a given component was deleted.

 


--------------------------------------------------------------------------------
Actuarial Methods


What are the main limitations of standard deviation as an indicator of credit risk?

The distribution of credit risk losses is skewed (asymmetric). Hence the standard deviation combines information related to the right-hand side of the

curve (upside), which is not directly relevant to credit risk assessments. To complement standard deviation, percentiles of loss distribution can be used.


--------------------------------------------------------------------------------
Marginal and Cumulative Default Risk

The default rates on a portfolio have been estimated at 2% for the coming year and 4% for next year. What is the expected payment of 2-year obligations?


The cumulative default rate is (1.00-0.02)(1.00-0.04) = 0.9408.

 

94% of obligations are likely to be paid back.




--------------------------------------------------------------------------------



