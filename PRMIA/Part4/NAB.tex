National Australian Bank FX Options scandal



Foreign currency trader staff fraud

In 2004, NAB discovered that as a result of unauthorised spot trades on its foreign currency options desk, losses totalling A$360 million had been covered up. Investigations by Price Waterhouse Coopers and the Australian Prudential Regulation Authority highlighted a need for cultural change. The losses were a result of a failed speculative position where the traders falsified profits to trigger bonuses over a number of years. 

In order to actually generate the reported profits, the traders speculated on the US dollar, betting that it would rise against the Australian dollar and other currencies. 

In 2006, two former NAB foreign currency options traders were sentenced on charges brought by ASIC and incurred jail terms.
 
Convictions
Ficarra and Bullenwere found guilty in May on a string of charges related to the unauthorised trading on NAB's foreign exchange options desk, which cost the bank $326 million in 2003/04.
 
The presiding judge said the former traders' conduct was dishonest by the standards of ordinary decent people and Bullen and Ficarra knew it would be regarded as such.
 


 
 Ficarra was the most junior of four offenders on the foreign exchange options desk. His role was considered "largely mechanical" , carrying out Duffy's instructions.
