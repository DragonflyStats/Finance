\documentclass[PRMIA4A.tex]{subfiles} 

\begin{document} 
\newpage
\section{ORANGE COUNTY}


This case study focuses on the bankruptcy of Orange County, US in December 1994 after suffering losses of $\$1.6$ billion from a wrong-way bet on interest rates in one of its principal investment pools.

\subsection{Learning Outcomes}
The candidate should be able to:
\begin{itemize}
	\item Describe the Timeline of Events
	\item Describe the lessons learnt
	\item Discuss the events leading up to the losses, the risks
	incurred and the mitigation processes described
\end{itemize}


Robert Citron was the county treasurer for Orange County.


Robert Citron, the treasurer of Orange County who controlled the $\$7.5$ billion pool had riskily invested the pool’s funds in a leveraged portfolio of mainly interest-linked securities. His strategy depended on short-term interest rates remaining relatively low when compared with medium-term interest rates.

But from Febuary 1994, the Federal Reserve Bank began to raise US interest rates causing many securities in Orange County’s investment pool to fall in value.
On December 6, 1994 Orange County declared bankruptcy after suffering losses of arounf $\$1.6$ billion.

Orange County declared Chapter 9 bankruptcy on December 6, 1994. The bankruptcy was brought on by Citron's investment strategies, which seemed to be an effort to earn high incomes for the county to pay for increased demand for county services in a time of strong opposition to raising taxes


As controller of the various Orange County funds, Citron had taken a highly leveraged position using repurchase agreements (repos) and floating rate notes (FRNs).


In February 1994, the Federal Reserve Bank began to raise US interest rates, causing many securities in Orange County's investment pools to fall in value. 

\subsection*{Potential Mitigation}
Beware the unconstrained star performer, even when he or she has a long track record. Where there’s excess reward, there’s risk – though it might take time to surface.

If the organisational structure, planning and risk oversight mechanisms of an institution are fractured, it is easy for powerful individuals to hide risk in the gaps.

Borrowing short and investing long means liquidity risk, as every bank knows.
Risk-averse investors must tie investment objectives to investment actions by means of a strict framework of investment policies, guidelines, risk reporting and independent and expert oversight.

Risk reporting should be complete, and easily comprehensible to independent professionals. Strategies that are not posiible to explain to third parties should not be employed by the risk averse.

In 1994, Orange County announced that its investment pool had lost $\$1.6$ billion. The announcement from the Southern California county seemed as unthinkable as local authorities announcing they had discovered a glacier. Not only was this the largest loss by a local government investment pool, which forced the county to file for bankruptcy, it shattered the pristine image of municipal bonds. The strange story of how the impossible became possible starts and ends with one man - Robert Citron.

As treasurer of Orange County, Robert Citron was considered to be somewhat of an investing whiz. He consistently beat neighboring investment pools by at least 2\% and, as a result, a steady flow of cash came his way. Unfortunately, of the schools, cities and districts that rushed to invest with him, very few looked into how Citron was able to produce such amazing returns.

\subsection*{Cause of Financial Failure}

In the simplest terms, Citron counted on interest rates remaining low. From this view, the difference in yield on a short-term yield and a long-term yield offered an opportunity for arbitrage, so Citron used structured notes to take advantage of this.

Although this increased the risk as well as the potential profit, it was a viable strategy. However, Citron leveraged the entire portfolio to further magnify the gains. And, therein, lied the problem.

Citron did a series of reverse repurchase agreements that allowed him to use his securities as collateral for loans to buy yet more securities. Through this method, he turned the sizable $\$7$ billion portfolio into a $\$20$ billion dollar position. The massive leveraging amplified his gains while interest rates followed his predicted course.

In February 1994, however, the Feds began to raise interest rates and Citron's amplified gains turned into amplified losses. As the rate hikes continued, the losses became too much to control.

\subsection*{Bankruptcy and the Aftermath}

The county was forced into bankruptcy and came up with a recovery plan to float $\$800$ million in bonds. The bankruptcy tarnished the county's image and the municipal bonds were sold at a discount to the treasury. Luckily, the issue proved to be sufficient to protect the investors, schools among them, from insolvency. Citron, however, never served prison time for his actions.

\subsection*{Reverse Repurchase Agreement}

The purchase of securities with the agreement to sell them at a higher price at a specific future date.
For the party selling the security (and agreeing to repurchase it in the future) it is a repo; for the party on the other end of the transaction (buying the security and agreeing to sell in the future) it is a reverse repurchase agreement.


On December6 1994, Orange County declared Bankruptcy after suffering losses of around \$1.6 Billion.

Losses Results ina wrong-way bet on interest rates in one of its main investment pools

largest financial failure ofo US local government

Treasurer : Robert Citron
Controlled 7.5 billion in assets

Riskily Invested in Interest Linked Securities

Strategy dependes on Short term interest rates staying relatively low compared to medium term interest rates

in Feb 1994, the federal reserve began to raise rates, causing manny securities in Orange County's portfolio to fall in value.

Citron ignored the losses at first, but demands for billions of dollars of collateral from Wall Street Counterparties, and 
a threat of a run of deposits from local government investors grew into a liquidity trap from which he ould not escape.

%============================================================================= %
\subsection*{Orange County - Aftermath}


Citron was fined \$100000 and given 1 years house arrest (lenient, but he did not operate for personal gain), on the basis 
of 6 felony counts.

The county issued recovery bonds and with local taxes buoyant, came out of bankruptcy after 18 months.

Moodys awarded the country with a credit rating key for borrowing

Merrill Lynch made and out-of-contract settlement.


\end{document}