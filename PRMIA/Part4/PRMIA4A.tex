\documentclass[]{article}

\usepackage{subfiles}
\usepackage{amsmath}
\usepackage{amssymb}
\usepackage{graphicx}
%============================================ %

\title{PRMIA I Financial Instruments}
\author{Kevin O'Brien}
%\date{} % Activate to display a given date or no date (if empty),
         % otherwise the current date is printed 


\begin{document}



\maketitle
\tableofcontents

\title{PRMIA IV  - Case Studies}
\author{Kevin O'Brien}
%\date{} % Activate to display a given date or no date (if empty),
         % otherwise the current date is printed 



%---------------------------------------------------%
\newpage
\section{NATIONAL AUSTRALIA BANK – FX OPTIONS}
This case study consists of the “Investigation into foreign exchange losses at the National Australia Bank, 12 March 2004.”

The candidate should be able to:
\begin{itemize}
\item Describe the sequence of events and trading activities that led to the losses
\item Describe the analysis of the losses and how they occurred, with an initial
focus on foreign currency option transactions entered into on or after
1 October 2003
\item Describe the key policies, procedures, systems and control failures within
the foreign currency options trading business responsible for the losses.
\item Describe the Impacts on customers and third parties
\item Discuss the events leading up to the losses, the risks incurred and the
mitigation processes described
\end{itemize}
%---------------------------------------------------%
%---------------------------------------------------%
\newpage
\section{TAISEI FIRE AND MARINE INSURANCE CO.}
This case study focuses on the events leading up to the bankruptcy of Taisei Fire and Marine Insurance Co
(“TFMI”) with losses of USD $\$2.5$ billion, and describes how TFMI management were unaware of the risks they had
retained, irrespective of reinsurance policies they had placed with Fortress Re. It also describes the fundamental
characteristics ‘finite’ reinsurance policies.

 
In November 2001, following the September, 11th 2001 (“9/11”) terrorist attack on the World Trade
Center, Taisei Fire and Marine Insurance Co (TFMI) collapsed, due to catastrophic insurance claims of
$\$2.5$ billion. 
 
In December 2002, Sompo Japan acquired Taisei Fire and Marine Insurance.
\subsection{Events}
\begin{description}
\item[1972] – TFMI enters into first management agreement with Fortress Re
\item[2001] – Sept, The terrorist attack on the World Trade Centre
\item[2001] – Nov, TFMI files for protection (rehabilitation) under Japanese law
\item[2002] – TFMI has estimated excess of liabilities over assets of US$\$765$ million
\item[2002] – Jan, The remaining business lines of TFMI are acquired by another Japanese insurance company
\end{description} 

Taisei Fire and Marine Insurance Co has become the first Japanese company to go under directly as a result of the terrorist attacks of September 11, after it filed for insolvency, equivalent to bankruptcy for a regular company, with the Tokyo District Court on Thursday. Company sources disclosed that the midsize casualty-insurance company had a negative net worth of 39.8 billion yen ($\$US324.7$ million) at the end of September due to massive payouts in the wake of the terrorist attacks in the United States.
 
Taisei president Ichiro Ozawa told a news conference that the company, which sold reinsurance on the airplanes used in the terrorist assaults, had total payments of 74.4 billion yen connected to the attacks. "An accident involving an airplane crashing in the middle of New York was not something we had included in our calculations," said Ozawa, who admitted that Taisei had not been aware of the risk because it had not checked the contracts made by its overseas agent. "We left it to the overseas agent to decide," he said.
 
The company is only the second Japanese nonlife insurer to collapse since the end of World War II, following Daiichi Mutual Fire and Marine Insurance Co, which was ordered by the then Financial Supervisory Agency to suspend operations in May 2000. But it is the first casualty insurer to seek court protection and the fourth insurer to collapse in Japan since October. The shock waves will most certainly move through the insurance industry in particular.
 
Taisei's revenues from insurance premiums totaled 88.7 billion yen in the year ended March 2001, with its assets amounting to 411.4 billion yen, the 16th-largest among Japan's 37 nonlife insurers. Its solvency margin stands at 815.2 percent, well above the 200 percent threshold regarded by financial regulators as the level that indicates the basic health of an insurer.
 
The move was first confirmed by Financial Services Minister Hakuo Yanagisawa, who quickly stressed on Thursday that "no other casualty insurers show signs of suffering a similar problem" at present.
 
Taisei has 364.8 billion yen in total debt, the second-largest after Mycal Corp among Japanese companies that have failed this year, according to private research firm Teikoku Databank Ltd.
The insurer had planned to merge operations with Yasuda Fire and Marine Insurance Co and Nissan Fire and Marine Insurance Co to form Sompo Japan Insurance Inc next April. Yasuda and Nissan are now expected to combine their businesses as planned and the merged company will likely acquire Taisei. Sompo Japan is set to become the country's second-largest insurance group after Millea Holdings Inc, also to be established next April, by Tokio Marine and Fire Insurance Co and Nichido Fire and Marine Insurance Co.
 
Yasuda said in a statement on Thursday that Taisei's failure will not affect its plans to create Sompo Japan. "We will start discussions immediately with an eye to protecting Taisei's customers, staff and outlets as much as possible in a joint effort with Nissan Fire and Marine Insurance," the statement said.
 
Yasuda expects to make 2.6 billion yen in insurance payouts related to the terrorist attacks, while Nissan projects payments of 74.4 billion yen.
The Tokyo Stock Exchange said it would delist the company's shares on February 23. The collapse shook the market, and forced a selloff in other insurers. The benchmark Nikkei-225 index was down 1.2 percent at 10,533.83 at the noon break on Thursday, though it recovered to close at 10,696.8.
 
\subsection*{Trade surplus plunges}

Japan's once-massive trade surplus fell by 32 percent last month compared with the previous year, with both exports and imports dropping off, the latest in a long line of indicators that the world's second-largest economy is in the midst of a prolonged recession, with most analysts now forecasting that it will last through at least the end of next year.
 
Still, Japan enjoyed a surplus for the month of $\$3.8$ billion, and its trade balance with the United States was in Japan's favor at $\$5$ billion. Sales of high-tech products have been hit hard because of the sharp slowdown in the United States, the key market for Japanese goods.
The size of the surplus was once a source of bitter controversy with Washington. But given how steadily it has been declining, Washington has expressed concern over the alarming weaknesses in the Japanese economy, and the possible damage it could do to the world's financial system. Japan has the largest national debt in the developed world.
 
The Japanese administration agreed this week upon a second supplementary budget this year in an effort to generate growth, after already implementing one only late last week. The extra spending in effect breaks a campaign promise by Prime Minister Junichiro Koizumi that he would cap government borrowing at 30 trillion yen. He now says his priority remains long-term restructuring, but says he is hoping to buy some time by alleviating the current economic pain.
 

%---------------------------------------------------%
\newpage
\section{WASHINGTON MUTUAL}
This case study focuses on how in September 2008 Washington Mutual - due to a strategy of low lending standards
and bad quality acquisitions - was seized by the US regulators after a history dating back to 1889.

\subsection{Learning Outcomes}

\begin{itemize}
\item Describe how the bank’s acquisition of Long Beach Financial in 1999, and
Providian in 2005 – both forays into sub-prime lending - brought about the
eventual shrinkage of the credit quality of the bank’s loan book
\item Describe the effect and dependency of FHLB funding when only $60\%$ of the
bank’s assets were funded by depositors
\item Characterize the deteriorating effect on earnings that substantially
increased provisions and net charge-offs would have
\item Identify the events of 2007/8 which contributed significantly to the seizure
of the bank by federal authorities
\end{itemize}

\subsection{Long Beach Financial in 1999}


%---------------------------------------------------%
\newpage
\section{FANNIE MAE AND FREDDIE MAC}

This case study provides an insight into the day-to-day workings of two of the largest US financial intermediaries in the asset backed security business, and the events leading up to their bailout by the US Government in 2008.

\subsection{Learning Outcomes}

\begin{itemize}
\item Describe how the intervention of politicians, and the creativity of banks in
selling to sub-prime lenders, overwhelmed the capabilities of both
organizations, and derailed what was essentially a viable and valuable
business model
\item Describe how the computer models were ineffective in stress testing the
multiplicity of variable repayment and interest plans initially sold to
sub-prime lenders
\item Characterize the fundamentals of disintermediation within the asset backed
securities value-chain
\item Show how intricately linked the originate-and-distribute model is to investor
confidence, both locally and globally
\end{itemize}
%---------------------------------------------------%
\newpage
\section{LTCM}
This case study focuses on the collapse of hedge fund, Long-Term Capital Management.

\subsection{Learning Outcomes}
\begin{itemize}
\item Describe the events that led to the collapse of LTCM
\item Describe the lessons learnt
\item Describe how UBS made a loss due to LTCM
\item Discuss the events leading up to the losses, the risks incurred and the mitigation processes described
\end{itemize}
%---------------------------------------------------%


%---------------------------------------------------%
\newpage
\section{METALLGESELLSCHAFT}

This case study focuses on the losses of approximately $\$1.5$ billion made by the "Energy Group" of Metallgesellschaft AG in December, 1993.

\subsection{Learning Outcomes}
The candidate should be able to:
\begin{itemize}
\item Describe the trading strategies employed by the conglomerate
\item Describe how proper supervision could have averted disaster
\item Describe how similar financial crises may be avoided in the future
\item Discuss the events leading up to the losses, the risks incurred and the
mitigation processes described
\end{itemize}
%---------%
\subsection{Backwardisation and Contango}

Backwardation  is a theory developed in respect to the price of a futures contract and the contract's time to expire. Backwardation says that as the contract approaches expiration, the futures contract will trade at a higher price compared to when the contract was further away from expiration. This is said to occur due to the convenience yield being higher than the prevailing risk free rate.
 
Contango describes the situation whereby the futures price is above the expected future spot price. Consequently, the price will decline to the spot price before the delivery date. Contango is the opposite of backwardation.
%---------------------------------------------------%
\newpage
\section{WORLDCOM}

WorldCom achieved its position as a significant player in the telecommunications industry through the successful completion of 65 acquisitions. Between 1991 and 1997, WorldCom spent almost $\$60$
 billion in the acquisition of many of these companies and accumulated $\$41$ billion in debt.
n Mergers and Acquisitions, especially large ones, present significant managerial challenges in at least two areas.

First, management must deal with the challenge of integrating new and old organizations into a single smoothly functioning business. The second challenge is the requirement to account for the financial aspects of the acquisition.

%---------------------------------------------------%
\newpage
\section{NORTHERN ROCK}

This case study focuses on the events leading up to the first run on a UK bank since 1866 – that of Northern Rock in August 2007.

\subsection{Learning Outcomes}
The candidate should be able to:
\begin{itemize}
\item Describe the Timeline of Events
\item Describe how the bank’s business model was a significant contributory
factor in the crisis
\item Understand how liquidity management is so vitally important when
managing a mismatched funding book
\item Describe the multi-dimensional (co-variance) risk problems of Northern
Rock
\item Understand Reputational Risk, and the difference between Solvency and
Liquidity,
\item Discuss the role of the regulator, and Bank of England
\item Discuss the events leading up to the losses, the risks incurred, the eventual
UK government rescue operation, and the mitigation process described
\end{itemize}
%--------%
\subsection{Summary}
\begin{itemize}
\item In 1997, Northern Rock converted to bank status, thereby allowing it to conduct a full range of banking busi
ness, with a central securitisation and funding strategy focused predominantly on secured wholesale and other
capital market funding
\item During the summer of 2007, it was a victim of the general market turbulence and lack of confidence triggered byworsening
developments in the US sub-prime mortgage market
\item Also in the summer of 2007, notwithstanding unprecedented UK government support, depositors withdrew funds
on a large scale
\item In September 2007, Northern Rock was forced to seek massive financial assistance from the Bank of England,
and eventually reverted to UK government ownership
\end{itemize}
%Northern Rock
 
UK govt. nationalised Northern Rock.
Northern Rock plc was best known for becoming the first bank in 150 years to suffer a bank run after having had to approach the Bank of England for a loan facility, to replace money market funding, during the credit crisis in 2007.



\subsection*{Granite}
Granite is a securitisation vehicle created by the British bank Northern Rock, based in Guernsey.
The purpose of Granite is to parcel up the mortgages provided by the bank and sell the value to investors. Granite has a value of around £45 billion.
Northern Rock, advised by Credit Suisse, have decided to let Granite go into run-off, meaning that Northern Rock the bank will no longer supply it with fresh mortgages and bondholders will be repaid as old mortgages expire.
In plans made public on 8 December 2009 certain wholesale deposits are to be held by the renamed assets company, Northern Rock (Asset Management) plc, on behalf of Granite.

\subsection*{Nationalisation}
In 2008 the Northern Rock bank was nationalised by the British Government, due to financial problems caused by the subprime mortgage crisis. In 2010 the bank was split into two parts (assets and banking) to aid the eventual sale of the bank back to the private sector.

%---------------------------------------------------%
\newpage
\section{CHINA AVIATION OIL (SINGAPORE)}

This case study provides an insight of the losses (US$\$550$ million) incurred by China Aviation Oil (Singapore)
(“CAO”) in 2004 due to a combination of trading losses in oil futures being misreported in a period of transition
between accounting regulations, and without the appropriate and necessary risk management policies, procedures
and systems being in place.

\subsection{Learning Outcomes}
\begin{itemize}
\item Describe how the absence of corporate risk management objectives, policies
and reporting, lead to the losses being incurred and hidden
\item Describe how proper supervision could have averted the losses
\item  Discuss the events leading up to the losses, the risks incurred and the
mitigation processes described
\item  Describe how the difference in the accounting requirements of IFRS and IAS
39 lead to the misreporting of the losses
\end{itemize}

\begin{itemize}
\item China Aviation Oil (Singapore) Corporation Ltd ("CAO") is the largest purchaser of jet fuel in the Asia Pacific region and the key supplier of imported jet fuel to the civil aviation industry of the People's Republic of China ("PRC").
\item China's jet fuel import monopoly, Singapore-listed China Aviation Oil (CAO), reported in 2004 whopping losses to the tune of US $\$550$ million, due largely to speculation and unwise derivative trading.

\item China Aviation Oil said in November 2004 it was forced to close speculative trades when it could not meet funding requirements needed to back the contracts after oil prices surged to a record the previous month.

\item Chen Jiulin (44) the former head of China Aviation Oil (Singapore) Corp has been sentenced to a prison term of more than four years for his role in a scandal that drove China's biggest jet-fuel trader to the brink of bankruptcy.  Chen Jiulin pleaded guilty to six charges, including failing to disclose a US $\$550$ million trading loss and deceiving adviser Deutsche Bank AG.

\item CAO made bets on the direction of oil prices and ran up losses as New York oil futures surged to a record US $\$55.67$ a barrel on October 25 2004.

\item Instead of leaving the market and accepting losses of several million dollars, the company raised its bets until it was faced with losses that it could not meet.

\item The trading loss was close to the company's market value of US $\$570$ million when the shares were suspended on November 29. The stock had fallen 49 per cent since touching a record high on March 23.

\item "Due to the company's inability to pay, the company's creditor-banks commended the forced closing of a number of derivative contracts," Chen said in the statement posted on the company's official website. "As a result, the potential losses that the company had been facing were transformed into actual realized losses."

\end{itemize}





\subsection*{Timeline}
Q3 2003 CAO began making speculative option trades to profit from favourable market movements in oil-related commodities.
 
October 2004 The situation came to a head when international oil prices significantly exceeded USD 38.00 price, forcing CAO to fund significant margin calls on its open (short) derivative positions. 
 


China's jet fuel import monopoly, Singapore-listed China Aviation Oil (CAO), reported in 2004 whopping losses to the tune of US$\$550$ million, due largely to speculation and unwise derivative trading.
 
China Aviation Oil said in November 2004 it was forced to close speculative trades when it could not meet funding requirements needed to back the contracts after oil prices surged to a record the previous month.

Chen Jiulin (44) the former head of China Aviation Oil (Singapore) Corp has been sentenced to a prison term of more than four years for his role in a scandal that drove China's biggest jet-fuel trader to the brink of bankruptcy.  Chen Jiulin pleaded guilty to six charges, including failing to disclose a US$\$550$ million trading loss and deceiving adviser Deutsche Bank AG.
 
 
CAO made bets on the direction of oil prices and ran up losses as New York oil futures surged to a record US $\$55.67$ a barrel on October 25 2004.
Instead of leaving the market and accepting losses of several million dollars, the company raised its bets until it was faced with losses that it could not meet.

The trading loss was close to the company's market value of US$\$570$ million when the shares were suspended on November 29. The stock had fallen 49 per cent since touching a record high on March 23.

"Due to the company's inability to pay, the company's creditor-banks commended the forced closing of a number of derivative contracts," Chen said in the statement posted on the company's official website. "As a result, the potential losses that the company had been facing were transformed into actual realized losses."

%---------------------------------------------------%
\end{document}
