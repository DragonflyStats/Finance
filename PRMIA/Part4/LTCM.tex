
\section{Long Term Capital Management}
A large hedge fund led by Nobel Prize-winning economists and renowned Wall Street traders that nearly collapsed the global financial system in 1998 as a result of high-risk arbitrage trading strategies. 

The fund formed in 1993 and was founded by renowned Salomon Brothers bond trader John Meriwether. 

Long-Term Capital Management L.P. (LTCM) was a speculative hedge fund based in Greenwich, Connecticut that utilized absolute-return trading strategies (such as fixed-income arbitrage, statistical arbitrage, and pairs trading) combined with high leverage.

Interest Rate and Equity Derivatives was the source of the losses.

Board of directors members included Myron Scholes and Robert C. Merton, who shared the 1997 Nobel Memorial Prize in Economic Sciences.

Initially successful with annualized returns of over 40\% (after fees) in its first years, in 1998 it lost $4.6 billion in less than four months following the Russian financial crisis and the fund closed in early 2000.

%--------------------------------------------------------------------------------%
Learning Objectives

Describe the events that led to the collapse of LTCM

Describe the lessons learnt

Describe how UBS made a loss due to LTCM

Discuss the events leading up to the losses, the risks incurred and the mitigation processes described


%--------------------------------------------------------------------------------%
\subsection*{Events}

LTCM had the ability to put on interest rate swaps at the market rate for no initial margin. It meant being able to borrow 100\% of the value of any top-grade collateral, and with that cash to buy more securities and post them as collateral for further borrowing: in theory it could leverage itself to infinity.

Most of LTCM’s bets had been variations on the same theme, convergence between liquid treasuries and more complex instruments that commanded a credit or liquidity premium. Unfortunately convergence turned into dramatic divergence. LTCM’s counterparties, marking their LTCM exposure to market at least once a day, began to call for more collateral to cover the divergence.

%-------------------------------------------------------------------------------%




LTCM started with just over $1 billion in initial assets and focused on bond trading. The trading strategy of the fund was to make convergence trades, which involve taking advantage of arbitrage between securities that are incorrectly priced relative to each other. Due to the small spread in arbitrage opportunities, the fund had to leverage itself highly to make money. At its height in 1998, the fund had $5 billion in assets, controlled over $100 billion and had positions whose total worth was over a $1 trillion. 


Due to its highly leveraged nature and a financial crisis in Russia (i.e. the default of government bonds) which led to a flight to quality, the fund sustained massive losses and was in danger of defaulting on its loans. 

This made it difficult for the fund to cut its losses in its positions. The fund held huge positions in the market, totaling roughly 5\% of the total global fixed-income market. LTCM had borrowed massive amounts of money to finance its leveraged trades. Had LTCM gone into default, it would have triggered a global financial crisis, caused by the massive write-offs its creditors would have had to make. 
In September 1998, the fund, which continued to sustain losses, was bailed out with the help of the Federal Reserve and its creditors and taken over. A systematic meltdown of the market was thus prevented.

%--------------------------------------------------------------------------------%


