%------------------------------------------------%
%------------------------------------------------%
Continentall Illinois

On May 9th 1984 Continental Illinois began a frantic battle to counter reports that 
it was on the brink of insolvency for a combination of bad loas and funding liquidity risk.

At the root of the crisis la a massive portfolio of energy sector loans that had begun to turn sour on Continental
when US oil and gas sectors lurched into recession in 1981.

The \$33 billion asset bnk compounded its mistake by lending large amounts to LDCs prior to August
1982 start of the makor LDC financial crises of the 1980s.

In July 1982 Penn Square bank fails, causing concern about CI, a major participant in Penn Square's risky lending
programme.

In August 1982 Mexico defaults on debt, sparking LDC crisis.

Due diligence was not properly conducted by John Lytte, the executive in charge of Oil Lending.
He later plead guilty to fraud and receiving kick-backs for approving risky loans.
John Lytte was sentenced to 3.5 years in a federal prison.

The Penn Square fiasco caused a major run on CI's deposits once it became clear that CI was headed for failure.

Continental Illinois

On May 9, 1984, Continental Illinois,1 Chicagos largest bank and one of the top ten banks in the US, began a frantic battle to counter reports that it was on the brink of insolvency from a combination of bad loans and funding liquidity risk.
Energy portfolio and LDC Crisis
At the root of the crisis lay a massive portfolio of energy sector loans that had begun to turn sour on Continental when the US oil and gas sectors lurched into recession in 1981. The $33-billion asset bank had compounded its mistakes by lending large amounts to lesser-developed countries prior to the August 1982 start of the major LDC crisis of the 1980s. 
Rumours / Loss of Public Confidence
With investors and creditors spooked by rumours that the bank might fail or be taken over, Continental was quickly shut out of its usual domestic and international wholesale funding markets. The sharp drop in confidence can lead counterparties in the wholesale markets to suddenly withdraw funding from a damaged bank, spinning the institution into a funding liquidity crisis.
Emergency Funding and Guarantee – “Too Big To Fail”
By May 17, regulatory agencies and the banking industry had arranged billions of dollars in emergency funding for the stricken giant. The Federal Deposit Insurance Corporation tried to stem the bleeding away of the banks funds by extending a guarantee to uninsured depositors and creditors at the bank giving credibility to the notion that some banks should be considered too big to fail. 
The emergency help was followed by a package of permanent measures, making Continental the largest bank in the history of US banking ever to be rescued by government agencies. 
The FDICs share of the bill was later calculated to be $1.1 billion.		 

%---------------------------------------------------------------------------------------------------%
 Lessons learned
•	Its tempting for management to grow an institution by offering loans to the same credit-hungry industry sectors. That incentive should be balanced by ensuring that portfolio concentration risks are correctly measured and factored into lending decisions. 
•	Decentralized lending decisions require controls and risk monitoring at the centre of the bank. 
•	Deteriorating credit portfolios, leading to market rumour and the withdrawal of competitively priced funding, can send even large institutions careening down the liquidity death spiral. 
•	Some banks might be too big to fail from depositors and regulators point of view. But shareholders will still lose their shirts.	
 Growth
In the mid 1970s, Continental Illinois embarked on a strategy of growth that over the next half-decade would make the bank the largest commercial and industrial lender in the US. The banks lending to commercial and industrial clients grew from $5 billion to $14 billion between 1976 and 1981. 
Throughout this extended honeymoon period, the growth strategy seemed to give good results. In the late 1970s, Continental achieved an above-average return on equity putting it at the top of the commercial banking league and was showered with praise by equity analysts. In turn, its share price moved upwards, doubling in the five years to 1979 and continuing to rise to a peak in June 1981 (an exceptional performance). 
Commercial banking is notoriously dependent on national and regional economic cycles. A commercial banks interest rate margin the positive gap between the rate a bank pays depositors for funds and the interest rate it is able to charge borrowers can be quickly eaten up by any jump in customer default rates when recession bites. 
One result of this is that fast-growing commercial banks are often in danger of buying market share by lending at rates that will not offset the default risk of their loans over the entire economic cycle. (Early-cycle default rates are reassuringly  but deceptively low; they can change significantly as macro-economic conditions change.)
Lack of Diversification / Concentration on Energy Sector
This is a particular danger when banks concentrate lending in a particular industry sector. Through the late 1970s, Continental worked hard to build up its energy industry business, a sector in which it felt it had special expertise. 
Relationship with Penn Square
One of the tools that Continental employed to speed up this expansion in energy assets was its relationship with Oklahoma-based Penn Square Bank, a $436-million asset bank that specialised in oil and gas sector loans. While Penn Square originated large volumes of loans to the historically risky exploration sector of the US energy industry, Continental participated bought a share in the smaller banks lending book, ultimately to the tune of hundreds of millions of dollars.
Mexico and the LDCs
Continental also expanded its assets in another credit-hungry sector by lending to lesser-developed countries in Latin America such as Mexico. Many large commercial US banks followed the same strategy in the late 1970s. But Continentals credit exposures were compounded by its unusual funding strategy. 
Funding from the Money Markets
Traditionally, banks fund growth in their lending activities by attracting larger volumes of savings from retail depositors. Continental, however, had only a limited retail presence, due in part to federal and local banking regulations that restricted its ability to build a branch network.
This meant that the bank came to depend increasingly on funding from the wholesale money markets. Indeed, by 1981, Continental gained most of its funding through federal funds and by selling short-term certificates of deposit on the wholesale money markets, sourcing only 20% of its funding from traditional retail deposits.
The banks funding strategy meant that Continental was vulnerable to the wholesale funding markets interpretation of its concentrated exposure to the energy and LDC credit sectors. That worked well during the 1970s, but both sectors faced extraordinary challenges as the 1980s dawned. 
Oil Prices Drop (1981) – Continental Suffers
After years of oil price rises and industry expansion, oil prices began to drop in April 1981. They continued to decline for a number of years, along with the fortunes of energy-sector companies. Exploration and drilling companies were inevitably among the first to bear the full brunt of the downturn. Continentals share price began a steep decline, losing over half of its value from mid 1981 to mid 1982.
Through the first half of 1982, most analysts continued to think that any deterioration in Continentals credit portfolio would threaten the banks profitability, not its solvency. After all, Continental was a long-established and substantial bank, with a good earnings record and a triple-A agency rating for its debt. 
Collapse of Penn Square (1982)
Penn Square Bank Collapsed in July 1982 under the weight of poorly underwritten loans to the ailing energy sector. The collapse revealed the extent of Continentals participation in Penn Squares lending and underlined the damage that was being wrought on banks that had specialised in energy-linked lending. Continental was forced to report $1.3 billion in non-performing assets in the second quarter of 1982. 
Mexico Default (August 1982)
In August 1982  Mexico defaulted on its debt, triggering the 1980s LDC crisis. 
Difficulty in raising funds
With market participants becoming worried about Continentals long-term solvency, the bank began to feel the liquidity squeeze inherent in its funding strategy. It found itself paying higher rates on its certificates of deposit as it fell out of favour with savvy domestic counterparties in the wholesale money markets. Increasingly, the bank turned to the foreign Eurodollar money markets for funding, which it could secure by paying relatively high rates. This, however, depressed the banks profit margin and left it vulnerable to any further panic about its creditworthiness in the international markets. 
And while the bank agreed a plan with the OCC to improve its asset/liability management, loan administration and funding,  little could be done to prevent chronic deterioration in the health of the loan portfolio it had built up in the period before the Penn Square collapse. Wrong-footed by rising interest rates and increasing LDC debt problems, Continental announced another increase in non-performing loans, to $2.3 billion, at the end of the first quarter of 1984. 
Rumours circulated widely about the banks financial condition during the spring of that year. International banks began to charge Continental higher rates for funds, a widening of the banks credit spread that may itself have triggered further rumours when it became the subject of a news wire story on May 8. 7  On May 9, rumours of Continentals credit problems, and of a possible take-over, reached a crescendo as a run on the banks funding instruments developed in Tokyo and on other international money markets. 
The flight of billions of dollars of short-term funding from Continental was conducted electronically by institutional investors in the Eurodollar money markets. Most of these investors did not have to withdraw money; they simply refused to roll over the money they had on deposit with the bank at short maturities.
OCC intervenes
With financial journalists questioning Continentals solvency, the Office of the Comptroller of the Currency (OCC), its lead regulator, felt obliged to state publicly that it had no reason to question the banks soundness. 
The regulators reassurances were to little avail. On May 11 the bank was obliged to borrow $3.6 billion from the Federal Reserves discount window to make good the outflow of funds. Within days 16 banks had put together an additional $4.5 billion emergency package of loans for the stricken bank but the run on the bank continued.
Meanwhile, the reliance of other US banks on Continental led to fears that a failure of the bank would trigger a loss of confidence in US banks by foreign investors and lead to a systemic crisis. In particular, some 2300 US banks held deposits in correspondence accounts at Continental, leading to fears that an unknown number perhaps a hundred or so might be catapulted into insolvency in the event of a crisis. (Studies after the event have tended to downplay the likely number of fatalities, suggesting that only a handful of small banks were at risk of failure.8 )
The Comptroller of the Currency at the time, Todd Conover, later summed up the regulators dilemma at the moment of crisis:
We debated at some length how to handle the situation. In our collective judgment, had Continental failed and been treated in a way in which depositors and creditors were not made whole, we could very well have seen a national, if not international, financial crisis the dimensions of which were difficult to imagine. None of us wanted to find out.9 
On May 17, the FDIC, Federal Reserve and OCC announced how much they were willing to pay to remain ignorant of whether Continental really was too big to fail. The package they announced in a joint press statement included: 
- An additional $2 billion in funding for the bank arranged by the FDIC and the banking industry;
- Federal Reserve assurances about providing liquidity; 
- Twenty-four major banks would provide $5.3 billion in unsecured funds until a more permanent solution could be found; and, controversially, 
- The FDIC said it would make good all of the banks depositors and creditors, setting aside the $100,000 limit on deposit insurance that should have governed its payouts.
The FDIC extension of its guarantee to all uninsured depositors and creditors was highly unusual. It was made because most of Continentals funding was in the form of uninsured deposits and regulators believed that this would continue to drive the bank run unless such an assurance was given.
The emergency measures provided a temporary breathing space in which regulators looked for another institution that would agree to take over Continental. The worries about Continentals impaired portfolio and legal entanglements were such that the search proved fruitless.
In July, regulators hammered out a controversial approach to permanently rescuing the institution under which the FDIC would purchase some $4.5 billion book value of impaired loans with an adjusted book value of $3.5 billion in return for assuming $3.5 billion of the banks Federal Reserve debt. 
The so-called permanent assistance program was put into place on 26 September. But the impaired loan portfolio, composed largely of energy and international shipping loans, was so large and complex that the FDIC had to ask Continental itself to manage the loans through a special unit (under strict FDIC oversight).
Under the permanent assistance program, the FDIC also acquired preferred stock in Continentals holding company, in return for infusing $1 billion in equity capital into the bank. This transaction effectively left the government agency as the 80% owner of the bank a nationalisation in everything but name.10  

%---------------------------------------------------------------------------------------------------% 
The aftermath
The rescue package meant that, controversially, creditors who held the debt of the bank holding company did not lose their money. It also gave fresh life to the widely held belief that some banks were simply too big to fail that their demise would have such disastrous systemic implications that regulators would never permit them to collapse.
But the label too big too fail is potentially misleading. Shareholders in the bank lost most of the value of their investment immediately after the FDICs stock purchase. They lost the rest of the value of their investment later on, when the FDIC exercised an option to purchase the remaining shares. (The option became exercisable because the losses from the FDIC-assumed impaired loan portfolio reached a pre-agreed level.) 
Even so, the Continental rescue clearly suggested that uninsured depositors and creditors would be bailed out using taxpayers money if regulators became scared enough about the systemic effects of letting a large bank go to the wall. Shortly after the Continental episode, regulators testifying to Congress admitted that the eleven largest banks in the US were indeed regarded as too big to fail.
This implied government guarantee upset smaller banks in the US, who argued that the too big to fail concept was, in effect, a government subsidy for larger banks. Following the savings and loan sector debacle later in the 1980s, a series of reforms of US banking regulation addressed some of the issues of the too big to fail debate, particularly in terms of making sure that uninsured depositors bore a greater proportion of the cost of bank failure.11   
But the question of how regulators will react when they are again faced with potential systemic repercussions from the meltdown of a very large bank remains open. The answer is more important than ever: since 1984 major US banks have vastly increased in size through merger and acquisition.12  
The rescue of Continental Illinois had many lessons for regulators, but these should not be allowed to outshine the lessons the story contains for bank risk managers. Continentals troubles remain an object lesson in how a disaster can begin with underwriting decisions, snowball through the poor management of portfolio concentration risk, and culminate in a violent explosion of market rumour and funding liquidity risk.  
	
%---------------------------------------------------------------------------------------------------% 
Timeline
1976 - 1981: Continental grows fast, lending to the energy sector and lesser-developed countries

1981: Continental hits a high point in terms of asset size and by now employs some 12,000 people

1981: The US energy sector, hit by oil price falls, moves towards recession

July 1982: Energy-sector specialist Penn Square Bank fails, causing concern about Continental, a major participant in Penn Squares risky lending programme

August 1982: Mexico defaults on debt, sparking LDC debt crisis

Spring 1984: With non-performing loans rising sharply, Continentals critics are now wondering about its solvency

9 May: Final liquidity crisis begins in Eurodollar markets as rumours swirl about Continentals creditworthiness

11 May: Continental forced to borrow $3.6 billion through Fed discount window

14 May: 16 banks put together rescue package of funding, but the run on Continental continues

17 May: Unprecedented interim assistance package announced by panicked regulators. Uninsured depositors and creditors given FDIC assurances.

July: With it proving difficult to find a purchaser for the bank, regulators work with Continental to devise a permanent solution

26 September: Permanent solution put in place, effectively nationalising Continental using FDIC funds

1991: FDIC sells off its last equity stake in Continental, bringing the rescue programme to a close some seven years after the banks near collapse	
%---------------------------------------------------------------------------------------------------%
	













Continental Illinois

 

On May 9, 1984, Continental Illinois,1 Chicagos largest bank and one of the top ten banks in the US, began a frantic battle to counter reports that it was on the brink of insolvency from a combination of bad loans and funding liquidity risk.

Energy portfolio and LDC Crisis

At the root of the crisis lay a massive portfolio of energy sector loans that had begun to turn sour on Continental when the US oil and gas sectors lurched into recession in 1981. The $33-billion asset bank had compounded its mistakes by lending large amounts to lesser-developed countries prior to the August 1982 start of the major LDC crisis of the 1980s.

Rumours / Loss of Public Confidence

With investors and creditors spooked by rumours that the bank might fail or be taken over, Continental was quickly shut out of its usual domestic and international wholesale funding markets. The sharp drop in confidence can lead counterparties in the wholesale markets to suddenly withdraw funding from a damaged bank, spinning the institution into a funding liquidity crisis.

Emergency Funding and Guarantee – “Too Big To Fail”

By May 17, regulatory agencies and the banking industry had arranged billions of dollars in emergency funding for the stricken giant. The Federal Deposit Insurance Corporation tried to stem the bleeding away of the banks funds by extending a guarantee to uninsured depositors and creditors at the bank giving credibility to the notion that some banks should be considered too big to fail.

The emergency help was followed by a package of permanent measures, making Continental the largest bank in the history of US banking ever to be rescued by government agencies.

The FDICs share of the bill was later calculated to be $1.1 billion.
 

 
 






Lessons learned

∙        Its tempting for management to grow an institution by offering loans to the same credit-hungry industry sectors. That incentive should be balanced by ensuring that portfolio concentration risks are correctly measured and factored into lending decisions.

∙        Decentralized lending decisions require controls and risk monitoring at the centre of the bank.

∙        Deteriorating credit portfolios, leading to market rumour and the withdrawal of competitively priced funding, can send even large institutions careening down the liquidity death spiral.

∙        Some banks might be too big to fail from depositors and regulators point of view. But shareholders will still lose their shirts.
 

 
 






Growth

In the mid 1970s, Continental Illinois embarked on a strategy of growth that over the next half-decade would make the bank the largest commercial and industrial lender in the US. The banks lending to commercial and industrial clients grew from $5 billion to $14 billion between 1976 and 1981.

Throughout this extended honeymoon period, the growth strategy seemed to give good results. In the late 1970s, Continental achieved an above-average return on equity putting it at the top of the commercial banking league and was showered with praise by equity analysts. In turn, its share price moved upwards, doubling in the five years to 1979 and continuing to rise to a peak in June 1981 (an exceptional performance).

Commercial banking is notoriously dependent on national and regional economic cycles. A commercial banks interest rate margin the positive gap between the rate a bank pays depositors for funds and the interest rate it is able to charge borrowers can be quickly eaten up by any jump in customer default rates when recession bites.

One result of this is that fast-growing commercial banks are often in danger of buying market share by lending at rates that will not offset the default risk of their loans over the entire economic cycle. (Early-cycle default rates are reassuringly  but deceptively low; they can change significantly as macro-economic conditions change.)

Lack of Diversification / Concentration on Energy Sector

This is a particular danger when banks concentrate lending in a particular industry sector. Through the late 1970s, Continental worked hard to build up its energy industry business, a sector in which it felt it had special expertise.

Relationship with Penn Square

One of the tools that Continental employed to speed up this expansion in energy assets was its relationship with Oklahoma-based Penn Square Bank, a $436-million asset bank that specialised in oil and gas sector loans. While Penn Square originated large volumes of loans to the historically risky exploration sector of the US energy industry, Continental participated bought a share in the smaller banks lending book, ultimately to the tune of hundreds of millions of dollars.

Mexico and the LDCs

Continental also expanded its assets in another credit-hungry sector by lending to lesser-developed countries in Latin America such as Mexico. Many large commercial US banks followed the same strategy in the late 1970s. But Continentals credit exposures were compounded by its unusual funding strategy.

Funding from the Money Markets

Traditionally, banks fund growth in their lending activities by attracting larger volumes of savings from retail depositors. Continental, however, had only a limited retail presence, due in part to federal and local banking regulations that restricted its ability to build a branch network.

This meant that the bank came to depend increasingly on funding from the wholesale money markets. Indeed, by 1981, Continental gained most of its funding through federal funds and by selling short-term certificates of deposit on the wholesale money markets, sourcing only 20% of its funding from traditional retail deposits.

The banks funding strategy meant that Continental was vulnerable to the wholesale funding markets interpretation of its concentrated exposure to the energy and LDC credit sectors. That worked well during the 1970s, but both sectors faced extraordinary challenges as the 1980s dawned.

Oil Prices Drop (1981) – Continental Suffers

After years of oil price rises and industry expansion, oil prices began to drop in April 1981. They continued to decline for a number of years, along with the fortunes of energy-sector companies. Exploration and drilling companies were inevitably among the first to bear the full brunt of the downturn. Continentals share price began a steep decline, losing over half of its value from mid 1981 to mid 1982.

Through the first half of 1982, most analysts continued to think that any deterioration in Continentals credit portfolio would threaten the banks profitability, not its solvency. After all, Continental was a long-established and substantial bank, with a good earnings record and a triple-A agency rating for its debt.

Collapse of Penn Square (1982)

Penn Square Bank Collapsed in July 1982 under the weight of poorly underwritten loans to the ailing energy sector. The collapse revealed the extent of Continentals participation in Penn Squares lending and underlined the damage that was being wrought on banks that had specialised in energy-linked lending. Continental was forced to report $1.3 billion in non-performing assets in the second quarter of 1982.

Mexico Default (August 1982)

In August 1982  Mexico defaulted on its debt, triggering the 1980s LDC crisis.

Difficulty in raising funds

With market participants becoming worried about Continentals long-term solvency, the bank began to feel the liquidity squeeze inherent in its funding strategy. It found itself paying higher rates on its certificates of deposit as it fell out of favour with savvy domestic counterparties in the wholesale money markets. Increasingly, the bank turned to the foreign Eurodollar money markets for funding, which it could secure by paying relatively high rates. This, however, depressed the banks profit margin and left it vulnerable to any further panic about its creditworthiness in the international markets.

And while the bank agreed a plan with the OCC to improve its asset/liability management, loan administration and funding,  little could be done to prevent chronic deterioration in the health of the loan portfolio it had built up in the period before the Penn Square collapse. Wrong-footed by rising interest rates and increasing LDC debt problems, Continental announced another increase in non-performing loans, to $2.3 billion, at the end of the first quarter of 1984.

Rumours circulated widely about the banks financial condition during the spring of that year. International banks began to charge Continental higher rates for funds, a widening of the banks credit spread that may itself have triggered further rumours when it became the subject of a news wire story on May 8. 7  On May 9, rumours of Continentals credit problems, and of a possible take-over, reached a crescendo as a run on the banks funding instruments developed in Tokyo and on other international money markets.

The flight of billions of dollars of short-term funding from Continental was conducted electronically by institutional investors in the Eurodollar money markets. Most of these investors did not have to withdraw money; they simply refused to roll over the money they had on deposit with the bank at short maturities.

OCC intervenes

With financial journalists questioning Continentals solvency, the Office of the Comptroller of the Currency (OCC), its lead regulator, felt obliged to state publicly that it had no reason to question the banks soundness.

The regulators reassurances were to little avail. On May 11 the bank was obliged to borrow $3.6 billion from the Federal Reserves discount window to make good the outflow of funds. Within days 16 banks had put together an additional $4.5 billion emergency package of loans for the stricken bank but the run on the bank continued.

Meanwhile, the reliance of other US banks on Continental led to fears that a failure of the bank would trigger a loss of confidence in US banks by foreign investors and lead to a systemic crisis. In particular, some 2300 US banks held deposits in correspondence accounts at Continental, leading to fears that an unknown number perhaps a hundred or so might be catapulted into insolvency in the event of a crisis. (Studies after the event have tended to downplay the likely number of fatalities, suggesting that only a handful of small banks were at risk of failure.8 )

The Comptroller of the Currency at the time, Todd Conover, later summed up the regulators dilemma at the moment of crisis:

We debated at some length how to handle the situation. In our collective judgment, had Continental failed and been treated in a way in which depositors and creditors were not made whole, we could very well have seen a national, if not international, financial crisis the dimensions of which were difficult to imagine. None of us wanted to find out.9

On May 17, the FDIC, Federal Reserve and OCC announced how much they were willing to pay to remain ignorant of whether Continental really was too big to fail. The package they announced in a joint press statement included:

- An additional $2 billion in funding for the bank arranged by the FDIC and the banking industry;


- Federal Reserve assurances about providing liquidity;


- Twenty-four major banks would provide $5.3 billion in unsecured funds until a more permanent solution could be found; and, controversially,


- The FDIC said it would make good all of the banks depositors and creditors, setting aside the $100,000 limit on deposit insurance that should have governed its payouts.

The FDIC extension of its guarantee to all uninsured depositors and creditors was highly unusual. It was made because most of Continentals funding was in the form of uninsured deposits and regulators believed that this would continue to drive the bank run unless such an assurance was given.

The emergency measures provided a temporary breathing space in which regulators looked for another institution that would agree to take over Continental. The worries about Continentals impaired portfolio and legal entanglements were such that the search proved fruitless.

In July, regulators hammered out a controversial approach to permanently rescuing the institution under which the FDIC would purchase some $4.5 billion book value of impaired loans with an adjusted book value of $3.5 billion in return for assuming $3.5 billion of the banks Federal Reserve debt.

The so-called permanent assistance program was put into place on 26 September. But the impaired loan portfolio, composed largely of energy and international shipping loans, was so large and complex that the FDIC had to ask Continental itself to manage the loans through a special unit (under strict FDIC oversight).

Under the permanent assistance program, the FDIC also acquired preferred stock in Continentals holding company, in return for infusing $1 billion in equity capital into the bank. This transaction effectively left the government agency as the 80% owner of the bank a nationalisation in everything but name.10






 












The aftermath

The rescue package meant that, controversially, creditors who held the debt of the bank holding company did not lose their money. It also gave fresh life to the widely held belief that some banks were simply too big to fail that their demise would have such disastrous systemic implications that regulators would never permit them to collapse.

But the label too big too fail is potentially misleading. Shareholders in the bank lost most of the value of their investment immediately after the FDICs stock purchase. They lost the rest of the value of their investment later on, when the FDIC exercised an option to purchase the remaining shares. (The option became exercisable because the losses from the FDIC-assumed impaired loan portfolio reached a pre-agreed level.)

Even so, the Continental rescue clearly suggested that uninsured depositors and creditors would be bailed out using taxpayers money if regulators became scared enough about the systemic effects of letting a large bank go to the wall. Shortly after the Continental episode, regulators testifying to Congress admitted that the eleven largest banks in the US were indeed regarded as too big to fail.

This implied government guarantee upset smaller banks in the US, who argued that the too big to fail concept was, in effect, a government subsidy for larger banks. Following the savings and loan sector debacle later in the 1980s, a series of reforms of US banking regulation addressed some of the issues of the too big to fail debate, particularly in terms of making sure that uninsured depositors bore a greater proportion of the cost of bank failure.11  

But the question of how regulators will react when they are again faced with potential systemic repercussions from the meltdown of a very large bank remains open. The answer is more important than ever: since 1984 major US banks have vastly increased in size through merger and acquisition.12 

The rescue of Continental Illinois had many lessons for regulators, but these should not be allowed to outshine the lessons the story contains for bank risk managers. Continentals troubles remain an object lesson in how a disaster can begin with underwriting decisions, snowball through the poor management of portfolio concentration risk, and culminate in a violent explosion of market rumour and funding liquidity risk. 



 

 
 

 
 






Timeline

1976 - 1981: Continental grows fast, lending to the energy sector and lesser-developed countries




1981: Continental hits a high point in terms of asset size and by now employs some 12,000 people




1981: The US energy sector, hit by oil price falls, moves towards recession




July 1982: Energy-sector specialist Penn Square Bank fails, causing concern about Continental, a major participant in Penn Squares risky lending programme




August 1982: Mexico defaults on debt, sparking LDC debt crisis




Spring 1984: With non-performing loans rising sharply, Continentals critics are now wondering about its solvency




9 May: Final liquidity crisis begins in Eurodollar markets as rumours swirl about Continentals creditworthiness




11 May: Continental forced to borrow $3.6 billion through Fed discount window




14 May: 16 banks put together rescue package of funding, but the run on Continental continues




17 May: Unprecedented interim assistance package announced by panicked regulators. Uninsured depositors and creditors given FDIC assurances.




July: With it proving difficult to find a purchaser for the bank, regulators work with Continental to devise a permanent solution




26 September: Permanent solution put in place, effectively nationalising Continental using FDIC funds




1991: FDIC sells off its last equity stake in Continental, bringing the rescue programme to a close some seven years after the banks near collapse
 

 
 






∙         










 
 









 
 









 
 







 

