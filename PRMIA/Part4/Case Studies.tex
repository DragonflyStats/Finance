\documentclass[12pt, a4paper]{report}
\usepackage{natbib}
\usepackage{vmargin}
\usepackage{epsfig}
\usepackage{subfigure}
%\usepackage{amscd}
\usepackage{amssymb}
\usepackage{amsbsy}
\usepackage{amsthm}
%\usepackage[dvips]{graphicx}
\bibliographystyle{chicago}
\renewcommand{\baselinestretch}{1.2}

% left top textwidth textheight headheight % headsep footheight footskip
\setmargins{3.0cm}{2.5cm}{15.5 cm}{23.5cm}{0.5cm}{0cm}{1cm}{1cm}

\pagenumbering{arabic}


\begin{document}
\author{Kevin O'Brien}
\title{PRMIA IV}

\tableofcontents \setcounter{tocdepth}{2}

\newpage
\chapter{Case Studies}
\section{Barings}
Nick Leeson , commodity trader in Singapore for Barings Banks.

Barings The Long Strangle position would have partially hedged
Nick Leesons primary option position at Barings.
\section{NAB}
In net terms the currency options desk was exposed to a weakening
US dollar/strengthening Australian dollar.

\subsection{One-Hour Wwindow } The National’s VaR limit approved by
the Board was A\$80m, of which A\$3.25 million was allocated to
the currency options desk. The traders used the time between 9:00
am and 10:00 am, referred to as the 'one-hour window', to amend
the incorrect deal rates and reverse the false transactions in
Horizon to prevent their detection by the checking process.

\section{Bank Gesellscahft Berlin}
\subsection{Creation and Structure}
Bankgesellschaft Berlin was created in 1994 by merging the private
Berliner Bank with several institutions owned by the regional
government of Berlin. Among the notable public banks included were
the state bank, or Landesbank Berlin, and the Berliner
Hypothekenbank mortgage bank.
\subsubsection{Bank for a New Capital}
For Berlin's political elite, the new BgB offered a potent tool
for the rebirth of a reunited city that aspired to become
Germany's financial and commercial, as well as its political,
capital. \\ German banks have often worked in close partnership
with politicians in the post-war years, a combination that has
been cited as an important factor in building modern Germany's
economic might. But in BgB's case, the promotion of regional
growth led to wheeling and dealing that undermined rigorous bank
risk management.
\subsubsection{Hybrid Structure} The mix of
public and private interests in the new E200 billion-asset bank
meant that while Berlin's regional government owned the majority
of the shares in the bank, the new institution was also publicly
listed. That made it an unusual hybrid in the German banking
system, which generally offers a clear division between public
banks (backed by state guarantees) and their private competitors.

\subsection{Aubis AG}
The concentration of property-related risk in BgB's portfolio also
arose from conventional loans to property developers, such as
Aubis AG.
\\
In 1995-96 Aubis persuaded the Berlin Hypo, controlled by veteran
bank executive Klaus Landowsky, to extend loans worth hundreds of
millions of euros to fund the refurbishing of Cold War period
flats in the Berlin region, with the aim of generating premium
rental income.
\subsubsection{Conflict of Interest} Landowsky
was not just a bank executive; he had also built a parallel career
as a regional political grandee, and was entrenched in the Berlin
senate as the long-term leader of the CDU parliamentary party,
which controlled the coalition that dominated Berlin's political
life from 1991 until the summer of 2001 .
\\
Later investigations revealed that the two businessmen who
controlled Aubis (and who themselves had strong historical links
to the CDU) had made a campaign donation to Landowsky's party .
Landowsky later strongly maintained that the donation had nothing
to do with his role at the bank, or with the bank's decision to
grant credit to the property developers.
\subsubsection{Takeover of Assets}
But the Berlin Hypo's decision to extend credit to the property
developer proved expensive for the bank when Aubis found itself
unable to rent out its refurbished properties at appropriate
rates. In 1999, with the East German and Berlin property market in
freefall and Aubis in deep trouble, BgB took over the company's
under-performing assets rather than force the property company
into bankruptcy. As with many other BgB property-linked deals, the
recovery rate on the loan was poor.
\subsection{Immobilien und Beteiligungen AG (IBAG)}
In late 2000, the bank embarked on a bold plan to re-engineer this
increasingly ominous risk portfolio. The idea, developed with
investment bankers, was to sell off the more profitable operating
business assets of property subsidiary IBG, using an offshore
investment vehicle called Immobilien und Beteiligungen AG (IBAG),
which could subsequently be listed. The money from the sale could
then be used to plug the hole in the property-linked liabilities
that remained in the bank's portfolio. \\On 2 January 2001, the
business press duly reported that BgB's real estate business had
been transferred to IBAG in preparation for the spin-off


\section{Riggs Bank}

Riggs Riggs bank opened multiple accounts and accepted millions of
dollars in deposits from Mr. Pinochet with no serious inquiry into
questions regarding the source of his wealth. Riggs bank helped
Mr. Pinochet set up offshore shell corporations and open accounts
in the names of those corporations to disguise his control of the
accounts. Riggs bank altered the names of his personal accounts to
disguise their ownership. Riggs bank conducted transactions
through Riggs’ own accounts to hide Mr. Pinochet’s involvement in
some cash transactions Riggs did produce three KYC client profiles
prepared during 1998, 1999, and 2002. Riggs did not produce any
KYC documentation related to the opening of this account in
December 1994 in the United States.
\subsection{General Pinochet}
Riggs bank failed to conduct a 'Know Your Customer' assessment.

\subsection{Equatorial Guinea}
\section{Continental Illinois / Penn Square}
\section{Credit Lyonnais}
Credit Lyonnais
\subsection{MGM Studios}
Crédit Lyonnais became the leading lender to Hollywood studios in
the 1980s. It also financed Giancarlo Parretti's takeover of MGM
in 1990 for \$1.25 billion. However, Paretti started looting the
company, fired most of the accounting staff and appointed his
21-year-old daughter to a senior financial post and used company
money to buy presents for several girlfriends.\\\\In June 1991, CL
finally had enough, and decided to assume ownership of MGM, fired
Paretti and began a lawsuit against him. Credit Lyonnais's sudden
emergence as the de facto owner of the world's most famous movie
studio broke like an intense summer storm in the French
press.\\Overall, CL lost \$5 billion from its Hollywood deals.

\subsection{Consortium de Réalisation (CDR)}
The bank's finances were saved from disaster by moving its debts
and liabilities into a new state-owned company, Consortium de
Réalisation (CDR). The CDR was a highly controversial creation, as
many did not believe that the French government should have bailed
out the bank.To allow the bailout, the European commission imposed
severe limitations, principally on the international activities,
and the bank was forced to sell many entities in the following
years.
\\
The asset management company CDR was created to Prepare the bank
for privatization. The asset management company CDR would take
over troubled loans of Credit Lyonnais to the tune of 135-200
billion Ffr CDR would be under the management of Credit Lyonnais.
Later in August 1995, CDR \& CL were fully separated.
\\
 The new Credit Lyonnais would lend to CDR 100 billion Ffr at below market rates. Later under a new bail out package,
CL was allowed to charge normal interest rates to CDR

\subsection{Fire}
On May 5, 1996 a major fire destroyed much of Crédit Lyonnais'
Paris headquarters. The fire began in the main trading room of the
bank and was one of the worst fires to damage a Paris building in
25 years. It burned for over 12 hours, destroying two-thirds of
the building, along with crucial bank archives and computer data.

\section{US Saving and Loan Crisis}
The Garn-St. Germain Depository Institutions Act of 1982 is an Act
of Congress, that deregulated the Savings and Loan industry. This
Act turned out to be one of many contributing factors that led to
the Savings and Loan crisis of the late 1980s.

\section {LTCM}
Long-Term Capital Management was a U.S. hedge fund which used
trading strategies such as fixed income arbitrage, statistical
arbitrage etc.\\ In 1998 it lost $\$4.6$ billion in less than four
months following the Russian financial crisis. \\This lead to a
massive bailout by other major banks and investment houses,
 which was supervised by the Federal Reserve.
\\
\\
Despite the presence of Nobel laureates closely identified with
option theory what did LTCM seem to rely too much on for assessing
risk: - Theoretical market-risk models stress testing

\subsection{Background}
LTCM was founded in 1994 by John Meriwether, the former
vice-chairman and head of bond trading at Salomon Brothers.Board
of directors members included Myron Scholes and Robert C. Merton,
who shared the 1997 Nobel Memorial Prize in economic sciences.

\subsection{Bailout}
Goldman Sachs, AIG and Berkshire Hathaway offered then to buy out
the fund's partners for \$250 million, to inject \$3.75 billion
and to operate LTCM within Goldman's own trading division. The
offer was rejected and the same day the Federal Reserve Bank of
New York organized a bailout of \$3.625 billion by the major
creditors to avoid a wider collapse in the financial markets.

\subsection{Aftermath } UBS lost \$690 million according to
newspaper reports.


\section{Bankers Trust(1994)}
The bank suffered irreparable reputational damage in early 1994,
when some complex derivative transactions caused large losses for
some major corporate clients. \\\\Two of these - Gibson Greetings
and Procter \& Gamble (P\&G) - successfully sued BT, asserting
that they had not been informed of or [in the latter case] had
been unable to understand the risks involved. The bank's row with
P\&G made the front page of major US magazines. \\\\This was
worsened when several Bankers Trust bankers were caught on tape
remarking that their client [Gibson Greetings] would not be able
to understand what they were doing.



\section{Orange County}
\section{Daiwa (1995)}
In 1995, one of Daiwa Bank's bond traders, Toshihide Iguchi, in
New York lost $\$1.1$ billion speculating in the bond market. The
company was later indicted for not reporting crimes by Iguchi
including selling unauthorized sale of client's securities to
cover losses.
\\
Toshihide Iguchi is a former Japanese government bond trader at
Daiwa Bank responsible involving 30,000 unauthorized trades over a
period of $11$ years beginning in $1984$. In $1995$, he was
sentenced to four years in jail for fraud and falsifying
documents. He also received $\$2.6$ million in fines.

\section{California Power Crisis}
California's population increased by 13\% during the 1990s, the
state did not build any new major power plants during that
time.\\By keeping the consumer price of electricity artificially
low, the California government discouraged citizens from
practicing conservation.\\Shortages of energy supply for
California resulted in rolling blackouts and brownouts. Rolling
blackouts began in June 2000 and recurred several times in the
following 12 months. Price instability and spikes lasted from May
2000 to September 2001.
\\
\\
\\
The utility companies suffered major losses because a gap opened
up between the rates that Californian utilities were allowed to
charge consumers, and the price they had to pay for supplies in
the wholesale electricity market. Orange County Robert Citron's
bet on interest rates was Short-term interest rates will be low as
compared to medium-term interest rates.
\subsection{Deregulation}
When California deregulated in 1996, public utility companies
could now sell electricity to neighboring states such as Oregon,
Washington and Nevada at a higher profit.
\\
Deregulation required the Investor Owned Utilities, or IOUs,
(primarily Pacific Gas and Electric[PG\&E], Southern California
Edison [SCE], and San Diego Gas and Electric[SDG\&E]) to sell off
a significant part of their electricity generation to wholly
private, unregulated companies such as AES, Reliant, and Enron.\\
The buyers of those power plants then became the wholesalers from
which the IOUs needed to buy the electricity that they used to own
themselves.
\subsection{Brown Outs}
Shortages of energy supply for California resulted in rolling
blackouts and brownouts. Rolling blackouts began in June 2000 and
recurred several times in the following 12 months. Price
instability and spikes lasted from May 2000 to September 2001.


\section{MetallGesellschaft(MG)} In 1993, the company lost over
1.4 billion dollars after speculating increase in oil price in oil
futures market. A subsequent drop in oil price left the company
buying the oil at a higher price than the market price.\\\\
Metallgesellschaft revealed publicly that its “Energy Group” was
responsible for losses of approximately \$1.5 billion. These
losses were due mainly to cash-flow problems resulting from large
oil forward contracts it had written.
\\
\\
MG absorbed which of the following risks in so far as the
structure of MG's deals would create losses if the futures price
curves of each of the commodities shifted from mostly
backwardation to mostly contango. \\I. Crude commodity price risk
\\II. Gasoline commodity price risk \\III. Heating oil commodity
price risk IV. Price structure risk During 1993, an intra-OPEC
battle for market share lead to excess supply of crude in physical
market that in turn lead to serious contango.
\section{Worldcom}
WorldCom achieved its position as a significant player in the
telecommunications industry through the successful completion of
65 acquisitions. \\Between 1991 and 1997, WorldCom spent almost
\$60 billion in the acquisition of many of these companies and
accumulated \$41 billion in debt.  \\
Mergers and Acquisitions, especially large ones, present
significant managerial challenges in at least two areas. First,
management must deal with the challenge of integrating new and old
organizations into a single smoothly functioning business. \\The
second challenge is the requirement to account for the financial
aspects of the acquisition.

\chapter{PRMIA IV-B}
\section{Group of Thirty Best Practices }
\subsection{Structure}
The Global Derivatives Study consists of
\begin{itemize}
\item 20 Recommendations for dealers and end users \item 4
Recommendation for regulators \item An Overview of Derivatives
Activity  - what derivatives are, the needs they serve, their
risks, and their relationship to traditional financial
instruments\item Three Appendices
\end{itemize}

The Study offers \textbf{20 recommendations} to help dealers and
end-users manage derivatives activity and continue to benefit from
its use.

\subsection{The Recommendations}
1) Recommendations for Management -
\\
2) \textbf{Marking to Market}-  Dealers should mark their
derivatives positions to market, on at least a daily basis, for
risk management purposes.
\\ Lower-of-cost-or-market accounting, and accruals
accounting, are not appropriate for risk management.
\\\\3)\textbf{Market Valuation Methods}
Derivatives portfolios should be valued based on mid-market levels
less specific adjustments, or on appropriate bid or offer levels.
Mid-market valuation adjustments should allow for expected future
costs such as unearned credit spread, close-out costs, investing
and funding costs, and administrative costs.
\\\\4)\textbf{Identifying Revenue Sources }- Dealers should measure the components of revenue regularly and in
sufficient detail to understand the sources of risk.
\\\\
5)\textbf{Measuring Market Risk} - Dealers should use a consistent
measure to calculate daily the market risk of their derivatives
positions and compare it to market risk limits.\\\\\indent Market
risk is best measured as "value at risk" using probability
analysis based upon a common confidence interval (e.g., two
standard deviations) and time horizon (e.g., a one-day exposure).
\\\indent Components of market risk that should be considered
across the term structure include: absolute price or rate change
($\delta$); convexity ($\gamma$); volatility (vega); time decay
($\theta$); basis or correlation; and discount rate ($\rho$).
\\\\
6)\textbf{Stress Simulations} -  Dealers should regularly perform
simulations to determine how their portfolios would perform under
stress conditions.
\\
Assumptions that are valid for normal markets may no longer hold
true in abnormal markets. These simulations should reflect both
historical events and future possibilities.
\\\\
6)\textbf{Investing and Funding Forecasts} - Dealers should
periodically forecast the cash investing and funding requirements
arising from their derivatives portfolios.
\\\\
7)\textbf{Investing and Funding Forecasts} - Dealers should
periodically forecast the cash investing and funding requirements
arising from their derivatives portfolios.
\\\\
8  Dealers should have a market risk management function, with
clear independence and authority, to ensure that risk management
responsibilities.
\\\\
9)\textbf{Practices by End-Users} - As appropriate to the nature,
size, and complexity of their derivatives activities, endusers
should adopt the same valuation and market risk management
practices that are recommended for dealers.
\\\\
10)\textbf{Measuring Credit Exposure} - Dealers and end-users
should measure credit exposure on derivatives in two ways.\\-
Current exposure, which is the replacement cost of derivatives
transactions, that is, their market value\\-Potential exposure,
which is an estimate of the future replacement cost of derivatives
transactions. It should be calculated using probability analysis
based upon broad confidence intervals.
\\\\
11)\textbf{Aggregating Credit Exposures} -  Credit exposures on
derivatives, and all other credit exposures to a counterparty,
should be aggregated taking into consideration enforceable netting
arrangements. Credit exposures should be calculated regularly and
compared to credit limits.

\subsection{The Appendices} Appendix I contains the six working
papers that form the basis of the Study and whose analysis
supports the Recommendations. Each covers one of the main areas
studied by the Working Group:
\begin{itemize}
\item valuation and market risk\item 4 credit risk \item
enforceability \item systems, operations, and controls \item
accounting and reporting \item systemic issues.
\end{itemize}
Appendix II is a compilation of legal memoranda discussing issues
of enforceability in nine jurisdictions: Australia, Brazil,
Canada, England, France, Germany, Japan, Singapore, and the United
States.
\\\\Appendix III summarizes the findings of the Survey of Industry
Practice conducted for this Study. A representative group of 80
dealers and 72 end-users responded to a detailed questionnaire.

\section{Governance Principles}
\section{Standards of Best Practice, Conduct and Ethics}

According to the G-30, \textbf{Marking to Market} is the only
valuation that correctly reflects the current value of derivatives
cash flows to be managed.\\ It also provides information about
market risk and appropriate hedging actions.
\\
\\
Derivative credit exposure should be measured by Current Exposure
and Potential Exposure







\end{document}
