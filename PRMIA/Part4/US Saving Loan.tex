







Introduction

In February 1989, newly elected President George Bush announced to the American public that he would set up a programme to rescue the stricken Savings & Loan industry.

The underwriting of US thrifts by the financial industry and the US taxpayer cost an extraordinary $153 billion. The extent of the disaster turned it into a major threat to the US financial system, and one of the most expensive financial sector crises the world has seen.

The losses were the result of unmanaged asset/liability gaps that led to interest rate exposures, speculative investments in junk bonds and service industries, fraud, and - most especially - massive losses from lending to and investing in the US commercial real estate sector.
 






Background

The S&L industry was a conservative residential mortgage sector, surrounded by legislation put in place in the 1930s to promote home ownership.

The sector had its own regulator, the Federal Savings & Home Loan Banking Board (FSHLC), and its own insurance fund to protect depositors, the Federal Savings and Loan Insurance Corporation (FSLIC).

These agencies were funded by the S&L industry and backed by the US taxpayer.

But the sleepy S&L industry was the child of a particular regulatory and interest rate environment, and between 1960 and 1980 that environment changed out of recognition. From the early 1960s there were growing worries that the S&L industry was not competing effectively for funds with commercial banks and securities markets, leading to large swings in the amount of money available for mortgage lending.3 But the real threat emerged in the 1970s as inflation joined forces with the deregulation of US interest rate markets to produce an increasingly volatile interest rate market.

Sharp price movements demand new risk management structures, skills and tools. But as interest rates became more volatile, particularly in the late 1970s, the S&L industry failed to tackle the risk inherent in the funding of long-term, fixed-interest mortgages by means of short-term deposits.

Regulation Q

Regulation Q , introduced in the 1960s, intended to help the S&L sector had put a ceiling on the interest rate that S&Ls could offer to depositors. This measure succeeded for a while in dampening competition for depositor funds between banks and S&Ls.

But as new money market funds began to compete fiercely during the 1970s for depositors' money by offering interest rates set by the market, S&Ls suffered significant withdrawal of deposits during periods of high interest rates - a process known as disintermediation.

Liquidity Gap

After deposit interest rate ceilings were eliminated between 1980 and 1982  S&Ls tried to compete for funds by offering above-market rates, an unsustainable gap opened up between the cost of their funding liabilities (short-term interest rates) and the income generated by their assets (long-term, fixed-rate mortgage repayments).

 

Worse, as interest rates moved higher, the economic value of existing S&L portfolios of long-term, low interest rate residential mortgages moved sharply lower, threatening institutions with insolvency.

Oil Shock of 1979

The trigger for the closing shut of this asset/liability trap was the shock rise in oil prices in 1979, pushing up inflation and headline interest rates around the world. By 1980, with interest rates on US government debt hitting 16%, many S&Ls had already been fatally wounded.

Luckily for the owners of thrifts, regulators in the early 1980s lacked the political, financial or human resources to close large numbers of institutions. Rigorous enforcement would have meant paying out much more money to insured depositors than was held in the industry-funded FSLIC insurance fund. It would also have meant working with literally hundreds of insolvent institutions, and overcoming numerous political obstacles at a federal and state level to radical S&L industry reform.

Instead, between 1980 and 1982, regulators, industry lobbyists and legislators put together various legislative and regulatory mechanisms to postpone the threatened insolvency of the sector in the hope that interest rates would quieten down and S&Ls would be able to engineer themselves back into profitability.

As we recount in our accompanying Timeline, the sum effect of these mechanisms was to loosen S&L capital restrictions, while offering S&Ls new freedom to extend their activities into potentially lucrative (and therefore risky) areas.

In particular, regulatory rules on 'net worth' - the amount left over when S&L liabilities were subtracted from assets and taken as a key indicator of solvency - were changed so that thrifts could continue to operate even when their net worth reached historically low levels.4

The loosening of the solvency and risk capital regime surrounding S&Ls also included important changes to the treatment of an accounting concept known as 'supervisory goodwill'. Supervisory goodwill helped to compensate any institution that, in a regulator-agreed merger, took on the economically impaired tangible assets of another, insolvent institution (such as mortgages paying low rates of interest). Although largely meaningless at an economic level, supervisory goodwill could be used to balance out the thrift's books in terms of its capital requirements and its accounting numbers.

Indeed, changes to the accounting and capital treatment of supervisory goodwill in 1982 made it possible for acquiring thrifts to post stronger apparent accounting and capital numbers for up to ten years after merging with a failing institution, even though their underlying economic situation had deteriorated.5

The legislative moves of the early 1980s also included the raising of the level of insured deposits from $40,000 to $100,000. The issue of deposit insurance is critical to the S&L scandal. In an uninsured environment, depositors would have been wary of continuing to fund the industry, whatever the rate of interest paid by the S&Ls, for fear of losing their savings in a collapsed institution.

But because savers knew their deposits were insured by government guarantees, even badly damaged institutions could attract funds by paying interest rates marginally above the market rate. The increase in the insured amount, and the phasing out of interest rate ceilings on the interest that S&Ls could pay depositors after 1980, helped the S&Ls to attract depositors through both traditional and non-traditional means.

In particular, S&Ls began to access increasing amounts of their funding via brokered deposits. In this market, which emerged through the 1980s, brokers gathered together deposits from individual savers and channelled these to institutions that offered higher rates of interest - significantly increasing the rate at which S&Ls could take in deposits and build their business.

But the loosening of regulatory restrictions on S&L activity meant that institutions could use these new funds to gamble their way into profit. The S&L industry's focus began to shift away from mortgage assets and towards assets that were more immediately lucrative (and risky).

In particular, from the early 1980s, S&Ls began to both lend to real estate developers and to invest in real estate, construction and service companies. In key regions, such as Texas and Florida, S&L lenders competed with other lenders such as commercial bankers to fuel a real-estate boom, as US investors queued to take advantage of a 1981 change in federal tax laws that rewarded investments in construction.

But although this lending sector offered sweet upfront fees and relatively high interest rate margins, it was a honey trap that could be sprung by any marked downturn in the commercial real estate market. Furthermore, as both the credit quality of developers and the value of the bank's security depended upon the same fundamental risk factors - property values, rental income, occupancy rates - the loss rates on commercial real estate loans were likely to prove savage in any slump.

An additional problem was that while the regulations surrounding the S&L industry grew more and more lax, the already-limited quality of the on-site supervision of individual institutions declined in the early to mid 1980s, partly due to regulator budgetary restraints.6

With the restrictions eased, supervision at a low ebb, and little funding market discipline (as a result of deposit insurance), the sector's main bulwark against poor decisions became the integrity and internal risk management practices of individual S&Ls.

In too many cases, these proved feeble defences. In particular, the traditional risk management skills of mortgage lenders, where credit risk is relatively low and predictable, and property and collateral prices relatively stable, did not equip most S&Ls to venture into the strongly cyclical commercial real estate market.

Across the S&L industry, many institutions allowed bad practices to evolve that allowed economic risks to be underestimated and lie unrecorded, while dubious and fraudulent revenues were recorded up front.

In relation to commercial real estate lending, these practices included:

∙        Over-emphasis on the up-front fees generated from advancing commercial real-estate loans (fees that were often paid from the money that the bank itself loaned to the developer).

∙        Loosening of underwriting standards which should have ensured creditors were likely to possess robust cashflows from the development, and which should have limited the size of any loan to a fraction of the value of any property used as security.

∙        The use of untrustworthy property value appraisals that often took little account of the likely downward movements of property prices in local markets but sometimes included speculative assumptions about upward movements (misvaluation of assets with uncertain values is a common theme in failed bank risk management)

∙        The practice of adding unpaid interest payments to the capital to be repaid on a loan, and a host of similar practices designed to prop up apparent asset quality and revenue streams even as an S&L's asset portfolio began to deteriorate

∙        Churning impaired loans by using bank credit to persuade developers to purchase shaky bank collateral and investments at inflated values (from the bank itself, or the impaired credit).

Out and out corruption also played its part, both in terms of direct economic losses - it is estimated to account for at least 15% of the total S&L loss, some put the figure much higher - and in setting the scene for reckless decision making, misvaluation and deliberately obscure financial reporting and documentation.7

From 1982, the FHLBB jettisoned many regulations concerning S&L ownership, so that individuals and small powerful groups of shareholders could more easily gain control of institutions. But the opportunity to grow a financial institution fast from a minimal capital base - leveraging the risk and potential return of any initial investment - attracted the wrong style of management and owner to the industry. Even while the underlying economics of the S&L industry remained poor, the relaxed rules meant that between late 1982 and late 1985 many new thrifts entered the market - 133 in 1984 alone - and the total asset-base of the thrift industry grew by 56%.8

After a short respite in 1983 to 1984, as a more favourable interest rate environment helped ease the original cause of the S&L malaise, the mid to late 1980s saw S&Ls lurch back into crisis once more. This time round, a series of US regional crises, triggered by collapses in the oil, property and farming sectors, acted to realise the credit and investment risks now embedded in S&L portfolios. By 1984, for example, the oil-price inspired boom of the early 1980s in Texas was faltering, and by 1987 that state's oil and real estate sectors were in deep recession. A similar process of increasing rates of default and falling collateral values remorselessly undermined S&L asset around the US, right through until 1992 (though Texan S&Ls remained among the worst hit).9

From late 1984, the FHLBB began to try to tighten up on S&L risk management and to put a brake on risky investment activities at near-insolvent S&Ls. But it was too little, too late. By the end of 1986 it was clear that the FSLIC, the safety net that insured S&L depositors - and which regulators depended upon when resolving failed institutions - was itself insolvent in the face of overwhelming sector losses.

As our Timeline recounts, from 1986 onwards, politicians and regulators struggled with a series of measures to fund the restructuring of the industry, but these failed to match up to the scale of the problem. Although by 1988 the FSLIC was able to resolve thrifts with assets of $96 billion, much of the restructuring was accomplished through encouraging mergers and acquisitions between S&Ls (rather than by confronting the fundamental problem that much of the sector was insolvent).

S&L owners and managers generally opposed moves to restructure the industry through the wholesale closure of financially precarious institutions, and proved to be powerful political lobbyists.

In one incident in 1987 that has come to be seen as symbolic of dubious lobbying during the S&L crisis, the chairman of the FHLBB and other key regulators were taken to task by five leading senators over the pressure that regulators were exerting on the activities of a specific S&L, Lincoln Savings & Loan. The meeting had been set up by the controller of the S&L, Charles Keating, a politically well-connected developer, and his assembling of the senators - known in later press accounts as the 'Keating Five' - was interpreted by some of the regulators as a show of force. Lincoln Savings & Loan was to become one of the largest S&L failures when it was closed in 1989, sparking a series of major court cases.

Financial Institutions Reform, Recovery and Enforcement Act of 1989 (FIRREA),

After President Bush's speech,  Congress passed the Financial Institutions Reform, Recovery and Enforcement Act of 1989 (FIRREA), substantially restructuring US financial industry regulation.

Under FIRREA, the discredited FHLBB and the insolvent FSLIC were abolished in favour, respectively, of the new Office of Thrift Supervision and an extended function for the FDIC (already responsible for deposit insurance in other US banking sectors),11 while the Resolution Trust Corporation was set up to liquidate hundreds of insolvent institutions.

With new powers and funding, regulators began to act aggressively to close down institutions, though it quickly became clear that the situation in the S&L industry was even worse than had been imagined.

In 1989 and 1990 the S&L crisis reached its height in terms of public expense, with the RTC resolving 318 thrifts with total assets of $135 billion in 1989 and 213 thrifts with total assets of $130 billion in 1990.

The drain on the public purse continued into 1993. But market fundamentals in the form of the interest rate environment, and the bottoming out of regional economies and real estate markets, were beginning to turn up again for S&L asset portfolios. From 1993 to 1995 the industry began to stabilise, and the portions of the industry that had survived the great S&L crisis moved steadily back towards profit over the next few years. 







Cost

Setting aside ongoing legal action, the thrift crisis cost an extraordinary $153 billion - easily the most expensive financial sector crisis the world has ever seen. Of this, the US taxpayer paid out $124 billion while the thrift industry itself paid $29 billion.12

Aftermaths

The consequences of the S&L crisis for the structure and regulation of the US financial industry were profound. The number of institutions in the S&L industry fell by about half between 1986 and 1995, partly due to the closure of around 1000 institutions by regulators, the most intense series of institution failures since the 1930s.13 The failures prompted an overhaul of the regulatory structure for US banking and thrifts, a shake-up in the system of deposit insurance and implied government guarantees, and a series of legal battles and corruption scandals fought out in the courts.

Regulators shifted towards a policy of earlier intervention in failing institutions so that the principal costs are more likely to be borne by shareholders than other stakeholders. There was also a shift towards more risk-sensitive regulatory regimes with respect to both net-worth assessments and the payments made by individual institutions to deposit insurance funds, while deposit insurance reform made it less likely that taxpayers would shoulder so great a burden in any future crisis.14

At a wider level, the S&L crisis taught politicians, regulators and bankers how misleading rules-driven regulatory and accounting numbers can be in relation to risky bank activities. At different stages of the crisis, and at many different levels from bank executives through to regulators and politicians, a formalistic reporting of the financial condition of S&Ls was deliberately selected by interested parties to cover up the true economic extent of the unfolding disaster. It was a risk-reporting failure on a grand scale that greatly worsened the long-term economic consequences for the ultimate stakeholder: the US taxpayer.



 






Lessons learned

∙        Regulatory capital and accounting numbers are not a good guide to risk-adjusted profitability in the banking industry

∙        It's easy for wounded institutions to move out of the frying pan and into the fire by taking cash up front for assuming long-term or unreported risks

∙        Poorly controlled lending institutions can easily 'recycle' poor loans (by granting more credit) and fabricate fee income (by churning transactions) to disguise their true level of risk and return

∙        Rapid growth into new lines of business signals the need for tighter risk management and financial controls

 






Timeline of Events

1932: Federal Home Loan Bank Act establishes Federal Home Loan Bank Board (FHLBB) 1933: Home Owner's Act promotes home ownership via mortgage loans offered by savings & loans associations regulated by the FHLBB

1934: National Housing Act sets up Federal Savings and Loan Insurance Corporation (FSLIC) to insure deposits at S&L institutions

1960s: Congress applies Regulation Q to the S&L industry to put a ceiling on the interest rate that S&Ls can pay to depositors.

1970s: Congress deregulates interest rates opening up potential asset/liability and interest rate risks for S&Ls, but politicians fail to act on various studies and commissions recommending a mix of consolidated supervision and liberalised regulation of the sector.

1979-1982: Sharply raised interest rates lead to an asset/liability crisis at many S&Ls that is at its worst in 1980 to 1982.

November 1980: Following the March enactment of the Depository Institutions Deregulation and Monetary Control Act of 1980 (DIDMCA), which allowed the Bank Board to ease the previous statutory 5% of net worth requirement to anywhere between 3% and 6%, FHLBB eases 'net worth' rules to only 4% of insured accounts. DIDMCA also raises the bar on federally insured deposits from $40,000 to $100,000 and allows some S&Ls to put money into property development and other risky activities.

1981: Changes in federal tax regulations under the Economic Recovery Tax Act of 1981 help spark the beginnings of the real-estate boom of the early to mid 1980s.

September 1981: FHLBB introduces various rules and accounting changes to make the financial condition of S&Ls look better, including allowing the deferral of losses from the sale of impaired assets over a ten-year period, and the issuance of capital 'certificates' that artificially boost apparent capitalisation.

January 1982: Net worth rules eased again to only 3% of insured accounts.

July 1982: FHLBB allows S&Ls to amortise 'supervisory goodwill' over a period of up to 40 years, up from an original 10-year restriction. Garn-St Germain Depository Institutions Act of 1982 allows easing of capital rules, and greatly eases restrictions on the proportion of a property's value that S&Ls can loan to a property developer. Deposit interest rate ceilings (Regulation Q) phased out for S&Ls, enabling them to compete for wholesale funds by offering high rates of interest.

Late 1982: FHLBB starts to count equity capital as part of an S&L's reserves

January 1983: Restrictions lifted on state-chartered S&Ls in California with regard to investments in property and service companies, as state legislators compete with federal legislators to ease restrictions on S&Ls.

1983: Interest rates fall, temporarily making some - though not all - of the S&L industry solvent on an economic basis. But the opportunity for rational closure of institutions and reform of healthy institutions is missed.

Late 1984 and after: Regulators begin to tighten up regulations to try to prevent weaker institutions making rash loans and investments following a number of attention-grabbing S&L collapses.

1984-89: S&Ls pay above-market rates to attract deposits, particularly in hot spots such as the Texas S&L industry. It's clear that the industry is in deep trouble but its regulators lack resources and political backing to close insolvent institutions quickly enough.

1986: FSLIC, itself clearly insolvent by year-end 1986, resolves 54 thrifts with total assets of around $16 billion. But far more thrifts are insolvent according to their book values, while many others hover on the brink of book insolvency. The economic reality is even worse, with perhaps half the industry now under the water.

1986-1992: During the later 1980s, the real-estate bubble bursts in regions around the US, partly prompted by the passing of the Tax Reform Act in 1986, which removes federal tax incentives to invest in commercial real estate.

1987: The passing of the Competitive Equality Banking Act, and the setting up of a Financing Corporation (FICO) to fund FSLIC resolution of failing thrifts by means of issuing bonds, channel some limited resources to the programme of S&L closure, but the emphasis remains on keeping wounded S&Ls afloat.

1988: Regulators resolve 185 thrifts with total assets of $96 billion, but it's not enough to stabilise the industry and many resolutions continue to be by means of regulator-agreed acquisition: sharing rather than ending the economic woe.

1989-1990: In terms of public expense, the S&L crisis is at its height. RTC resolves 318 thrifts with total assets of $135 billion in 1989 and 213 thrifts with total assets of $130 billion in 1990.

1990-92: RTC continues to resolve large numbers of thrifts, but the annual figure for 1992 falls to 59 institutions with $44 billion assets.

1993-95: The number of thrifts requiring RTC intervention falls away sharply to only 13 over this three-year period as industry fundamentals begin to improve. The crisis is over, but legal wrangling over the restructuring process will continue into the next millennium.
 

 



%==================================================================%


% PRMIA 4

US Saving and Loan (1980s)

Fallout estimated in the regionof USD 30-50 billion, but actually turned out to be $153 billion.

Losses were result of unmanaged asset liability gaps that led to interest rate exposures.



%---------------------------------------------------------

Introduction
In February 1989, newly elected President George Bush announced to the American public that he would set up a programme to rescue the stricken Savings & Loan industry. 
The underwriting of US thrifts by the financial industry and the US taxpayer cost an extraordinary $153 billion. The extent of the disaster turned it into a major threat to the US financial system, and one of the most expensive financial sector crises the world has seen. 
The losses were the result of unmanaged asset/liability gaps that led to interest rate exposures, speculative investments in junk bonds and service industries, fraud, and - most especially - massive losses from lending to and investing in the US commercial real estate sector.
 
Background
The S&L industry was a conservative residential mortgage sector, surrounded by legislation put in place in the 1930s to promote home ownership. 
The sector had its own regulator, the Federal Savings & Home Loan Banking Board (FSHLC), and its own insurance fund to protect depositors, the Federal Savings and Loan Insurance Corporation (FSLIC). 
These agencies were funded by the S&L industry and backed by the US taxpayer. 
But the sleepy S&L industry was the child of a particular regulatory and interest rate environment, and between 1960 and 1980 that environment changed out of recognition. From the early 1960s there were growing worries that the S&L industry was not competing effectively for funds with commercial banks and securities markets, leading to large swings in the amount of money available for mortgage lending.3 But the real threat emerged in the 1970s as inflation joined forces with the deregulation of US interest rate markets to produce an increasingly volatile interest rate market. 
Sharp price movements demand new risk management structures, skills and tools. But as interest rates became more volatile, particularly in the late 1970s, the S&L industry failed to tackle the risk inherent in the funding of long-term, fixed-interest mortgages by means of short-term deposits. 
Regulation Q
Regulation Q , introduced in the 1960s, intended to help the S&L sector had put a ceiling on the interest rate that S&Ls could offer to depositors. This measure succeeded for a while in dampening competition for depositor funds between banks and S&Ls.
But as new money market funds began to compete fiercely during the 1970s for depositors' money by offering interest rates set by the market, S&Ls suffered significant withdrawal of deposits during periods of high interest rates - a process known as disintermediation. 
Liquidity Gap
After deposit interest rate ceilings were eliminated between 1980 and 1982  S&Ls tried to compete for funds by offering above-market rates, an unsustainable gap opened up between the cost of their funding liabilities (short-term interest rates) and the income generated by their assets (long-term, fixed-rate mortgage repayments). 

Worse, as interest rates moved higher, the economic value of existing S&L portfolios of long-term, low interest rate residential mortgages moved sharply lower, threatening institutions with insolvency. 
Oil Shock of 1979
The trigger for the closing shut of this asset/liability trap was the shock rise in oil prices in 1979, pushing up inflation and headline interest rates around the world. By 1980, with interest rates on US government debt hitting 16%, many S&Ls had already been fatally wounded. 
Luckily for the owners of thrifts, regulators in the early 1980s lacked the political, financial or human resources to close large numbers of institutions. Rigorous enforcement would have meant paying out much more money to insured depositors than was held in the industry-funded FSLIC insurance fund. It would also have meant working with literally hundreds of insolvent institutions, and overcoming numerous political obstacles at a federal and state level to radical S&L industry reform. 
Instead, between 1980 and 1982, regulators, industry lobbyists and legislators put together various legislative and regulatory mechanisms to postpone the threatened insolvency of the sector in the hope that interest rates would quieten down and S&Ls would be able to engineer themselves back into profitability. 
As we recount in our accompanying Timeline, the sum effect of these mechanisms was to loosen S&L capital restrictions, while offering S&Ls new freedom to extend their activities into potentially lucrative (and therefore risky) areas. 
In particular, regulatory rules on 'net worth' - the amount left over when S&L liabilities were subtracted from assets and taken as a key indicator of solvency - were changed so that thrifts could continue to operate even when their net worth reached historically low levels.4 
The loosening of the solvency and risk capital regime surrounding S&Ls also included important changes to the treatment of an accounting concept known as 'supervisory goodwill'. Supervisory goodwill helped to compensate any institution that, in a regulator-agreed merger, took on the economically impaired tangible assets of another, insolvent institution (such as mortgages paying low rates of interest). Although largely meaningless at an economic level, supervisory goodwill could be used to balance out the thrift's books in terms of its capital requirements and its accounting numbers. 
Indeed, changes to the accounting and capital treatment of supervisory goodwill in 1982 made it possible for acquiring thrifts to post stronger apparent accounting and capital numbers for up to ten years after merging with a failing institution, even though their underlying economic situation had deteriorated.5 
The legislative moves of the early 1980s also included the raising of the level of insured deposits from $40,000 to $100,000. The issue of deposit insurance is critical to the S&L scandal. In an uninsured environment, depositors would have been wary of continuing to fund the industry, whatever the rate of interest paid by the S&Ls, for fear of losing their savings in a collapsed institution. 
But because savers knew their deposits were insured by government guarantees, even badly damaged institutions could attract funds by paying interest rates marginally above the market rate. The increase in the insured amount, and the phasing out of interest rate ceilings on the interest that S&Ls could pay depositors after 1980, helped the S&Ls to attract depositors through both traditional and non-traditional means. 
In particular, S&Ls began to access increasing amounts of their funding via brokered deposits. In this market, which emerged through the 1980s, brokers gathered together deposits from individual savers and channelled these to institutions that offered higher rates of interest - significantly increasing the rate at which S&Ls could take in deposits and build their business. 
But the loosening of regulatory restrictions on S&L activity meant that institutions could use these new funds to gamble their way into profit. The S&L industry's focus began to shift away from mortgage assets and towards assets that were more immediately lucrative (and risky). 
In particular, from the early 1980s, S&Ls began to both lend to real estate developers and to invest in real estate, construction and service companies. In key regions, such as Texas and Florida, S&L lenders competed with other lenders such as commercial bankers to fuel a real-estate boom, as US investors queued to take advantage of a 1981 change in federal tax laws that rewarded investments in construction. 
But although this lending sector offered sweet upfront fees and relatively high interest rate margins, it was a honey trap that could be sprung by any marked downturn in the commercial real estate market. Furthermore, as both the credit quality of developers and the value of the bank's security depended upon the same fundamental risk factors - property values, rental income, occupancy rates - the loss rates on commercial real estate loans were likely to prove savage in any slump. 
An additional problem was that while the regulations surrounding the S&L industry grew more and more lax, the already-limited quality of the on-site supervision of individual institutions declined in the early to mid 1980s, partly due to regulator budgetary restraints.6 
With the restrictions eased, supervision at a low ebb, and little funding market discipline (as a result of deposit insurance), the sector's main bulwark against poor decisions became the integrity and internal risk management practices of individual S&Ls. 
In too many cases, these proved feeble defences. In particular, the traditional risk management skills of mortgage lenders, where credit risk is relatively low and predictable, and property and collateral prices relatively stable, did not equip most S&Ls to venture into the strongly cyclical commercial real estate market. 
Across the S&L industry, many institutions allowed bad practices to evolve that allowed economic risks to be underestimated and lie unrecorded, while dubious and fraudulent revenues were recorded up front. 
In relation to commercial real estate lending, these practices included: 
•	Over-emphasis on the up-front fees generated from advancing commercial real-estate loans (fees that were often paid from the money that the bank itself loaned to the developer). 
•	Loosening of underwriting standards which should have ensured creditors were likely to possess robust cashflows from the development, and which should have limited the size of any loan to a fraction of the value of any property used as security. 
•	The use of untrustworthy property value appraisals that often took little account of the likely downward movements of property prices in local markets but sometimes included speculative assumptions about upward movements (misvaluation of assets with uncertain values is a common theme in failed bank risk management) 
•	The practice of adding unpaid interest payments to the capital to be repaid on a loan, and a host of similar practices designed to prop up apparent asset quality and revenue streams even as an S&L's asset portfolio began to deteriorate 
•	Churning impaired loans by using bank credit to persuade developers to purchase shaky bank collateral and investments at inflated values (from the bank itself, or the impaired credit). 
Out and out corruption also played its part, both in terms of direct economic losses - it is estimated to account for at least 15% of the total S&L loss, some put the figure much higher - and in setting the scene for reckless decision making, misvaluation and deliberately obscure financial reporting and documentation.7 
From 1982, the FHLBB jettisoned many regulations concerning S&L ownership, so that individuals and small powerful groups of shareholders could more easily gain control of institutions. But the opportunity to grow a financial institution fast from a minimal capital base - leveraging the risk and potential return of any initial investment - attracted the wrong style of management and owner to the industry. Even while the underlying economics of the S&L industry remained poor, the relaxed rules meant that between late 1982 and late 1985 many new thrifts entered the market - 133 in 1984 alone - and the total asset-base of the thrift industry grew by 56%.8 
After a short respite in 1983 to 1984, as a more favourable interest rate environment helped ease the original cause of the S&L malaise, the mid to late 1980s saw S&Ls lurch back into crisis once more. This time round, a series of US regional crises, triggered by collapses in the oil, property and farming sectors, acted to realise the credit and investment risks now embedded in S&L portfolios. By 1984, for example, the oil-price inspired boom of the early 1980s in Texas was faltering, and by 1987 that state's oil and real estate sectors were in deep recession. A similar process of increasing rates of default and falling collateral values remorselessly undermined S&L asset around the US, right through until 1992 (though Texan S&Ls remained among the worst hit).9 
From late 1984, the FHLBB began to try to tighten up on S&L risk management and to put a brake on risky investment activities at near-insolvent S&Ls. But it was too little, too late. By the end of 1986 it was clear that the FSLIC, the safety net that insured S&L depositors - and which regulators depended upon when resolving failed institutions - was itself insolvent in the face of overwhelming sector losses. 
As our Timeline recounts, from 1986 onwards, politicians and regulators struggled with a series of measures to fund the restructuring of the industry, but these failed to match up to the scale of the problem. Although by 1988 the FSLIC was able to resolve thrifts with assets of $96 billion, much of the restructuring was accomplished through encouraging mergers and acquisitions between S&Ls (rather than by confronting the fundamental problem that much of the sector was insolvent). 
S&L owners and managers generally opposed moves to restructure the industry through the wholesale closure of financially precarious institutions, and proved to be powerful political lobbyists. 
In one incident in 1987 that has come to be seen as symbolic of dubious lobbying during the S&L crisis, the chairman of the FHLBB and other key regulators were taken to task by five leading senators over the pressure that regulators were exerting on the activities of a specific S&L, Lincoln Savings & Loan. The meeting had been set up by the controller of the S&L, Charles Keating, a politically well-connected developer, and his assembling of the senators - known in later press accounts as the 'Keating Five' - was interpreted by some of the regulators as a show of force. Lincoln Savings & Loan was to become one of the largest S&L failures when it was closed in 1989, sparking a series of major court cases. 
Financial Institutions Reform, Recovery and Enforcement Act of 1989 (FIRREA),
After President Bush's speech,  Congress passed the Financial Institutions Reform, Recovery and Enforcement Act of 1989 (FIRREA), substantially restructuring US financial industry regulation.
Under FIRREA, the discredited FHLBB and the insolvent FSLIC were abolished in favour, respectively, of the new Office of Thrift Supervision and an extended function for the FDIC (already responsible for deposit insurance in other US banking sectors),11 while the Resolution Trust Corporation was set up to liquidate hundreds of insolvent institutions. 
With new powers and funding, regulators began to act aggressively to close down institutions, though it quickly became clear that the situation in the S&L industry was even worse than had been imagined. 
In 1989 and 1990 the S&L crisis reached its height in terms of public expense, with the RTC resolving 318 thrifts with total assets of $135 billion in 1989 and 213 thrifts with total assets of $130 billion in 1990. 
The drain on the public purse continued into 1993. But market fundamentals in the form of the interest rate environment, and the bottoming out of regional economies and real estate markets, were beginning to turn up again for S&L asset portfolios. From 1993 to 1995 the industry began to stabilise, and the portions of the industry that had survived the great S&L crisis moved steadily back towards profit over the next few years.  

 
Cost
Setting aside ongoing legal action, the thrift crisis cost an extraordinary $153 billion - easily the most expensive financial sector crisis the world has ever seen. Of this, the US taxpayer paid out $124 billion while the thrift industry itself paid $29 billion.12 
Aftermaths
The consequences of the S&L crisis for the structure and regulation of the US financial industry were profound. The number of institutions in the S&L industry fell by about half between 1986 and 1995, partly due to the closure of around 1000 institutions by regulators, the most intense series of institution failures since the 1930s.13 The failures prompted an overhaul of the regulatory structure for US banking and thrifts, a shake-up in the system of deposit insurance and implied government guarantees, and a series of legal battles and corruption scandals fought out in the courts. 
Regulators shifted towards a policy of earlier intervention in failing institutions so that the principal costs are more likely to be borne by shareholders than other stakeholders. There was also a shift towards more risk-sensitive regulatory regimes with respect to both net-worth assessments and the payments made by individual institutions to deposit insurance funds, while deposit insurance reform made it less likely that taxpayers would shoulder so great a burden in any future crisis.14 
At a wider level, the S&L crisis taught politicians, regulators and bankers how misleading rules-driven regulatory and accounting numbers can be in relation to risky bank activities. At different stages of the crisis, and at many different levels from bank executives through to regulators and politicians, a formalistic reporting of the financial condition of S&Ls was deliberately selected by interested parties to cover up the true economic extent of the unfolding disaster. It was a risk-reporting failure on a grand scale that greatly worsened the long-term economic consequences for the ultimate stakeholder: the US taxpayer. 

 
Lessons learned
•	Regulatory capital and accounting numbers are not a good guide to risk-adjusted profitability in the banking industry 
•	It's easy for wounded institutions to move out of the frying pan and into the fire by taking cash up front for assuming long-term or unreported risks 
•	Poorly controlled lending institutions can easily 'recycle' poor loans (by granting more credit) and fabricate fee income (by churning transactions) to disguise their true level of risk and return 
•	Rapid growth into new lines of business signals the need for tighter risk management and financial controls 


 
Timeline of Events
1932: Federal Home Loan Bank Act establishes Federal Home Loan Bank Board (FHLBB) 1933: Home Owner's Act promotes home ownership via mortgage loans offered by savings & loans associations regulated by the FHLBB 
1934: National Housing Act sets up Federal Savings and Loan Insurance Corporation (FSLIC) to insure deposits at S&L institutions 
1960s: Congress applies Regulation Q to the S&L industry to put a ceiling on the interest rate that S&Ls can pay to depositors. 
1970s: Congress deregulates interest rates opening up potential asset/liability and interest rate risks for S&Ls, but politicians fail to act on various studies and commissions recommending a mix of consolidated supervision and liberalised regulation of the sector. 
1979-1982: Sharply raised interest rates lead to an asset/liability crisis at many S&Ls that is at its worst in 1980 to 1982. 
November 1980: Following the March enactment of the Depository Institutions Deregulation and Monetary Control Act of 1980 (DIDMCA), which allowed the Bank Board to ease the previous statutory 5% of net worth requirement to anywhere between 3% and 6%, FHLBB eases 'net worth' rules to only 4% of insured accounts. DIDMCA also raises the bar on federally insured deposits from $40,000 to $100,000 and allows some S&Ls to put money into property development and other risky activities. 
1981: Changes in federal tax regulations under the Economic Recovery Tax Act of 1981 help spark the beginnings of the real-estate boom of the early to mid 1980s. 
September 1981: FHLBB introduces various rules and accounting changes to make the financial condition of S&Ls look better, including allowing the deferral of losses from the sale of impaired assets over a ten-year period, and the issuance of capital 'certificates' that artificially boost apparent capitalisation. 
January 1982: Net worth rules eased again to only 3% of insured accounts. 
July 1982: FHLBB allows S&Ls to amortise 'supervisory goodwill' over a period of up to 40 years, up from an original 10-year restriction. Garn-St Germain Depository Institutions Act of 1982 allows easing of capital rules, and greatly eases restrictions on the proportion of a property's value that S&Ls can loan to a property developer. Deposit interest rate ceilings (Regulation Q) phased out for S&Ls, enabling them to compete for wholesale funds by offering high rates of interest. 
Late 1982: FHLBB starts to count equity capital as part of an S&L's reserves 
January 1983: Restrictions lifted on state-chartered S&Ls in California with regard to investments in property and service companies, as state legislators compete with federal legislators to ease restrictions on S&Ls. 
1983: Interest rates fall, temporarily making some - though not all - of the S&L industry solvent on an economic basis. But the opportunity for rational closure of institutions and reform of healthy institutions is missed. 
Late 1984 and after: Regulators begin to tighten up regulations to try to prevent weaker institutions making rash loans and investments following a number of attention-grabbing S&L collapses. 
1984-89: S&Ls pay above-market rates to attract deposits, particularly in hot spots such as the Texas S&L industry. It's clear that the industry is in deep trouble but its regulators lack resources and political backing to close insolvent institutions quickly enough. 
1986: FSLIC, itself clearly insolvent by year-end 1986, resolves 54 thrifts with total assets of around $16 billion. But far more thrifts are insolvent according to their book values, while many others hover on the brink of book insolvency. The economic reality is even worse, with perhaps half the industry now under the water. 
1986-1992: During the later 1980s, the real-estate bubble bursts in regions around the US, partly prompted by the passing of the Tax Reform Act in 1986, which removes federal tax incentives to invest in commercial real estate. 
1987: The passing of the Competitive Equality Banking Act, and the setting up of a Financing Corporation (FICO) to fund FSLIC resolution of failing thrifts by means of issuing bonds, channel some limited resources to the programme of S&L closure, but the emphasis remains on keeping wounded S&Ls afloat. 
1988: Regulators resolve 185 thrifts with total assets of $96 billion, but it's not enough to stabilise the industry and many resolutions continue to be by means of regulator-agreed acquisition: sharing rather than ending the economic woe. 
1989-1990: In terms of public expense, the S&L crisis is at its height. RTC resolves 318 thrifts with total assets of $135 billion in 1989 and 213 thrifts with total assets of $130 billion in 1990. 
1990-92: RTC continues to resolve large numbers of thrifts, but the annual figure for 1992 falls to 59 institutions with $44 billion assets. 
1993-95: The number of thrifts requiring RTC intervention falls away sharply to only 13 over this three-year period as industry fundamentals begin to improve. The crisis is over, but legal wrangling over the restructuring process will continue into the next millennium. 

