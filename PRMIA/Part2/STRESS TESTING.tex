PRMIA 2 - STRESS TESTING

Stress testing is a simple form of scenario analysis. Rather than consider the evolution of risk factors over several time steps, stress testing considers changes in risk factors over a single time step. That horizon is usually a single trading day, but stress testing can be considered over longer horizons—a week, two weeks, a quarter or even a year. Usually, stress testing is used to assess market risk, and that is the application this article focuses on. However, any scenario analysis that employs a single time step may be referred to as a stress test.

Used for market risk, a single scenario consists of projected values for applicable risk factors at the end of the horizon. Based on these values, a portfolio is marked-to-market. The result is compared with the portfolio's current market value, and the portfolio loss is calculated as the difference between the two.


 


Scenarios can be constructed in an ad hoc manner. If management is concerned about the effect of an inverted yield curve or a breakdown in a specific correlation, a scenario can be constructed specifically to assess that eventuality. Stress testing can also be systematized. A firm may specify certain fixed scenarios (defined in terms of percent changes in applicable risk factors) and then perform periodic stress testing with those scenarios. In this manner, a firm might present stress test results in its daily risk report. Such stress scenarios may be hypothetical, perhaps reflecting contingencies that are a recurring concern of management. They can also be historically based. With that approach, stress scenarios may reflect the percentage changes in risk factors experienced during selected historical periods of market turmoil—stock market crashes, currency devaluations, etc.
Stress testing has much in common with value-at-risk (VaR). Both assess market risk. Both consider the change in market risk over a fixed horizon due to changes in specific risk factors. Indeed, if stress testing is conducted with randomly generated scenarios, the analysis would not be called stress testing. It would be called a Monte Carlo VaR measure.
There is some misunderstanding about the purpose of stress testing. This can be traced to the early literature on VaR from the mid-1990s. Like any tool, VaR has limitations, and those limitations were significant with many of the crude VaR implementations of the day. In light of those limitations, it became customary to recommend stress testing as a supplement to VaR. The phrase "VaR should always be supplemented with stress testing" is familiar to practitioners who worked in financial risk management during that period. Actually, the advice was dubious. No one ever identified how stress testing addressed the limitations of VaR measures of the day. For the most part, it didn't.
There was a perception that stress testing allowed for the analysis of extreme events that VaR didn't address. For example, if a firms was using one-day 90% USD VaR, results would reflect losses that might be experienced one day out of ten. What about losses that might be experienced one day out of 100—or one day out of 1000? On the surface, stress testing, with its ability to assess arbitrarily extreme events, seemed well suited to answer such questions. It was not. Although, stress testing can be used to assess losses under any scenario, it associates no probabilities with those scenarios. If stress testing indicates that a firm will lose a billion dollars under one extreme scenario, it is difficult to make sense out of the result. Is that a scenario that will occur once every thousand days or once every thousand years? Given an extreme enough scenario, it is possible to predict ruin for any portfolio.
Used as a supplement to VaR, stress testing is primarily useful for offering an intuitive sense of what sorts of scenarios are causing the VaR to be what it is. In this way, it can be a nice supplement for VaR.
 
 

The one significant shortcoming of VaR that stress testing does address is sudden changes in historical correlations. If two currencies have been pegged to one another, they will exhibit a high historical correlation. A VaR analysis based on that historical correlation will not address the risk that one of the currencies may be devalued relative to the other. If this is a scenario that concerns management, a simple stress test will offer more insights than would, say, a VaR analysis performed with a modified correlation assumption.
In summary, stress testing can be a nice supplement for VaR analyses, and many firms use it for that purpose. For assessing the risk of a breakdown in historical correlations, stress testing can be valuable. Other than that, as a tool for addressing vaguely defined limitations of a VaR measure, stress testing is largely a placebo.
