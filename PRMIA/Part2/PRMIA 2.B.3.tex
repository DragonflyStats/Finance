



\subsection{The Moments of a Distribution}
Moments capture qualities of the distribution of the data about the origin (denoted a, usually either the mean or zero)

the $k-$th moment about an origin is given by

\[ \frac{\sum^{n}_{i=1} (x_i - a)^k}{n} \]


\begin{itemize}
\item The first Moment about zero is the mean (a=0,k=1)

\item If a is the arithmetic mean and k=2, we have the second moment about the mean, which is the variance

\item If a is the arithmetic mean and k=3, we have the third moment about the mean, which is 
the measure of skewness.

\item If a is the arithmetic mean and k=4, we have the fourth moment about the mean, which is 
the measure of ``peakedness" of the distribution.

\end{itemize}

%------------------------------%
