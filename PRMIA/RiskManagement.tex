
\documentclass[12pt, a4paper]{report}
\usepackage{natbib}
\usepackage{vmargin}
\usepackage{graphicx}
\usepackage{epsfig}
\usepackage{subfigure}
%\usepackage{amscd}
\usepackage{amssymb}
\usepackage{subfigure}
\usepackage{amsbsy}
\usepackage{amsthm, amsmath}
%\usepackage[dvips]{graphicx}
\bibliographystyle{chicago}
\renewcommand{\baselinestretch}{1.1}

% left top textwidth textheight headheight % headsep footheight footskip
\setmargins{2.0cm}{2cm}{15.5 cm}{23.5cm}{0.5cm}{0cm}{1cm}{1cm}

\pagenumbering{arabic}


\begin{document}
\author{Kevin O'Brien}
\title{Risk Management}
\date{\today}
\maketitle

\tableofcontents \setcounter{tocdepth}{2}


%-----------------------------------------------------------------------------------------%
\chapter{Operational Risk Management}

\section{Basel II}
Basel II is the second of the Basel Accords, which are
recommendations on banking laws and regulations issued by the
Basel Committee on Banking Supervision. The purpose of Basel II,
which was initially published in June 2004, is to create an
international standard that banking regulators can use when
creating regulations about how much capital banks need to put
aside to guard against the types of financial and operational
risks banks face. Advocates of Basel II believe that such an
international standard can help protect the international
financial system from the types of problems that might arise
should a major bank or a series of banks collapse. \\ In practice,
Basel II attempts to accomplish this by setting up rigorous risk
and capital management requirements designed to ensure that a bank
holds capital reserves appropriate to the risk the bank exposes
itself to through its lending and investment practices. Generally
speaking, these rules mean that the greater risk to which the bank
is exposed, the greater the amount of capital the bank needs to
hold to safeguard its solvency and overall economic stability.


\section{Basic Indicator Approach}
The basic approach or basic indicator approach is a set of
operational risk measurement techniques proposed under Basel II
capital adequacy rules for banking institutions. Basel II requires
all banking institutions to set aside capital for operational
risk. Basic indicator approach is much simpler compared to the
alternative approaches (i.e. standardized approach (operational
risk) and advanced measurement approach) and this has been
recommended for banks without significant international
operations. Based on the original Basel Accord, banks using the
basic indicator approach must hold capital for operational risk
equal to the average over the previous three years of a fixed
percentage of positive annual gross income. Figures for any year
in which annual gross income is negative or zero should be
excluded from both the numerator and denominator when calculating
the average. The fixed percentage ‘alpha’ is typically 15 percent
of annual gross income.

\section{Standardized Approach}
In the context of operational risk, the standardized approach or
standardized approach is a set of operational risk measurement
techniques proposed under Basel II capital adequacy rules for
banking institutions. \\Basel II requires all banking institutions
to set aside capital for operational risk. Standardized approach
falls between basic indicator approach and advanced measurement
approach in terms of degree of complexity. \\Based on the original
Basel Accord, under the Standardized Approach, banks’ activities
are divided into eight business lines: corporate finance, trading
\& sales, retail banking, commercial banking, payment \&
settlement, agency services, asset management, and retail
brokerage. Within each business line, gross income is a broad
indicator that serves as a proxy for the scale of business
operations and thus the likely scale of operational risk exposure
within each of these business lines. The capital charge for each
business line is calculated by multiplying gross income by a
factor (denoted beta) assigned to that business line. Beta serves
as a proxy for the industry-wide relationship between the
operational risk loss experience for a given business line and the
aggregate level of gross income for that business line.


\section{Advanced Measure Approach}
The advanced measurement approach (AMA) is a set of operational
risk measurement techniques proposed under Basel II capital
adequacy rules for banking institutions. Under this approach the
banks are allowed to develop their own empirical model to quantify
required capital for operational risk. Banks can use this approach
only subject to approval from their local regulators. Also,
according to section 664 of original Basel Accord, in order to
qualify for use of the AMA a bank must satisfy its supervisor
that, at a minimum:
\begin{itemize} \item Its board of directors
and senior management, as appropriate, are actively involved in
the oversight of the operational risk management framework; \item
It has an operational risk management system that is conceptually
sound and is implemented with integrity; \item and It has
sufficient resources in the use of the approach in the major
business lines as well as the control and audit areas.
\end{itemize}


\section{Solvency II}
Solvency II is the updated set of regulatory requirements for
insurance firms that operate in the European Union. It is
scheduled to come into effect on 31 Dec 2012. The rationale for
European Union insurance legislation is to facilitate the
development of a Single Market in insurance services in Europe,
whilst at the same time securing an adequate level of consumer
protection. \\The third-generation Insurance Directives
established an "EU passport" (single licence) for insurers based
on the concept of minimum harmonization and mutual recognition.
Many Member States have concluded that the current EU minimum
requirements are not sufficient and have implemented their own
reforms, thus leading to a situation where there is a patchwork of
regulatory requirements across the EU. This hampers the
functioning of the Single Market.\\ Solvency II will be based on
economic principles for the measurement of assets and liabilities.
It will also be a risk-based system as risk will be measured on
consistent principles and capital requirements will depend
directly on this. While the Solvency I Directive was aimed at
revising and updating the current EU Solvency regime, Solvency II
has a much wider scope.

\section{RiskMetrics}
RiskMetrics describes a methodology based on J P Morgan's approach
to quantifying market risk in portfolios of fixed-income
instruments, equities, foreign exchange, commodities, and their
derivatives in the financial markets of 22 countries.
%--------------------------------------------------------------------
\newpage
\chapter{Key Methodologies}

\section{Exposure at Default}
In general EAD can be seen as an estimation of the extent to which
a bank may be exposed to a counterparty in the event of, and at
the time of, that counterparty’s default. It is a measure of
potential exposure (in currency) as calculated by a Basel Credit
Risk Model for the period of 1 year or until maturity whichever is
sooner. Based on Basel Guidelines, Exposure at Default (EAD) for
loan commitments measures the amount of the facility that is
likely to be drawn if a default occurs[1]. Under Basel II a bank
needs to provide an estimate of the exposure amount for each
transaction, commonly referred to as Exposure at Default (EAD), in
banks’ internal systems. All these loss estimates should seek to
fully capture the risks of an underlying exposure. EAD is mainly
used in banking businesses.

\section{Loss Given Default}
Loss Given Default or LGD is a common parameter in Risk Models and
also a parameter used in the calculation of Economic Capital or
Regulatory Capital under Basel II for a banking institution. This
is an attribute of any exposure on bank's client. Exposure is the
amount that one may lose in an investment.
\\
LGD is the credit loss incurred if an obligor defaults. \\ Loss
Given Default is facility-specific because such losses are
generally understood to be influenced by key transaction
characteristics such as the presence of collateral and the degree
of subordination.

\subsection{LGD Calculation}
Theoretically, LGD is calculated in different ways, but the most
popular is 'Gross' LGD, where total losses are divided by EAD.
Another method is to divide Losses by the unsecured portion of a
credit line (where security covers a portion of EAD - Exposure at
Default). This is known as `Blanco' LGD. If collateral value is
zero in the last case then Blanco LGD is equivalent to Gross LGD.
Different types of statistical methods can be used to do this.
\\Gross LGD is most popular amongst academics because of its
simplicity and because academics only have access to bond market
data, where collateral values often are unknown, uncalculated or
irrelevant. \\Blanco LGD is popular amongst some practitioners
(banks) because banks often have many secured facilities, and
banks would like to decompose their losses between losses on
unsecured portions and losses on secured portions due to
depreciation of collateral quality. The latter calculation is also
a subtle requirement of Basel II, but most banks are not
sophisticated enough at this time to make those types of
calculations.

%--------------------------------------------------------------------
\newpage
\chapter{Quantitative Finance}

\section{The Greeks}
\begin{itemize} \item Delta, gamma, theta, vega and rho \item
Higher-order Greeks \item How traders use the Greeks \end{itemize}

\section{Monte Carlo simulation}
\begin{itemize} \item Justification for pricing by Monte Carlo simulation \item Grids and
discretization of derivatives \item The explicit finite-difference
method \end{itemize}

\section{Yield Curves}
\begin{itemize}
\item Heath, Jarrow and Morton model \item Evolution of the entire
yield curve \item Risk neutrality \end{itemize}

%--------------------------------------------------------------------
\newpage
\chapter{Case Studies}

\section{Taisei}


 In November 2001, following the September, 11th 2001
terrorist attack on the World Trade Center, Taisei Fire and Marine
Insurance Co (TFMI) collapsed, due to catastrophic insurance
claims of \$2.5 billion. TFMI, together with two other Japanese
companies, had a management agreement with Fortress Re, which
pooled the funds of the companies to share the risks of reinsuring
aviation portfolios. All four planes that crashed on the World
Trade Center and other sites during the 9/11 attack were reinsured
in the Fortress Re pool. The participated companies' lack of
skills in management of Fortress Re and their limited
understanding of liabilities in the pool were revealed after the
event. Apparently, TFMI had completely relied on Fortress Re's
management decisions. Even though it was true that the unforeseen
nature of terrorist attack was a trigger for TFMI's bankruptcy,
this event showed that delegating the entire authority of managing
the pool to the Fortress Re management added considerable risks to
the companyfs portfolio, and did not reduce any risks. TFMI's
collapse was an example of how a company overlooked the potential
risk of greinsuranceh, and transferred risk only on the
accounting book, not in the real world. \subsection{Background}
For the fiscal year ending March, 2001, TMFI was the 15th largest
Japanese non-life insurer, with \$740 million in premium.

TMFI's solvency margin (assets over liabilities) was $815\%$, well
in excess of the $200\%$ demanded by the Japanese authorities as
the required level for financial strength. TMFI had large volumes
of domestic Japanese property and casualty business. It also had a
large volume of inward reinsurance premiums, most of which came
through a US reinsurance pool managed by Fortress Re, a North
Carolina-based reinsurance managing agency .

TMFI, together with Nissan Fire \& Marine and Chiyoda Fire \&
Marine Insurance (since merged with Dai-Tokyo Fire \& Marine
Insurance as Aioi Insurance) were the three members of the
Fortress Re pool. Fortress Re acted as the reinsurance pool
manager responsible for securing the inward reinsurance business.
The Fortress Re pool included aviation, marine and many other
reinsurance products but by 2001 the aviation component had grown
to nearly $70\%$.

For many years, Fortress Re had accepted inward reinsurance and
used traditional reinsurance products to offset significant
portions to other reinsurers. However, it was increasingly relying
on finite reinsurance, and from early 2000, relied solely on this
form of reinsurance for its risk limitation.

It also transpired that $25\%$ of the pool's business had been
ceded to Carolina Re, a Bermuda-based reinsurer, which was owned
by the principals and close family of Fortress executives.
Carolina Re subsequently became insolvent with $\$350$ million in
liabilities and was liquidated by a Bermuda court in late 2001.
\subsection{How did it happen - the unforeseen nature of claims} The
Fortress Re pool recorded substantial profits from this
arrangement. However liabilities were building up prior to 9/11
with a series of aviation claims: \begin{itemize} \item TWA off
Long Island, New York (1996)\item Swissair off the coast of Nova
Scotia (1998)\item EgyptAir of the East Coast of the US (1999),
\item Alaska Airlines off the West Coast of the US (2000) \item
Concorde in Paris (2000). \end{itemize}Whilst the prime reason for
the TFMI's bankruptcy was lack of sufficient reinsurance
protection against a single major event, the 9/11 terrorist
attack, it was also true that TFMI, as well as the other two
companies, lacked underwriting controls and relied almost entirely
on Fortress Re's management team for information on risks in the
portfolio. The three companies in the pool gave Fortress Re wide
powers to conduct business on their behalf and to arrange
reinsurance protection. TFMI had only limited understanding of the
pool's liabilities, complicated further by the impact of the use
of finite reinsurance. \noindent The finite reinsurance model
allowed Fortress Re to claim reinsurance claims payments from the
finite reinsurers and it paid premiums to cover these deals over a
5- year period. As the risks were spread over time, the future
premiums were not accounted for as current liabilities on the
books of the pool members, giving a false impression of
profitability.

\noindent A further issue which disguised the actual risks faced
by TFMI was the accounting risk transfer procedures. Under
Generally Accepted Accounting Principles, if a policy did not
transfer risk, it was a financial agreement in which all premiums
are treated as deposits. If reinsurance agreements were found not
to involve enough risk transfer, the company should have to make
an adjustment to take all the losses into the income statement and
consider the effect on its solvency margin. However, this
procedure made it hard for TFMI to determine whether risk had
actually been transferred and whether it had sufficient
catastrophe cover for the future. In summary, TFMIfs
over-reliance on the pool managerfs judgments and limited
understanding of its poolfs liabilities associated with
accounting procedures for risk transfer meant TFMI had inadequate
reinsurance protection, therefore, TFMI was not protected from
large losses. Consequently, TFMI's solvency margin of $850\%$ was
not enough to cover catastrophic reinsurance claims in a worst
case scenario, as actually happened on September 11th.


%--------------------------------------------------------------------------
\subsection{Summary}
\begin{enumerate}
\item limited understanding of the pool's liabilities
\end{enumerate}


\newpage

