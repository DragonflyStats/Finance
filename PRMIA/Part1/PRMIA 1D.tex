\documentclass[]{article}

\usepackage{graphicx}
\usepackage{framed}
\usepackage{amsmath}
\usepackage{amssymb}

%-------------------------------------------------------------------------%
\begin{document}

\section{PRMIA Section 1D}

CHAPTER 1 provides a general overview of risk and risk aversion, introduces the utility function and mean�
variance criteria. Various risk-adjusted performance measures are described. A summary of several widely used
utility functions is presented in the appendix.


\subsection{Risk and Risk Aversion - Learning Outcome Statement}
The candidate should be able to:
\begin{itemize}
\item Explain the concepts of Utility and Utility Maximization
\item Explain theways of the determination of utility function
\item Explain the concept of Risk Aversion
\item Discuss the Mean-Variance Criterion
\item Define the Sharpe, Treynor and Information Ratios
\item Define Jenson�s Alpha, RAROC, RoVaR and RAPM
\item Define the Sortino, Omega Index and Kappa Ratios
\end{itemize}
%-----------------------------------------------------------------%
\section{The CAPM and 
Multifactor Models }
CHAPTER 4 provides a rigorous description of the CAPM model, including betas, systematic risk, alphas and
performance measures. Arbitrage pricing theory and multifactor models are also introduced in this chapter.

\subsection{Learning Outcome Statement}
The candidate should be able to:
\begin{itemize}
\item Describe the Capital Asset Pricing Model(CAPM)
\item Describe Beta as a Measure of Relative Risk
\item List the assumptions of the CAPM
\item Define risk premium
\item Derive the Security Market Line
\item Define and Calculate the Sharpe Ratio and Jensen�s Alpha
\item Describe the Single Index Model
\item Describe systematic and specific risk
\item Describe the Arbitrage Pricing Theory (APT)
\end{itemize}
%-------------------------------------------------------------------------%
\newpage
\section{The Term Structure of Interest Rates}
\begin{itemize}
\item Describe yield to maturity as an internalrate ofreturn
\item Define spot curve, spotrate and term structure
\item Define and describe the yield curve
\item Demonstrate the process of bootstrapping
\item Define no-arbitrage pricing
\item Calculate implied forward rates
\item Describe normal, flat and inverted yield curves
\item Describe the pure expectations theory
\item Describe the liquidity preference theory
\item Describe the preferred habitat theory
\item Describe the market segmentation theory
\item Compare and contrast the Ho-Lee, Hull-White and Black-Derman-Toy models
\item Compare and contrast single-factor and multi-factor models
\item Describe mean reversion
\item Calculate the value of non-callable bonds using term structure models
\item Describe the impact of an embedded call on the value of a bond using term
structure models
\item Calculate effective duration and convexitywithin a term structure model
\item Define Option Adjusted Spread
\item Discuss the implications of choosing one term structure model
overthe others
\end{itemize}
\newpage
%----------------------------------------------------------------------%
\subsection{Option Pricing}
CHAPTER 8 introduces the principles of option pricing. It startswith definitions of basic put and call options,
put�call parity, binomial models,risk-neutral methods and simple delta hedging. Then the Black�Scholes�Merton formula is introduced. Finally, implied volatility and smile effects are briefly described.

\begin{itemize}
\item Discuss the factor influencing option price
\item Describe put-call parity
\item Discuss the basic principles of the binomial option model
\item Define and discuss delta-hedging
\item Explain risk-neutral valuation
\item Calculate an option price using a one-step binomial model
\item Define the symbols and letters of inputs into the binomial model
\item Describe the basic principles of the Black-Scholes-Merton model
\item State the Black-Scholes-Merton formula for pricing a call option
\item Calculate an option price using Black-Scholes-Merton model
\item Identify and discuss the graphic representations of a put and a call
\item Define delta, gamma, vega, theta and rho
\item Define and discuss implied volatility
\item Define a volatility smile
\item Define intrinsic value and time value
\end{itemize}


\begin{itemize}
\item Compare and Contrast forward and futures contracts
\item Discuss some uses of stock index futures
\item Define index point and value of an index point
\item Describe index arbitrage and program trading
\item Calculate a minimum variance hedge ratio for a portfolio of stocks,
using futures, given beta
\item Describe some risks in index hedging
\item Discuss �tailing the hedge�
\item Compare and contrast currency forwards and futures contracts
\item Define covered interest parity
\item Calculate a forward exchange rate
\item Calculate a hedge ratio using foreign exchange futures
\item Discuss the relative basis riskswith commodity futures
\item Define forward rate agreement(FRA)
\item Discuss FRAs, their nomenclature, uses and settlement
\item Calculate T-bill and Eurodollarfutures prices
\item Construct a hedge using Eurodollar or T-bill futures
\item Define the tick value of a Eurodollar or T-bill futures contract
\item Define cheapest-to-deliver and conversion factor
\item Compare and contrast T-Bond and Gilt futures contracts
\item Define the tick value of a T-Bond and Gilt futures contract
\item Construct a hedge using T-bond futures
\item Compare and contrast stack and strip hedges
\end{itemize}
\end{document}
