
PRMIA 1.A.2 Portfolio Mathematics

%==============================================================================================================%
%==============================================================================================================%

Portfolio Mathematics - Learning Outcome Statement

The candidate should be able to:
\begin{itemize}
\item 	 Calculate the return, mean return, variance and standard deviation of a single asset
	
	
\item 	 Calculate the return, mean return, variance and standard deviation of a portfolio
	
	
\item 	 Calculate the correlation between two assets
	
	
\item 	 Identify a dominated portfolio
	
	
\item 	 Discuss the efficient frontier
	
	
\item 	 Calculate the minimum variance hedge ratio
	
	
\item 	 Describe how diversification reduces risk
	
	
\item 	 Describe the impact of serial correlation on the standard deviation of returns
	
	
\item 	 Calculate Value at Risk in a portfolio
	
	
\item 	 Calculate the probability that one portfolio will outperform another portfolio
	
	
\item 	 Calculate the probability of attaining a return goal
\end{itemize}
%==============================================================================================================%
%==============================================================================================================%


1) Mean and Variance of past returns

2) Mean and Variance of future returns

3) Mean-Variance Tradeoffs

4) Multiple Assets

5) A Hedging Example

6) Serial Correlation

7) Normally Distributed Returns 


%==============================================================================================================%
%==============================================================================================================%
Section 1: Mean and Variance of past returns


1)

2) 

3) Portfolio means, variance and Standard Deviation

4) Correlation

5) Correlation and Portfolio Variance


%==============================================================================================================%
%==============================================================================================================%
Section 2: Mean and Variance of future Returns


1) Single Asset

2) Covariance and Correlation

3) Mean and Variance of a linear combination

4) Example: Portfolio Returns

5) Example: Portfolio Profit

6) Example: Long and Short Positions

7) Example: Correlation


PRMIA 1A22 Mean and Variance of Future Returns

1) Single Asset
2) Covariance and Correlation
3) Mean and Variance of Linear Combinations
4) Example: Portfolio Returns
5) Example: Portfolio Profit
6) Example: Long and Short Position
7) Example: Correlation

1.A.2.2 Standard Definitions of mean and variance of a random variable for 
discrete and continuous cases.

1.A.2.2.2
Linear Combination

\[aX+bY\]

\[E(aX+bY) = aE(x) + bE(y) = a\mu_X + b\mu_Y\]



\[var(aX+bY) =  a^2 var(X) + b^2 var(Y)  +2ab cov(X,Y)\]


\[var(aX+bY) =  a^2 \sigma^2_X + b^2 \sigma^2_Y  +2ab \sigma_{XY}\]

Portfolio with weight $w$ on X and $1-w$ on Y is

\[\Pi = wX+(1-w)Y)\]


\[ E(\Pi) = wE(X) + (1-w)E(Y) \]

\[ \sigma^2_{\pi} =  w^2 \sigma^2_X + (1-w)^2\sigma^2_Y + 2w(1-w)\sigma_{XY} \]

%==============================================================================================================%
%==============================================================================================================%
Section 3: Mean-Variance Tradeoffs 


1) Achievable Expected Returns

2) Achievable Variance and Standard Deviation

3)

4) Efficient Frontier

5) Utility Maximization

6) Varying the Correlation Parameter



%==============================================================================================================%
%==============================================================================================================%
\subsection*{Section 4: Mutliple Assets}


Portfolio mean and variance

Vector Matrix Notation

Efficient Frontier


%==============================================================================================================%
%==============================================================================================================%
\subsection*{Section 5: A Hedging Example}

Gallon-on-Gallon Hedge

Minimum Variance Hedge

4) Effectiveness of the Optimal Hedge

5) Connection with regression


%==============================================================================================================%
%==============================================================================================================%

\subsection*{Section 6: Serial Correlation}


The correlation of a variable with itself over successive time intervals. 

Technical analysts use serial correlation to determine how well the past price of a security predicts the future price.


PRMIA 1.A.2.6 Serial Correlation

Relationship of variabilities in returns in a single asset over different time periods.

How is the standard deviation of annual returns of a singple asset related to the standard deviation of monthly returns for the same asset.

Notation: $ X_1 , X_2 , \ldots , X_n $ denotes the returns of a single asset over consecutive time periods.

%==================================================================%

$x_i$ continuously compounded returns.

Return $X$ over the entire period is the sum of the returns over the individual periods.

\[ X = X_1 + X_2 + \ldots X_n \]

\[ X_1 + X_2 + \ldots X_n = \mathrm{ln}(s_1/s_0) + \mathrm{ln}(s_2/s_1) + \mathrm{ln}(s_3/s_2) + \ldots + \mathrm{ln}(s_n/s_{n-1})\]

\[ X_1 + X_2 + \ldots X_n = \mathrm{ln}(s_n/s_0) \]
%==================================================================%
\subsection*{Uncorrelated Returns}
\[ \rho(X_i,X_j) = 0 \mbox{ for } i \neq j \]

\[ \mathrm{VAR}(X) = \mathrm{VAR}(X_1) + \mathrm{VAR}(X_2) + \ldots + \mathrm{VAR}(X_n) \]


The return variance is the sum of the variances over the individual periods.


If we further assume that there is a consistent level of variance.

\[ \mathrm{VAR}(X) = n \sigma^2 \]
\[ \mathrm{SD}(X) = \sqrt{n} \sigma \]

%==============================================================================================================%
%==============================================================================================================%


%--------------------------------------------------------------------------------%
Section 7: Normally Distributed Returns

1) The Distribution of Portfolio Returns

2) Value at Risk

3) Probability of Reaching a target

4) Probability of beating a Benchmark


