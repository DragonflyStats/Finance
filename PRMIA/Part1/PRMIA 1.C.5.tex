PRMIA 1.C.5 The Stock Market

PRMIA 1.C.5 The Stock Market
Learning Outcome Statement
Section 1: Introduction
Section 2: The Characteristics of Common Stocks
Section 3: Stock Markets and Their Participants
Section 4: The Primary Markets - IPOs and Private placement
Section 5: The Secondary Markets
Section 6: Trading Costs
Section 7: Buying on Margin
Section 8: Short sales and Stock Borrowing costs
Section 10: Summary
%----------------------------------------------------------%
Learning Outcome Statement
The candidate should be able to:
  \begin{itemize}	
\item	Describe the common characteristics of a stock
\item	  Define IPO, primary issue, and secondary offering
\item	  Discuss shareholder rights
\item	  Define dividend and ex-dividend trading
\item	  Compare and contrast ordinary and preference shares
\item	  Define market capitalization
\item	  Discuss stock indices
\item	  Define the dividend discount and Gordon growth models of stock valuation
\item	  Discuss the types of stock market participants
\item	  Define listing and float
\item	  Compare and contrast matched market and market maker
\item	  Define T+1 and T+3 settlement
\item	  Define private placement and seasoned new issue
\item	  Describe the process of an IPO
\item	  Describe the process of a private placement
\item	  Describe the role of exchanges
\item	  Describe the role of the OTC market
\item	  Define the bid/offer spread
\item	  Discuss margin trading
\item	  Discuss short-selling and borrowing stocks
\item	  Compare and contrast single-stock and index options
\end{itemize}	


Section 1: Introduction
%---------------------------------------------------------------%
\subsection*{Section 2: The Characteristics of Common Stocks}
1) Share Premium
2) Equity Shareholders Rights and Dividents
3)
4) Equity Price Data
5) Market Capitalisation
6) Stock Market Indices
7) Equity Valuation
%---------------------------------------------------------------%
\subsection*{Section 3: Stock Markets and Their Participants}
1) The main participants - Firms, Investments banks and investors
2) Market Mechanics
%---------------------------------------------------------------%
Section 4: The Primary Markets - IPOs and Private placement
1) Basic Primary Market Process
2) Initial Public Offerings
3) Private Placements
%---------------------------------------------------------------%
\subsection*{Section 5: The Secondary Markets}

1C5 Equity markets
ordinary shares
preference shares
equity price data

market capitalisation
stock market indicws
IC53
3 stock markets and their participants
main participants
firms banks and investors

market mechanics
primary market 
initial public offerings
basic primary market proces
%---------------------------------------------------------------%
\subsection*{Section 6: Trading Costs}
s

trading costs


%---------------------------------------------------------------%
\subsection*{Section 7: Buying on Margin}
The purchase of an asset by paying the margin and borrowing the balance from a bank or broker. Buying on margin refers to the initial or down payment made to the broker for the asset being purchased. The collateral for the funds being borrowed is the marginable securities in the investor's account. Before buying on margin, an investor needs to open a margin account with the broker. In the U.S., the amount of margin that must be paid for a security is regulated by the Federal Reserve Board.



leverage
percentage margin and maintainance margin
why trade on margin


%---------------------------------------------------------------%
\subsection*{Section 8: Short sales and Stock Borrowing costs}
%---------------------------------------------------------------%
\subsection*{Section 9: Exchange Traded Derivatives on Stocks} 
1) Single stock and index options
2) Expiration Dates
3) Strike Prices
4) Flex Options
5) Dividends and Corporate Actiond
6) Position limits
7)Trading

\subsection*{section 9}
exchange traded derivatives on stocks
flex options
position limits
dividends and corporate actions
trading

\subsubsection*{Flexible Exchange Option - FLEX}


 A non-standard option which can be customized, allowing both the writer and purchaser to define various terms. Flexible Exchange Options allow parties to negotiate the exercise style, strike price, expiration date and other features and benefits. They also give investors the opportunity to trade on a larger scale with expanded or eliminated position limits. 


%---------------------------------------------------------------%
\subsection*{Section 10: Summary}
8short sales and stock borrowing costs
short sales
stock borrowing


