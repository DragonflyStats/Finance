\section{PRMIA 1.A.3 Capital Allocation }

%--------------------------------------------------------------------------------%


Capital Allocation Learning Outcome Statement

The candidate should be able to:

\begin{itemize}
\item Describe efficient portfolios that satisfy the mean-variance criterion
\item Describe tolerances and preferences for Risk vs. Return
\item Show the efficient frontier for two assets
\item Show the efficient frontier for a multi-asset portfolio
\item Define the risk-free asset
\item Derive and describe the Capital Allocation Line
\item Describe the Capital Markets Line
\item Define the market portfolio
\item Describe the separation principle
\end{itemize}
%%------------------------------------------------------------------------------%%
Capital Allocation is the process of how businesses divide their financial resources and other sources of capital to different processes, people and projects. Overall, it is management's goal to optimize capital allocation so that it generates as much wealth as possible for its shareholders.


Capital Allocation is the process behind making a capital allocation decision is complex, as management virtually has an unlimited number of options to consider. 


For example, if a company ends up with a larger than expected windfall at the end of the year, management needs to decide whether to use the extra funds to buy back stock, issue a special dividend, purchase new equipment or increase the research and development budget. In one way or another, each one of these actions will likely benefit the shareholder, but the difficult part is in determining how much money should be allocated to each action in order to yield the most benefit.


Capital allocation line (CAL) is a graph created by investors to measure the risk of risky and risk-free assets. The graph displays to the investors on the return they can make by taking on a certain level of risk. It is also known as a "reward-to-variability ratio"


1)

2)

3)

4)

5) 

6)

7) 

8) The Separation Principle

9) 




--------------------------------------------------------------------------------


\section{The Separation Principal}


The Separation Principle is used to bring out the project cash flows of a particular project. 

 

It is an important part of capital budgeting. Before starting a new project, it is very important to estimate properly the inflow and outflow of cash. There are several methods that are used to bring out the exact figure of the project cash flow and Separation Principle is one of those methods. 


The Separation Principle treats the cash flow in a different way. At first, the project is divided in two parts. The first part deals with the investment side and the later part is related to the financing side. To get proper picture of the project cash flow, the cash flow is separated according to its relation with the investment of financing side. 


There are several unique features of Separation Principle. One of these features is that the cash flow related to the investment side of the project never considers the cost of financing. On the other hand, these charges of financing are considered while the cash flow calculations related to the financing side are done. 

  

These charges of financing are indicated through the charges of capital figure. The calculations of the returns related to the investment side are based on the hurdle rate that is the capital cost. 


Another important feature of separation principle is that the interest rates on the debt securities are excluded at the time of calculating the profits and payable taxes. Now, according to this theory, while bringing out the profit, if the applicable interest is subtracted, the same amount should be added to the profit that remains after paying the applicable tax. On the other hand, if the tax rate is imposed directly on the profit (from which interest and taxes are not adjusted) the results are not going to differ. 




--------------------------------------------------------------------------------


