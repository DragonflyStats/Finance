
PRMIA 1.A.4  The CAPM and Multifactor Models 

The CAPM and Learning Outcome Statement

Multifactor Models The candidate should be able to:
\begin{itemize}
\item Describe the Capital Asset Pricing Model (CAPM)
\item 
 Describe Beta as a Measure of Relative Risk
\item 
 List the assumptions of the CAPM
\item 
 Define risk premium
\item 
 Derive the Security Market Line
\item 
 Define and Calculate the Sharpe Ratio and Jensen’s Alpha
\item 
 Describe the Single Index Model
\item 
 Describe systematic and specific risk
\item 
 Describe the Arbitrage Pricing Theory (APT)
\end{itemize}

%%--------------------------------------------------------------------------------%
1) Introduction

2) The Capital Asset Pricing Model

3) Security Market Line

4) Performance Measurement

5) The Single Index Model

6) Multifactor Models and the APT

7) Summary




--------------------------------------------------------------------------------


\subsection{1.A.4.1}

The CAPM provides an elegant model of the determinants of the equilibrium expected or required return on any individual risky asset.




--------------------------------------------------------------------------------
\subsection{1.A.4.2}


\subsubsection{1.A.4.2.1 Estimating Beta}


(Ri-r)i=i+i(Rm-1)i+i



Beta and Systematic Risk


The CAPM predicts that only the covariance of returns between assets i and the market portfolio influences the average

excess return on asset i.


No additional variables should influence average excess returns on a stock.


All contibutions to the risk of asset i when it is held as part of a well-diversified portfolio , are summed up by its beta.




--------------------------------------------------------------------------------
\subsection{1.A.4.3 Security Market Line}


What is the SML?




--------------------------------------------------------------------------------
1.A.4.4.


1. Sharpe Ratio


Return relative to the risk (volatility) of the portfolio


Sp=risk premiumtotal risk=rp-rfp


Sp Sharpe's Index of portfolio performance of portfolio p.

p standard deviation of returns of portfolio p. 

rf riskless rate of interest




--------------------------------------------------------------------------------
\subsection{1.A.4.5 Single Index Market}


This is a statistical assumption that the return on any security Ri may be adequately represented as a linear

function of a single economic variable (i.e. Inflation or Interest Rates).


Rit=j+iIi+i


Ii is any economic variable that can be found to be correlated with Rit. The SIM hypothesis has no specific theoretical model

that seeks to explain this observed correlation.




--------------------------------------------------------------------------------
\subsection{1.A.4.6 Multifactor Models}



The Arbitrage Pricing Theory (APT)


In contrast to the CAPM, the APT does not assume any of the following


1) A single period horizon


2) that there are no taxes


3) That investors can freely borrow and lend at the risk free rate.


4) That investors select portfolios on the basis of mean and variance of return.


APT assumes that security returns are generated according to what is known as a factor model.


The Basic APT return equation is similar to that of the SML


E(ri) =rz+bi




The APT leads to the regression model


Equilibrium Return



1) Portfolio Returns


Specific Risk can be diversified away by holding a large number of securities.

\subsection{1.A.4.7 Summary}

The CAPm  implies that in equilibrium the expected excess return on any risky asset is 

proportional to the excess return in the market portfolio.


The constant of proportionality is the asset's "beta".


Beta provides a measure on the risk of an individual security when it held as part of a well diversified portfolio.



There are two key features of the APT;

1) Actual returns may depend on a number of market wide variables of "factors", each of which has its own beta coefficient.


2) Equilibrium returns on asset i depend on a weighted average of its factor's betas. These weights \lambda are the same for all assets

 are termed the "price" of each risk factor. 





--------------------------------------------------------------------------------
The CAPM says that the expected return of a security or a portfolio equals the rate on a risk-free security plus a risk premium. If this expected return does not meet or beat the required return, then the investment should not be undertaken. The security market line plots the results of the CAPM for all different risks (betas).


Using the CAPM model and the following assumptions, we can compute the expected return of a stock in this CAPM example: if the risk-free rate is 3%, the beta (risk measure) of the stock is 2 and the expected market return over the period is 10%, the stock is expected to return 17% (3%+2(10%-3%)).


The CAPM is a model that describes the relationship between risk and expected return and that is used in the pricing of risky securities.




 


The general idea behind CAPM is that investors need to be compensated in two ways: time value of money and risk. The time value of money is represented by the risk-free (rf) rate in the formula and compensates the investors for placing money in any investment over a period of time. The other half of the formula represents risk and calculates the amount of compensation the investor needs for taking on additional risk. This is calculated by taking a risk measure (beta) that compares the returns of the asset to the market over a period of time and to the market premium (Rm-rf).




--------------------------------------------------------------------------------


1A44
Performance measures

Jensens alpha is based onthe CAPM and is the intercept of the regression equation.
