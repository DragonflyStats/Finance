
PRMIA 1.C.7 The Structures of Commodities Markets


Commodity markets are markets where raw or primary products are exchanged. 

These raw commodities are traded on regulated commodities exchanges, in which they are bought and sold in standardized contracts.


The structure of the commodities market.


The candidate should be able to:
a.
  List four general types of commodities

b.
  Contrast base, strategic, minor and precious metals

c.
  Contrast grains, oilseeds and fibers

d.
  Define “on the spot” and “settlement of difference”

e.
  Define in store, ex store, Free on Board (FOB), Free alongside Ship (FOS),Cost Insurance & Freight (CIF) and Exchange for Physicals (EFP)

f.
  Discuss the uniqueness of the gold market

g.
  Define contango, backwardation, carrying cost (cost of carry) and lease rate

h.
  Discuss the impact of shortages on commodity prices and the history of short squeezes

i.
  Define short squeeze and demand for immediacy

j.
 Discuss the convenience yield theory

k.
  State the arbitrage equation for commodity pricing

l.
  Discuss the decomposition of risk factors in commodities

m.
  Discuss the importance of non-normality of commodity price distributions






--------------------------------------------------------------------------------


(b) Minor metals is a widely-used term in the metal industry that generally refers to primary metals not traded on the London Metal Exchange (LME). 




--------------------------------------------------------------------------------



PRMIA 1.C.7 The Structures of Commodities Markets

1.C.7.2

1.C.7.3 Spot Forward pricing relationship

1.C.7.5 Risk Management at the Commodity Trading Desk

1.C.7.6. The Distribution of commodity returns


1.C.7.2

Liquidity is defined as the ability ot buy or sell at fair market value with changing transaction costs.

A market is liquid if there are lots of buyers and sellers, and large volumes can be executed with small transaction costs.

 

Forwards generally trade in the OTC markets.

 

1.C.2.7.3 Delivery and Settlement methods

1.C.2.7.4 Commodity Market Liquidity

1.C.2.7.5 The special case of Gold as a reserve asset.

 


--------------------------------------------------------------------------------
1.C.7.3 Spot Forward pricing relationship

 

1. Backwardisation and contango

 

Forward Price = Spot price + carrying cost

 

The goal market is an exceptional commodity and is always in contango.

 

2. Reasons for backwardisation

 

 - convenience yield theory

 - demand for immediacy

 

3. The no-arbritrage condition

 

F-SS= r +C -q

 

1.C.7.4.2 Exchange Limits

 


--------------------------------------------------------------------------------


1.C.7.5 Risk Management at the Commodity Trading Desk

The no-arbitrarge condition

Change in interest rate

Change in yield

Storage costs are constant




--------------------------------------------------------------------------------


1.C.7.6. The Distribution of commodity returns

Evidence of non-normality

What drives commodity prices?


 %PRMIA 1.C.7.3
Backwardisation
Futures market term. It means a downward sloping forward curve (as in inverted yield curve).

It is a situation, and amount, the price of a commodity for future delivery is lower than the spot price, or a far-future delivery  price
lower than a nearer future delivery

The opposite market condition is called \textbf{contango}.

Contango
Futures market term. It means a upard sloping forward curve 

Formally, it is a situation, and amount, the price of a commodity for future delivery is higher than the spot price, or a far-future delivery  price
higher than a nearer future delivery.

A contango is normal for a non-perishable good that has a cost of carry. The contango should not exceed the cost of carry, because
producers and consumers can compare the futures contract price agains the spot price, plus storage, and choose which is the better one.

The opposite market condition is called \textbf{Backwardisation}.

%----------------------------------%
