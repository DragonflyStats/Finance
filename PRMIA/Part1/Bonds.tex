DEFINITION of 'Yield To Maturity (YTM)'

The rate of return anticipated on a bond if held until the end of its lifetime. YTM is considered a long-term bond yield expressed as an annual rate. The YTM calculation takes into account the bond’s current market price, par value, coupon interest rate and time to maturity. It is also assumed that all coupon payments are reinvested at the same rate as the bond’s current yield. YTM is a complex but accurate calculation of a bond’s return that helps investors compare bonds with different maturities and coupons.
%===================================================================%


\subsection*{Maturity}
The period of time for which a financial instrument remains outstanding. 
Maturity refers to a finite time period at the end of which the financial instrument will cease to exist and the principal is repaid with interest. The term is most commonly used in the context of fixed income investments, such as bonds and deposits.

\subsection*{Bullet Bond}
A debt instrument whose entire face value is paid at once on the maturity date. 
Bullet bonds are non-callable. Bullet bonds cannot be redeemed early by an issuer, so they pay a relatively low rate of interest 
because of the issuer's exposure to interest-rate risk. 
Both corporations and governments issue bullet bonds, and bullet bonds come in a variety of maturities, from short- to long-term.
A portfolio made up of bullet bonds is called a bullet portfolio.

%==================================================================%

A bullet bond is considered riskier than an amortizing bond because it gives the issuer a large repayment obligation 
on a single date rather than a series of smaller repayment obligations spread over several dates. 
As a result, issuers who are relatively new to the market or who have less than excellent credit ratings 
may attract more investors with an amortizing bond than with a bullet bond.

\end{document}
