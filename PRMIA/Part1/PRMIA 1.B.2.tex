PRMIA 1.B.2 The Analysis of Bonds
Learning Outcome Statement
The candidate should be able to:
 Define nominal (notional, face, par, maturity) value, maturity, term to maturity, coupon, coupon-rate, zero-coupon and vanilla bond
 Describe a bond as a series of cash flows
 Define index-linked bonds, securitized bonds, amortizing bonds, callable bonds, putable bonds and convertible bonds
 Define discount and premium
 Calculate the clean and dirty price of a bond
 Calculate current yield and yield to maturity
 Describe the relationship between yield and price
 Discuss the “pull to par” of bond prices
 Compare and contrast Macauley Duration and Modified Duration
 Define DVBP, dollar duration and key rate duration
 Calculate the modified duration of a bond
 Describe the shortcomings of Macauley and Modified durations
 Calculate the DVBP of a bond
 Discuss Effective Duration
 Discuss the duration of a floating rate note
 Describe the impact of an embedded call or put on duration
 Define basis point value (BPV)
 Calculate the hedge ratio for a bond using BPV
 Define and discuss convexity
 Describe the impact of an embedded call or put on convexity
 Discuss the various risks associated with a bond

1) Features of Bonds
2) Non Conventional Bonds
3) Pricing a conventional bonds
4)
5) 
6) Duration
7) 
8) 

Section 1.B.2.1 Features of Bonds

1) Types of issuer
2) Term to maturity
3) principal coupon rate
4) Convexity

Section 2 Non conventional Bonds

3) Zero Coupon Bonds
4) Securitised bonds
5) Bonds with embedded options

Section 3 Pricing a conventional bond

1) Bond Cash Flows
2) The Discount rates
3) Conventional Bond pricing
Pricing 


1.B.2.3 Pricing a conventional bond

AL = C \times {N_{xi} - N_{xe} \over \mbox{Day Base}}

AL = -C \times {\mbox{Days to next coupon} \over \mbox{Day Base}}


1.B.2.3.6 Clean and Dirty bond prices (accrued Interest)

Section 1.B.2.4 Market Yield

1) Yield Measurement
2) Current Yield
3) Yield to Maturity



1.B.2.4.1 Yield Measurement

1.B.2.4.2 Current Yield

n=CP100%


1.B.2.4.3 Yield to Maturity


Section 5 Relationship between bond yield and bond prices

Section 6 Duration





1.B.2.6

dpdr1P=-11+rD

MD =D1+r
1.B.2.6.2 Properties of the Macaulay Duration

The following three factors imply higher duration for a  bond
1) The lower the coupon
2) The lower the yield
3) Broadly the longer the maturity

1.B.2.7 Hedging Bond Positions

Basis point values are used to hedging positions
BPV =MD100P100

1.B.2.8 Convexity
Duration can be regarded as a first order emasures of inerest rate risk.
Convexity can be regarded as a second order interest rate risk

