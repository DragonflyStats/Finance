PRMIA 1.C.8 The Energy Markets 

PRMIA 1.C.8 The Energy Markets
1.C.8 Energy Markets
1.C.8.4 OTC Energy derivative markets

1) Introduction
2) Market Overview
3) Energy Futures Markets
4) OTC Energy derivatives Markets
5) Emerging Energy derivatives Market
6) The Future of Energy trading
7) Conclusion

The Energy Markets : Learning Outcome Statement
The candidate should be able to:
  Discuss the size of markets for energy
  Discuss the various energy futures markets
  List the major energy futures contracts
  Describe various options on energy
  Discuss using futures markets to hedge energy risk
  Construct an energy hedge using futures contracts
  Discuss physical delivery in energy markets
  Define basis contracts in OTC energy markets
  Discuss the role of the Singapore Market
  Discuss the role of the European Market
  Discuss the role of the North American OTC energy market
  Discuss the role of NordPool
  Discuss the role that Platts plays in the energy market
  Discuss the Coal market
  Discuss the weather derivatives market
  Discuss the emergence of green trading
  Define Heating Degree Day (HDD) and Cooling Degree Day (CDD)
  Discuss the issues of future energy trading


An energy derivative is a derivative instrument in which the underlying asset is based on energy products including oil, natural gas and electricity, which trade either on an exchange or over-the-counter. Energy derivatives can be options, futures or swap agreements, among others. The value of a derivative will vary based on the changes of the price of the underlying energy product. 

Energy derivatives can be used for both speculation and hedging purposes. Companies, whether they sell or just use energy, can buy or sell energy derivatives to hedge against fluctuations in the movement of underlying energy prices. Speculators can use derivatives to profit from the changes in the underlying price and can amplify those profits through the use of leverage.


(m) Platts is a provider of energy and metals information and a source of benchmark price assessments in the physical energy markets. 

(p) Green trading encompasses all forms of environmental financial trading, including carbon dioxide, sulfur dioxide (acid rain), nitrogen oxide (ozone), renewable energy credits, and energy efficiency (negawatts). All these emerging and established environmental financial markets have one thing in common, which is making the environment cleaner by either reducing emissions, using clean technology or not using energy through the use of financial markets.

(q) Heating degree day (HDD) is a measurement designed to reflect the demand for energy needed to heat a home or business. It is derived from measurements of outside air temperature. The heating requirements for a given structure at a specific location are considered to be directly proportional to the number of HDD at that location. 

A similar measurement, cooling degree day (CDD), reflects the amount of energy used to cool a home or business.

1.C.8.1  Introduction to Energy Markets
Energy trading begain in 1978 on the NYMEX with oil futures.
In the 1980s and 1990s Oil and Gas futures were traded on the NYMEX and IPE.
Asia has the highest oil consumption growth rates.
Electricity has become a fungible commodity.
A fungible commodity is a commodity that is freely interchangeable with another in satisfying an obligation.
 


1.C.8.4 OTC Energy derivative markets
All key terms of OTC derivative deals are negotibale. In effect they are customised transactions.
Common transaction types
1) Forward Contracts
2) Options
3) Swaps
4) Basis contracts




1.C.8.5 Emerging Energy Derivative Markets
 
 

1C835 Physical Delivery
1C836 Market Changes: Backwardiszation and Contango
1C822 The Risks
energy commodities are subject to numerous risks, including credit risk and coutnerparty risk
1C84 OTC derivative markets
1C841: Singapore market
1C842: European Markets
1C843: North American Markets
1C85: Emerging Energy Markets
coal trading
 


1.C.8.7 Conclusion 
