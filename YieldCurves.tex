
\documentclass[12pt, a4paper]{report}
\usepackage{natbib}
\usepackage{vmargin}
\usepackage{graphicx}
\usepackage{epsfig}
\usepackage{subfigure}
%\usepackage{amscd}
\usepackage{amssymb}
\usepackage{subfigure}
\usepackage{amsbsy}
\usepackage{amsthm, amsmath}
%\usepackage[dvips]{graphicx}
\bibliographystyle{chicago}
\renewcommand{\baselinestretch}{1.1}

% left top textwidth textheight headheight % headsep footheight footskip
\setmargins{2.0cm}{2cm}{15.5 cm}{23.5cm}{0.5cm}{0cm}{1cm}{1cm}

\pagenumbering{arabic}


\begin{document}
\author{Kevin O'Brien}
\title{Term structure of interest rates}
\date{\today}
\maketitle

\tableofcontents \setcounter{tocdepth}{2}


%-----------------------------------------------------------------------------------------%
\chapter{Term structure of interest rates}


\section{Yield Curves}
In finance, the yield curve is the relation between the interest
rate (or cost of borrowing) and the time to maturity of the debt
for a given borrower in a given currency. For example, the U.S.
dollar interest rates paid on U.S. Treasury securities for various
maturities are closely watched by many traders, and are commonly
plotted on a graph such as the one on the right which is
informally called "the yield curve." More formal mathematical
descriptions of this relation are often called the term structure
of interest rates.


\section{Expectations Theory}


\section{Types of yield Curve}
\subsection{Steep yield curve }

Historically, the 20-year Treasury bond yield has averaged
approximately two percentage points above that of three-month
Treasury bills. In situations when this gap increases (e.g.
20-year Treasury yield rises higher than the three-month Treasury
yield), the economy is expected to improve quickly in the future.
This type of curve can be seen at the beginning of an economic
expansion (or after the end of a recession). Here, economic
stagnation will have depressed short-term interest rates; however,
rates begin to rise once the demand for capital is re-established
by growing economic activity. \\In January 2010, the gap between
yields on two-year Treasury notes and 10-year notes widened to
2.90 percentage points, its highest ever. \subsection{Flat or
humped yield curve} A flat yield curve is observed when all
maturities have similar yields, whereas a humped curve results
when short-term and long-term yields are equal and medium-term
yields are higher than those of the short-term and long-term. A
flat curve sends signals of uncertainty in the economy. This mixed
signal can revert to a normal curve or could later result into an
inverted curve. It cannot be explained by the Segmented Market
theory discussed below. \subsection{Inverted yield curve} An
inverted yield curve occurs when long-term yields fall below
short-term yields. Under unusual circumstances, long-term
investors will settle for lower yields now if they think the
economy will slow or even decline in the future. \\Campbell R.
Harvey's 1986 dissertation showed that an inverted yield curve
accurately forecasts U.S. recessions. An inverted curve has
indicated a worsening economic situation in the future 6 out of 7
times since 1970.\\ The New York Federal Reserve regards it as a
valuable forecasting tool in predicting recessions two to six
quarters ahead. In addition to potentially signaling an economic
decline, inverted yield curves also imply that the market believes
inflation will remain low. This is because, even if there is a
recession, a low bond yield will still be offset by low inflation.
However, technical factors, such as a flight to quality or global
economic or currency situations, may cause an increase in demand
for bonds on the long end of the yield curve, causing long-term
rates to fall.

\section{Theories of Maturity}
\begin{itemize}
\item Market expectations (pure expectations) hypothesis \item
Liquidity preference theory \item Market segmentation theory \item
Preferred habitat theory
\end{itemize}

\end{document}
