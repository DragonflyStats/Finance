

Chapter 2: Risk and return: mean–variance analysis and the CAPM
Modern portfolio theory (MPT)

Chapter 2: Risk and return: mean–variance analysis and the CAPM

Introduction
Statistical characteristics of portfolios.
Diversification
Mean–variance analysis. 
The capital asset pricing model 
The Roll critique and empirical tests of the CAPM.


Modern portfolio theory (MPT)
Modern portfolio theory (MPT) is a theory of investment which attempts to maximize portfolio expected return for a given amount of portfolio risk, or equivalently minimize risk for a given level of expected return, by carefully choosing the proportions of various assets. 

Beta

= Covariance of the returns between the industry and the market/variance of the return on the market 

and 

coefficient of correlation = Covariance of returns between the industry and the market/standard deviation of the industry and the market 

we can relate beta to the coefficient of correlation as

Beta = [Coefficient of correlation x standard deviation of the industry]/ standard deviation of the market


Roll's critique
Roll's critique is a famous analysis of the validity of empirical tests of the Capital Asset Pricing Model (CAPM).
