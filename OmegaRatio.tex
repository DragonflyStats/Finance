The Omega Ratio is a risk-return performance measure of an investment asset, portfolio, or strategy. It was devised by Keating & Shadwick in 2002 and is defined as the probability weighted ratio of gains versus losses for some threshold return target.[1] The ratio is an alternative for the widely used Sharpe ratio and is based on information the Sharpe ratio discards.

Omega is calculated by creating a partition in the cumulative return distribution in order to create an area of losses and an area for gains relative to this threshold.

The ratio is calculated as:

{\displaystyle \Omega (r)={\frac {\int _{r}^{\infty }(1-F(x))\,dx}{\int _{-\infty }^{r}F(x)dx}}} \Omega (r)={\frac  {\int _{{r}}^{\infty }(1-F(x))\,dx}{\int _{{-\infty }}^{r}F(x)dx}},
where F is the cumulative distribution function of the returns and r is the target return threshold defining what is considered a gain versus a loss. A larger ratio indicates that the asset provides more gains relative to losses for some threshold r and so would be preferred by an investor. When r is set to zero the Gain-Loss-Ratio by Bernardo and Ledoit arises as a special case.[2]

Comparisons can be made with the commonly used Sharpe ratio which considers the ratio of return versus volatility.[3] The Sharpe ratio considers only the first two moments of the return distribution whereas the Omega ratio, by construction, considers all moments.
