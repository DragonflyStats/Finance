





3











What it is: 


The Sharpe ratio is measure of risk. It is named after Stanford professor and Nobel laureate William F. Sharpe.


How it works (Example): 


The Sharpe ratio is a ratio of return versus risk. The formula is:
(Rp-Rf)/ ?p
 where:

Rp = the expected return on the investor's portfolio
 Rf = the risk-free rate of return
 ?p = the portfolio's standard deviation, a measure of risk

For example, let's assume that you expect your stock portfolio to return 12% next year. If returns on risk-free Treasury notes are, say, 5%, and your portfolio carries a 0.06 standard deviation, then from the formula above we can calculate that the Sharpe ratio for your portfolio is:

(0.12 - 0.05)/0.06 = 1.17

This means that for every point of return, you are shouldering 1.17 "units" of risk.

Put another way, if portfolio X generates a 10% return with a 1.25 Sharpe ratio and portfolio Y also generates a 10% return with a 1.00 Sharpe ratio, then X is the better portfolio because it achieves the same return with less risk.


Why it Matters: 


The higher the Sharpe ratio is, the more return the investor is getting per unit of risk. The lower the Sharpe ratio is, the more risk the investor is shouldering to earn additional returns. Thus, the Sharpe ratio ultimately "levels the playing field" among portfolios by indicating which are shouldering excessive risk.

In addition to relying only on historical returns, one problem with the Sharpe ratio is that illiquid investments lower a portfolio's standard deviation (because those investments appear to be less volatile). The ratio is also distorted if the investments don't have a normal distribution of returns.
